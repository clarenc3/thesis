\section{Building and varying the Monte-Carlo prediction}
T2K has been taking data since 2010 with steadily increasing beam power and protons on target (POT), and is currently on ``run 9'', shown in \autoref{fig:t2k_pot}. This analysis uses data from runs 2 to 6: run 1 was omitted because parts of the detector was uninstrumented (and is only $\sim4\%$ of the run 1-6 data), and run 7 and beyond had not gone through full Monte-Carlo production until summer 2017. 

The overall efficiency of ND280 in runs 2 to 6 was approximately 85\%, collecting 9.7E20 POT out of 12E20 POT. The good POT\footnote{Defined as POT collected when all ND280 sub-detectors and global DAQ are flagged online} per run used in this analysis is listed in \autoref{tab:pot_2017}.

\begin{figure}
	\includegraphics[width=0.5\textwidth, trim={5mm 100mm 5mm 90mm}, clip]{{figures/pow_t2k_all_toRun77full}}
	\caption{T2K protons on target and beam power for run 1-9}
	\label{fig:t2k_pot}
\end{figure}

\begin{table}[h]
	\centering
	\begin{tabular}{ l c c c }
		\hline
		Run & Data POT (E19) & MC POT (E19) & Sand POT (E19) \\
		\hline
		\hline
		2a  FHC & 3.59337    & 92.15       & 10 \\
		2w  FHC & 4.33934    & 120.15      & 10 \\
		\hline
		3b FHC  & 2.17273    & 44.8        & 5 \\
		3c FHC  & 13.6447    & 263         & 25 \\
		\hline
		4a FHC  & 17.8271    & 349.9       & 30 \\
		4w FHC  & 16.4277    & 349.65      & 29.15 \\
		\hline
		5 RHC & 4.3468     & 208.25      & 20 \\
		\hline
		6b RHC & 12.8838    & 141.03      & 40 \\
		6c RHC & 5.07819    & 53.21       & 15 \\
		6d RHC & 7.75302    & 69.41       & 20 \\
		6e RHC & 8.51668    & 86.72       & 23 \\
		\hline
		\hline
		Total FHC & 58.00494 & 1219.65 & 109.15\\
		Total RHC & 38.57849 & 558.62  & 118 \\
		\hline
		Total & 96.58343 & 1778.27 & 227.15 \\
		\hline
	\end{tabular}
	\caption{Counted and generated proton-on-targets for the T2K ND280 2017 analysis. FHC denotes Forward Horn Current (neutrino dominated mode), RHC denotes Reverse Horn Current (anti-neutrino dominated mode)}
	\label{tab:pot_2017}
\end{table}
The Monte-Carlo is generated for the different run-periods to account for beam configurations, ND280 configurations, run-dependent calibrations, and so on. Runs marked ``a'' and ``w'' refer to the P0D detector's removable water bags being air filled (a) or water filled (w), which require different ND280 geometries in simulation.

To build the nominal distribution for direct comparison to data, a number of scalings and weights are applied. For the near-detector portion of MaCh3 and BANFF, all weights are applied on an event-by-event basis:
\begin{itemize}
	\item \textbf{POT weight:} \\
	A run-by-run scaling factor taking the ratio of total good flagged data to the generated Monte-Carlo POT. An event receives a one-time weight depending on what run it was from and how much MC was generated in the production. These numbers can be read off directly from \autoref{tab:pot_2017}. This weight $w_\text{POT}$ is only applied once and is not varied in the fit.
	
	\item \textbf{Flux weight:} \\
	A run-by-run correction to the nominal neutrino flux which the production was made in. The weight is applied as a function of $E_\nu^{true}$ with~$0 < E_\nu^{true} < 30 \text{ GeV}$~\red{show example of this, psycheND280utils/data/tuned13av2}, detailed in \autoref{subsec:syst_flux}. An event receives a weight depending on what run it was from and its $E_\nu^{true}$. The weight $w_\text{Flux}$ is only applied once and is not varied in the fit.
	
	\item \textbf{Beam covariance weight:} \\
	An event-by-event weight to vary the impact of the flux simulation. An event gets weighted as a function of the neutrino running mode (FHC or RHC), the flavour of neutrino ($\nu_\mu$, $\bar{\nu}_\mu$, $\nu_e$ or $\bar{\nu}_e$) and the neutrino energy. There are 100 parameters $\vec{b}$ in the fit---although only 50 being constrained directly by ND280 data, the remaining 50 being the SK flux parameters, constrained solely by their correlations with the ND280 flux parameters---outlined in detail in \autoref{subsec:syst_flux}.
	
	The weight $w_{\vec{b}\rightarrow \vec{b}'}$ is applied once to weight the simulation to nominal, and is recalculated for every iteration of the fit.
	
	\item \textbf{Cross-section weight:} \\
	An event-by-event weight taking the full generated event through the neutrino interaction simulation, calculating a weight to apply to the event. The applied weight is calculated as $w_{\vec{x} \rightarrow \vec{x}'}=\sigma(\vec{x}')/\sigma(\vec{x})$ for the generated neutrino interaction systematic parameter values $\vec{x}$ and the modified parameter values $\vec{x}'$. The event can receive a normalisation and a shape parameter depending on its interaction type and the interaction model considered in the analysis. Details of the model are found in \autoref{subsec:syst_xsec}.
	
	The weight $w_{\vec{x}\rightarrow \vec{x}'}$ is applied once to weight the simulation to nominal, and is recalculated for every iteration of the fit.
	
	\item \textbf{Detector covariance weight:} \\
	An event-by-event weight from the reconstruction and selection package to vary the impact of the detector simulation. It is applied as a function of the event's topology and detector (e.g. FGD1 CC0$\pi$) and which detector covariance bin the event bin falls into (e.g. $p_\mu=350\text{ MeV}, \cos\theta_\mu=0.99$). The weight is a normalisation parameter $\vec{d}$ for each bin in the detector covariance matrix scheme, explained in detail in \autoref{subsec:syst_nd280}.
	
	The weight $w_{\vec{d}\rightarrow \vec{d}'}$ is applied once to weight the simulation to nominal, and is recalculated for every iteration of the fit.
\end{itemize}

All weights are parameterised as multiplicative, so for a two parameter variation $x \rightarrow x'$ and $y \rightarrow y'$ we have the total weight 
\begin{equation}
w_{x \rightarrow x', y \rightarrow y'} = w_{x \rightarrow x', y \rightarrow y} \times w_{x \rightarrow x, y\rightarrow y'}
\end{equation}

Defining the beam parameters as $\vec{b}$, cross-section parameters as $\vec{x}$ and detector parameters as $\vec{d}$, we express one rescaled Monte-Carlo event $i$ as

\begin{equation}
\lambda_i\left(\vec{b}, \vec{x}, \vec{d}\right) = 1 \times w_i^\text{POT} \times w_i^\text{Flux} \times w^{\vec{b}\rightarrow \vec{b}'}_i \times w^{\vec{x} \rightarrow \vec{x}'}_i \times w^{\vec{d}\rightarrow \vec{d}'}_i
\label{eq:mc_scale}
\end{equation}

The effect of each weight on the MC sample statistics is shown in \autoref{tab:eventrate_mach3}. The nominal flux and detector weights are 1.0, so are not included in the table
\begin{sidewaystable}
	\centering
	\begin{tabular}{ l | c | c | c | c | c }
		\hline
		Sample & Raw MC & POT only & POT+xsec & POT+NDCov & POT+BeamCov \\ 
		\hline
		\hline
		\FGDCCNoPi{1}{\numu}& 337436 & 15905.2 & 15340.2 & 16246.1 & 16090.8 \\
		\FGDCCOnePi{1}{\numu}& 84982  & 4011.58 & 3819.3 & 4131.35 & 4058.36 \\
		\FGDCCOther{1}{\numu}& 65286 & 3071.21 & 3078.52 & 3374.21 & 3107.04 \\
		\FGDCCNoPi{2}{\numu}& 345467 & 16259.6 & 15749.8 & 16415 & 16449.2 \\
		\FGDCCOnePi{2}{\numu}& 70444  & 3318.26 & 3190.09 & 3321.4 & 3356.97 \\
		\FGDCCOther{2}{\numu}& 63402  & 2983.43 & 2995.68 & 3051.48 & 3018.22 \\
		\FGDCCOneTrk{1}{\numubar}& 54419  & 3744.02 & 3430.15 & 3872.18 & 3773.79 \\
		\FGDCCNTrk{1}{\numubar}& 15392  & 1056.8 & 986.32 & 1121.1 & 1065.2 \\
		\FGDCCOneTrk{2}{\numubar}& 55732  & 3833.73 & 3506.36 & 3906.32 & 3864.24 \\
		\FGDCCNTrk{2}{\numubar}& 15808  & 1088.15 & 1024.77 & 1122.73 & 1096.81 \\
		\FGDCCOneTrk{1}{\numu} in RHC& 18146 & 1246.03 & 1194.72 & 1262.00 & 1255.92 \\
		\FGDCCNTrk{1}{\numu} in RHC& 17156 & 1181.54 & 1172.13 & 1257.34 & 1190.94 \\
		\FGDCCOneTrk{2}{\numu} in RHC& 18052 & 1231.24 & 1189.64 & 1245.3 & 1241.02 \\
		\FGDCCNTrk{2}{\numu} in RHC& 16339 & 1127.64 & 1121.37 & 1150.7 & 1136.62 \\
		\hline
		\hline
	\end{tabular}
	\caption{Event rates broken by type of weight applied for the nominal MC samples}
	\label{tab:eventrate_mach3}
\end{sidewaystable}

\subsection{Defining the test-statistic}
As mentioned in \autoref{sec:ND280:sel}, data and simulation events are binned in \pmu, \cosmu for 14 different selections (seven per FGD): \numu CC0$\pi$, \numu CC1$\pi$, \numu CCOther, \numubar CC1Track, \numubar CCNTrack, \numu in RHC CC1Track and \numu in RHC CCNTrack. It is assumed that the data $n^{\alpha}$ in each bin $\alpha$, is a statistical fluctuation of the simulation in that bin, $\lambda^{\alpha}$, and the probability distribution to observe $n^{\alpha}$ given $\lambda^\alpha$ is Poissonian:
\begin{equation}
P(n^\alpha|\lambda^\alpha) = \frac{e^{-\lambda^\alpha} \left(\lambda^\alpha\right)^n}{\left(n^\alpha\right)!}
\end{equation}
The Monte-Carlo prediction $\lambda$ is influenced by the beam parameters $\vec{b}$, the cross-section parameters $\vec{x}$ and the ND280 detector parameters $vec{d}$ outlined in \autoref{eq:mc_scale}. Then the statistical contribution to the negative log-likelihood per \pmu \cosmu bin $\alpha$ is
\begin{equation}
-\log\mathcal{L}^\alpha_\text{Sample} = \lambda^\alpha(\vec{b},\vec{x},\vec{d}) - n^\alpha + n^\alpha\log\frac{n^\alpha}{\lambda^\alpha(\vec{b},\vec{x},\vec{d})}
\end{equation}

The majority of systematics in \autoref{sec:syst} are constrained by prior information outside of the ND280 data used in the fit. The constraints may come from calibration studies, external cross-section data, beam simulations, and so on. A covariance matrix $\boldsymbol{V}_{i,j}$ encoding the correlations and uncertainties of parameters $i$ to $j$ is used for the separate sources of systematics. If the central values from external constraints (priors) is $\vec{\mu}$ and the parameter values varied in the fit are $\vec{X}$, we assume a Gaussian probability density function for each parameter, giving rise to a negative log-likelihood contribution for systematic $K$
\begin{equation}
-\log\mathcal{L}^K_\text{Syst} = \frac{1}{2}( X_i - \mu_i ) \left(\boldsymbol{V}^K\right)^{-1}_{i,j} ( X_j - \mu_j )
\end{equation}
and for our three separate sources of systematics we then have in total
\begin{equation}
-\log\mathcal{L_\text{Syst}} = -\left(\log\mathcal{L_\text{Flux}} + \log\mathcal{L_\text{Interaction}} + \log\mathcal{L_\text{ND280}}\right)
\end{equation}
Finally, our combined test-statistic for \pmu, \cosmu bins $\alpha$ over samples $S$, using our three sets of systematics (flux with parameters $\vec{b}$, cross-section with parameters $\vec{x}$, ND280 with parameters $\vec{d}$), is
\begin{equation}
\label{eq:test_stat}
\begin{split}
-\log\mathcal{L_\text{Total}} & =  \sum_S^\text{Samples} \left( \sum_{\alpha}^\text{Bins} \left( \lambda^\alpha(\vec{b},\vec{x},\vec{d}) - n^\alpha + n^\alpha\log\frac{n^\alpha}{\lambda^\alpha(\vec{b},\vec{x},\vec{d})} \right) \right)_S \\
 & + \frac{1}{2} \left( \sum_{i,j}^\text{Flux} ( b_i - \mu_i ) \left(\boldsymbol{V}^b\right)^{-1}_{i,j} ( b_j - \mu_j ) \right) \\
 & + \frac{1}{2} \left( \sum_{i,j}^\text{Interaction} ( x_i - \mu_i ) \left(\boldsymbol{V}^x\right)^{-1}_{i,j} ( x_j - \mu_j ) \right) \\
 & + \frac{1}{2} \left( \sum_{i,j}^\text{ND280} ( d_i - \mu_i ) \left(\boldsymbol{V}^d\right)^{-1}_{i,j} ( d_j - \mu_j ) \right)
\end{split}
\end{equation}

\section{Parameters of interest}
flux and cross-section

\section{Fitting method: MCMC}
tests

\section{Nominal model}
\label{sec:nom_model}
The rates for the data and nominal model with all mentioned Monte-Carlo scalings and selections are presented in \autoref{tab:event_rates_2017}. We note that generally the Monte-Carlo rates of the CC0$\pi$ and CC1Track are underestimated (2-3\%) , CC$1\pi$ is overestimated (6-11\%) and CCOther is under-estimated (5-15\%). \numubar selections except FGD2 CC1Track \numubar are overestimated by 5\%, with FGD1 being 2\%. The \numu in RHC selections are mildly underestimated. The rates across the FGDs are consistent for all selections.
\begin{table}[h]
	\centering
	\begin{tabular}{ l | c c c }
		\hline
		\hline
		Sample & Data & Nominal MC & Data/MC \\
		\hline
		\FGDCCNoPi{1}{\numu}           & 17136 & 16723.80 & 1.02 \\% \hline
		\FGDCCOnePi{1}{\numu}          & 3954  & 4381.47 & 0.90 \\% \hline
		\FGDCCOther{1}{\numu}          & 4149  & 3943.95 & 1.05\\% \hline
		\hline
		\FGDCCNoPi{2}{\numu}           & 17443 & 16959.30 & 1.03 \\% \hline
		\FGDCCOnePi{2}{\numu}          & 3366  & 3564.23  & 0.94\\% \hline
		\FGDCCOther{2}{\numu}          & 4075  & 3570.94  & 1.14 \\% \hline
		\hline
		\FGDCCOneTrk{1}{\numubar}      & 3527 & 3587.77 & 0.98 \\% \hline
		\FGDCCNTrk{1}{\numubar}   	   & 1054 & 1066.91 & 0.99 \\% \hline
		\hline
		\FGDCCOneTrk{2}{\numubar}      & 3732 & 3618.29 & 1.03 \\% \hline
		\FGDCCNTrk{2}{\numubar}        & 1026 & 1077.24 & 0.95\\% \hline
		\hline
		\FGDCCOneTrk{1}{\numu} in RHC  & 1363 & 1272.17 & 1.07 \\% \hline
		\FGDCCNTrk{1}{\numu} in RHC    & 1370 & 1357.45 & 1.01 \\% \hline
		\hline
		\FGDCCOneTrk{2}{\numu} in RHC  & 1320 & 1262.63 & 1.05 \\% \hline
		\FGDCCNTrk{2}{\numu} in RHC    & 1253 & 1246.71 & 1.01\\% \hline
		\hline
		Total & 64768 & 63632.86 & 0.98 \\\hline
		\hline
	\end{tabular}
	\caption{Observed and predicted event rates for the different \nd~samples in the near-detector}
	\label{tab:event_rates_2017}
\end{table}

\autoref{fig:nominal2D_FGD1numu} shows the \pmu \cosmu distributions and their ratios for FGD1, normalised to bin width. For FGD1 CC0$\pi$, the largest Data/MC discrepancies are in the very forward region around 500-1000 MeV/c, with some areas of low cross-section  (e.g. \pmu 2-5 GeV/c, \cosmu 0.85-0.9) mismodelled. The lines of constant $Q^2$ suggest high $Q^2$ behaviour is over-estimated in Monte-Carlo, whereas for $0.05 < Q^2 < 0.15 \text{ GeV}^2$ it is under-estimated. For CC1$\pi$ the most forward-goign bins are almost consistently under-estimated. As for CC0$\pi$, the $Q^2 >0.1 \text{GeV}^2$ region is over-estimated, but it is less clear at lower $Q^2$. For CCOther there is a band-like behaviour in $Q^2$ going from over estimation to underestimation up until $Q^2\sim0.1\text{ GeV}^2$. The high-momentum areas are mostly under-estimated in Monte-Carlo. It is also clear that ND280 are dominated by $0.05 < Q^2 < 0.30\text{ GeV}^2$ events.

\begin{figure}[h]
	\begin{subfigure}[t]{0.32\textwidth}
		\includegraphics[width=\textwidth,page=1]{{figures/mach3/selection/2017b_nominal_withdebug_forthesis_ND280_nom.pdf}}
	\end{subfigure}
	\begin{subfigure}[t]{0.32\textwidth}
		\includegraphics[width=\textwidth,page=2]{{figures/mach3/selection/2017b_nominal_withdebug_forthesis_ND280_nom.pdf}}
	\end{subfigure}
	\begin{subfigure}[t]{0.32\textwidth}
		\includegraphics[width=\textwidth,page=3]{{figures/mach3/selection/2017b_nominal_withdebug_forthesis_ND280_nom.pdf}}
	\end{subfigure}

	\begin{subfigure}[t]{0.32\textwidth}
		\includegraphics[width=\textwidth,page=4]{{figures/mach3/selection/2017b_nominal_withdebug_forthesis_ND280_nom.pdf}}
	\end{subfigure}
	\begin{subfigure}[t]{0.32\textwidth}
		\includegraphics[width=\textwidth,page=5]{{figures/mach3/selection/2017b_nominal_withdebug_forthesis_ND280_nom.pdf}}
	\end{subfigure}
	\begin{subfigure}[t]{0.32\textwidth}
		\includegraphics[width=\textwidth,page=6]{{figures/mach3/selection/2017b_nominal_withdebug_forthesis_ND280_nom.pdf}}
	\end{subfigure}

	\begin{subfigure}[t]{0.32\textwidth}
		\includegraphics[width=\textwidth,page=7]{{figures/mach3/selection/2017b_nominal_withdebug_forthesis_ND280_nom.pdf}}
	\end{subfigure}
	\begin{subfigure}[t]{0.32\textwidth}
		\includegraphics[width=\textwidth,page=8]{{figures/mach3/selection/2017b_nominal_withdebug_forthesis_ND280_nom.pdf}}
	\end{subfigure}
	\begin{subfigure}[t]{0.32\textwidth}
		\includegraphics[width=\textwidth,page=9]{{figures/mach3/selection/2017b_nominal_withdebug_forthesis_ND280_nom.pdf}}
	\end{subfigure}

\caption{Data and nominal MC distributions and the Data/MC ratio for FGD1 \numu selections. Lines of constant $Q^2_\text{reco}$ are shown. Bin content is normalised to bin width.}
\label{fig:nominal2D_FGD1numu}
\end{figure}

\autoref{fig:nominal2D_FGD2numu} shows the same \numu selections for FGD2. The CC0$\pi$ selection is very similar to the FGD1 CC0$\pi$ selection, whereas the CC1$\pi$ selection appears better modelled for FGD2 than FGD1, although the opposite is true for CCOther.

\begin{figure}[h]
	\begin{subfigure}[t]{0.32\textwidth}
		\includegraphics[width=\textwidth,page=10]{{figures/mach3/selection/2017b_nominal_withdebug_forthesis_ND280_nom.pdf}}
	\end{subfigure}
	\begin{subfigure}[t]{0.32\textwidth}
		\includegraphics[width=\textwidth,page=11]{{figures/mach3/selection/2017b_nominal_withdebug_forthesis_ND280_nom.pdf}}
	\end{subfigure}
	\begin{subfigure}[t]{0.32\textwidth}
		\includegraphics[width=\textwidth,page=12]{{figures/mach3/selection/2017b_nominal_withdebug_forthesis_ND280_nom.pdf}}
	\end{subfigure}

	\begin{subfigure}[t]{0.32\textwidth}
		\includegraphics[width=\textwidth,page=13]{{figures/mach3/selection/2017b_nominal_withdebug_forthesis_ND280_nom.pdf}}
	\end{subfigure}
	\begin{subfigure}[t]{0.32\textwidth}
		\includegraphics[width=\textwidth,page=14]{{figures/mach3/selection/2017b_nominal_withdebug_forthesis_ND280_nom.pdf}}
	\end{subfigure}
	\begin{subfigure}[t]{0.32\textwidth}
		\includegraphics[width=\textwidth,page=15]{{figures/mach3/selection/2017b_nominal_withdebug_forthesis_ND280_nom.pdf}}
	\end{subfigure}

	\begin{subfigure}[t]{0.32\textwidth}
		\includegraphics[width=\textwidth,page=16]{{figures/mach3/selection/2017b_nominal_withdebug_forthesis_ND280_nom.pdf}}
	\end{subfigure}
	\begin{subfigure}[t]{0.32\textwidth}
		\includegraphics[width=\textwidth,page=17]{{figures/mach3/selection/2017b_nominal_withdebug_forthesis_ND280_nom.pdf}}
	\end{subfigure}
	\begin{subfigure}[t]{0.32\textwidth}
		\includegraphics[width=\textwidth,page=18]{{figures/mach3/selection/2017b_nominal_withdebug_forthesis_ND280_nom.pdf}}
	\end{subfigure}

\caption{Data and nominal MC distributions and the Data/MC ratio for FGD2 \numu selections. Lines of constant $Q^2_\text{reco}$ are shown. Bin content is normalised to bin width.}
\label{fig:nominal2D_FGD2numu}
\end{figure}

\autoref{fig:nominal2D_FGD12numubar} shows the \pmu \cosmu for the \numubar selection, which again sees mostly consistent behaviour for the two FGDs for both the 1Track and NTracks selection. The event are mostly underestimated at low \pmu and become overestimated as we go up in \cosmu. The $Q^2$ bands appear present, notably in the 1 Track selections for $0.05 < Q^2 < 0.10 \text{ GeV}^2$.

\begin{figure}
	\begin{subfigure}[t]{0.32\textwidth}
		\includegraphics[width=\textwidth,page=19]{{figures/mach3/selection/2017b_nominal_withdebug_forthesis_ND280_nom.pdf}}
	\end{subfigure}
	\begin{subfigure}[t]{0.32\textwidth}
		\includegraphics[width=\textwidth,page=20]{{figures/mach3/selection/2017b_nominal_withdebug_forthesis_ND280_nom.pdf}}
	\end{subfigure}
	\begin{subfigure}[t]{0.32\textwidth}
		\includegraphics[width=\textwidth,page=21]{{figures/mach3/selection/2017b_nominal_withdebug_forthesis_ND280_nom.pdf}}
	\end{subfigure}

	\begin{subfigure}[t]{0.32\textwidth}
		\includegraphics[width=\textwidth,page=22]{{figures/mach3/selection/2017b_nominal_withdebug_forthesis_ND280_nom.pdf}}
	\end{subfigure}
	\begin{subfigure}[t]{0.32\textwidth}
		\includegraphics[width=\textwidth,page=23]{{figures/mach3/selection/2017b_nominal_withdebug_forthesis_ND280_nom.pdf}}
	\end{subfigure}
	\begin{subfigure}[t]{0.32\textwidth}
		\includegraphics[width=\textwidth,page=24]{{figures/mach3/selection/2017b_nominal_withdebug_forthesis_ND280_nom.pdf}}
	\end{subfigure}

	\begin{subfigure}[t]{0.32\textwidth}
		\includegraphics[width=\textwidth,page=25]{{figures/mach3/selection/2017b_nominal_withdebug_forthesis_ND280_nom.pdf}}
	\end{subfigure}
	\begin{subfigure}[t]{0.32\textwidth}
		\includegraphics[width=\textwidth,page=26]{{figures/mach3/selection/2017b_nominal_withdebug_forthesis_ND280_nom.pdf}}
	\end{subfigure}
	\begin{subfigure}[t]{0.32\textwidth}
		\includegraphics[width=\textwidth,page=27]{{figures/mach3/selection/2017b_nominal_withdebug_forthesis_ND280_nom.pdf}}
	\end{subfigure}

	\begin{subfigure}[t]{0.32\textwidth}
		\includegraphics[width=\textwidth,page=28]{{figures/mach3/selection/2017b_nominal_withdebug_forthesis_ND280_nom.pdf}}
	\end{subfigure}
	\begin{subfigure}[t]{0.32\textwidth}
		\includegraphics[width=\textwidth,page=29]{{figures/mach3/selection/2017b_nominal_withdebug_forthesis_ND280_nom.pdf}}
	\end{subfigure}
	\begin{subfigure}[t]{0.32\textwidth}
		\includegraphics[width=\textwidth,page=30]{{figures/mach3/selection/2017b_nominal_withdebug_forthesis_ND280_nom.pdf}}
	\end{subfigure}
\caption{Data and nominal MC distributions and the Data/MC ratio for FGD1 and FGD2 \numubar selections. Lines of constant $Q^2_\text{reco}$ are shown. Bin content is normalised to bin width.}
\label{fig:nominal2D_FGD12numubar}
\end{figure}

\autoref{fig:nominal2D_FGD12numuRHC} shows the \numu in RHC selections, which generally populate higher \pmu due to the \numu flux in RHC mode. The samples are also statistically small so are more prone to larger statistical fluctuations. The CC1Track selection appears to have a pattern of underestimation at low \pmu, following the $Q^2$ band to high \pmu and high \cosmu.

\begin{figure}
	\begin{subfigure}[t]{0.32\textwidth}
		\includegraphics[width=\textwidth,page=31]{{figures/mach3/selection/2017b_nominal_withdebug_forthesis_ND280_nom.pdf}}
	\end{subfigure}
	\begin{subfigure}[t]{0.32\textwidth}
		\includegraphics[width=\textwidth,page=32]{{figures/mach3/selection/2017b_nominal_withdebug_forthesis_ND280_nom.pdf}}
	\end{subfigure}
	\begin{subfigure}[t]{0.32\textwidth}
		\includegraphics[width=\textwidth,page=33]{{figures/mach3/selection/2017b_nominal_withdebug_forthesis_ND280_nom.pdf}}
	\end{subfigure}

	\begin{subfigure}[t]{0.32\textwidth}
		\includegraphics[width=\textwidth,page=34]{{figures/mach3/selection/2017b_nominal_withdebug_forthesis_ND280_nom.pdf}}
	\end{subfigure}
	\begin{subfigure}[t]{0.32\textwidth}
		\includegraphics[width=\textwidth,page=35]{{figures/mach3/selection/2017b_nominal_withdebug_forthesis_ND280_nom.pdf}}
	\end{subfigure}
	\begin{subfigure}[t]{0.32\textwidth}
		\includegraphics[width=\textwidth,page=36]{{figures/mach3/selection/2017b_nominal_withdebug_forthesis_ND280_nom.pdf}}
	\end{subfigure}

	\begin{subfigure}[t]{0.32\textwidth}
		\includegraphics[width=\textwidth,page=37]{{figures/mach3/selection/2017b_nominal_withdebug_forthesis_ND280_nom.pdf}}
	\end{subfigure}
	\begin{subfigure}[t]{0.32\textwidth}
		\includegraphics[width=\textwidth,page=38]{{figures/mach3/selection/2017b_nominal_withdebug_forthesis_ND280_nom.pdf}}
	\end{subfigure}
	\begin{subfigure}[t]{0.32\textwidth}
		\includegraphics[width=\textwidth,page=39]{{figures/mach3/selection/2017b_nominal_withdebug_forthesis_ND280_nom.pdf}}
	\end{subfigure}

	\begin{subfigure}[t]{0.32\textwidth}
		\includegraphics[width=\textwidth,page=40]{{figures/mach3/selection/2017b_nominal_withdebug_forthesis_ND280_nom.pdf}}
	\end{subfigure}
	\begin{subfigure}[t]{0.32\textwidth}
		\includegraphics[width=\textwidth,page=41]{{figures/mach3/selection/2017b_nominal_withdebug_forthesis_ND280_nom.pdf}}
	\end{subfigure}
	\begin{subfigure}[t]{0.32\textwidth}
		\includegraphics[width=\textwidth,page=42]{{figures/mach3/selection/2017b_nominal_withdebug_forthesis_ND280_nom.pdf}}
	\end{subfigure}
\caption{Data and nominal MC distributions and the Data/MC ratio for FGD1 and FGD2 \numu in RHC selections. Lines of constant $Q^2_\text{reco}$ are shown. Bin content is normalised to bin width.}
\label{fig:nominal2D_FGD12numurhc}
\end{figure}

\red{fix z axis is \autoref{fig:nominal2D_FGD12numurhc}}

\autoref{fig:nominal1D_pmu} shows the projections of the 2D distributions onto \pmu, where we note generally good modelling. The CC0$\pi$ selections have a clear oscillation from underprediction to overprediction for FGD1 and FGD2 in $0 < p_\mu < 1 \text{ GeV}$. The FGD1 \numubar CC1Track selection appears to show similar behaviour although without the under-prediction at low \pmu, although FGD2 \numu CC1Track does not. The CC1$\pi$ selection shows a nearly consistent over-estimation of $\sim10\%$ in every bin for both FGDs. The CCOther selection is instead underestimated at the event distribution peak, into the tail ($0.5<p_\mu <1.5\text{ GeV}$). The CCNtrack distributions consistent with good modelling due to their large statistical errors.

\begin{figure}
\begin{subfigure}[t]{0.24\textwidth}
	\includegraphics[width=\textwidth,page=43]{{figures/mach3/selection/2017b_nominal_withdebug_forthesis_ND280_nom.pdf}}
\end{subfigure}
\begin{subfigure}[t]{0.24\textwidth}
	\includegraphics[width=\textwidth,page=44]{{figures/mach3/selection/2017b_nominal_withdebug_forthesis_ND280_nom.pdf}}
\end{subfigure}
\begin{subfigure}[t]{0.24\textwidth}
	\includegraphics[width=\textwidth,page=46]{{figures/mach3/selection/2017b_nominal_withdebug_forthesis_ND280_nom.pdf}}
\end{subfigure}
\begin{subfigure}[t]{0.24\textwidth}
	\includegraphics[width=\textwidth,page=48]{{figures/mach3/selection/2017b_nominal_withdebug_forthesis_ND280_nom.pdf}}
\end{subfigure}


\begin{subfigure}[t]{0.24\textwidth}
	\includegraphics[width=\textwidth,page=50]{{figures/mach3/selection/2017b_nominal_withdebug_forthesis_ND280_nom.pdf}}
\end{subfigure}
\begin{subfigure}[t]{0.24\textwidth}
	\includegraphics[width=\textwidth,page=52]{{figures/mach3/selection/2017b_nominal_withdebug_forthesis_ND280_nom.pdf}}
\end{subfigure}
\begin{subfigure}[t]{0.24\textwidth}
	\includegraphics[width=\textwidth,page=54]{{figures/mach3/selection/2017b_nominal_withdebug_forthesis_ND280_nom.pdf}}
\end{subfigure}

\begin{subfigure}[t]{0.24\textwidth}
	\includegraphics[width=\textwidth,page=56]{{figures/mach3/selection/2017b_nominal_withdebug_forthesis_ND280_nom.pdf}}
\end{subfigure}
\begin{subfigure}[t]{0.24\textwidth}
	\includegraphics[width=\textwidth,page=58]{{figures/mach3/selection/2017b_nominal_withdebug_forthesis_ND280_nom.pdf}}
\end{subfigure}
\begin{subfigure}[t]{0.24\textwidth}
	\includegraphics[width=\textwidth,page=60]{{figures/mach3/selection/2017b_nominal_withdebug_forthesis_ND280_nom.pdf}}
\end{subfigure}
\begin{subfigure}[t]{0.24\textwidth}
	\includegraphics[width=\textwidth,page=62]{{figures/mach3/selection/2017b_nominal_withdebug_forthesis_ND280_nom.pdf}}
\end{subfigure}

\begin{subfigure}[t]{0.24\textwidth}
	\includegraphics[width=\textwidth,page=64]{{figures/mach3/selection/2017b_nominal_withdebug_forthesis_ND280_nom.pdf}}
\end{subfigure}
\begin{subfigure}[t]{0.24\textwidth}
	\includegraphics[width=\textwidth,page=66]{{figures/mach3/selection/2017b_nominal_withdebug_forthesis_ND280_nom.pdf}}
\end{subfigure}
\begin{subfigure}[t]{0.24\textwidth}
	\includegraphics[width=\textwidth,page=68]{{figures/mach3/selection/2017b_nominal_withdebug_forthesis_ND280_nom.pdf}}
\end{subfigure}
\begin{subfigure}[t]{0.24\textwidth}
	\includegraphics[width=\textwidth,page=70]{{figures/mach3/selection/2017b_nominal_withdebug_forthesis_ND280_nom.pdf}}
\end{subfigure}
\caption{Data and nominal MC distributions selections projected onto \pmu, showing contributions by interaction mode. Bin content is normalised to bin width.}
\label{fig:nominal1D_pmu}
\end{figure}

\autoref{fig:nominal1D_cosmu} shows the projections of the 2D distributions onto \cosmu. Again we see consistency across the FGDs, with CC0$\pi$ showing another oscillatory behaviour, going from underestimation at low \cosmu to a good prediction until $\cos \theta_\mu \sim 0.93$, in which the underestimation is back, similar in magnitude. For CC1$\pi$ we see a similar oscillation but shifted by 10\% over-estimation. For \numu CCOther we have less consistency, although the bin above $\cos \theta_\mu = 0.93$ are all underestimated, and for FGD2 this continues as \cosmu decreases. The most forward bin appears to be good modelled for all \numu samples except FGD2 CCOther. For the RHC 1 track selections, we note FGD2 looking similar to the CC0$\pi$ selection, where FGD1 less so. The NTracks selections look similar for FGD1 and FGD2 with overestimates at high \cosmu. For RHC \numu selections, the NTrack selections appear more consistent than 1 track, with underestimates in the highest \cosmu bin. The 1 track appears consistently underestimated in $0.9 < \cos\theta_\mu < 1$.

\begin{figure}
	\begin{subfigure}[t]{0.24\textwidth}
		\includegraphics[width=\textwidth,page=43]{{figures/mach3/selection/2017b_nominal_withdebug_forthesis_ND280_nom.pdf}}
	\end{subfigure}
	\begin{subfigure}[t]{0.24\textwidth}
		\includegraphics[width=\textwidth,page=45]{{figures/mach3/selection/2017b_nominal_withdebug_forthesis_ND280_nom.pdf}}
	\end{subfigure}
	\begin{subfigure}[t]{0.24\textwidth}
		\includegraphics[width=\textwidth,page=47]{{figures/mach3/selection/2017b_nominal_withdebug_forthesis_ND280_nom.pdf}}
	\end{subfigure}
	\begin{subfigure}[t]{0.24\textwidth}
		\includegraphics[width=\textwidth,page=49]{{figures/mach3/selection/2017b_nominal_withdebug_forthesis_ND280_nom.pdf}}
	\end{subfigure}
	
	
	\begin{subfigure}[t]{0.24\textwidth}
		\includegraphics[width=\textwidth,page=51]{{figures/mach3/selection/2017b_nominal_withdebug_forthesis_ND280_nom.pdf}}
	\end{subfigure}
	\begin{subfigure}[t]{0.24\textwidth}
		\includegraphics[width=\textwidth,page=53]{{figures/mach3/selection/2017b_nominal_withdebug_forthesis_ND280_nom.pdf}}
	\end{subfigure}
	\begin{subfigure}[t]{0.24\textwidth}
		\includegraphics[width=\textwidth,page=55]{{figures/mach3/selection/2017b_nominal_withdebug_forthesis_ND280_nom.pdf}}
	\end{subfigure}
	
	\begin{subfigure}[t]{0.24\textwidth}
		\includegraphics[width=\textwidth,page=57]{{figures/mach3/selection/2017b_nominal_withdebug_forthesis_ND280_nom.pdf}}
	\end{subfigure}
	\begin{subfigure}[t]{0.24\textwidth}
		\includegraphics[width=\textwidth,page=59]{{figures/mach3/selection/2017b_nominal_withdebug_forthesis_ND280_nom.pdf}}
	\end{subfigure}
	\begin{subfigure}[t]{0.24\textwidth}
		\includegraphics[width=\textwidth,page=61]{{figures/mach3/selection/2017b_nominal_withdebug_forthesis_ND280_nom.pdf}}
	\end{subfigure}
	\begin{subfigure}[t]{0.24\textwidth}
		\includegraphics[width=\textwidth,page=63]{{figures/mach3/selection/2017b_nominal_withdebug_forthesis_ND280_nom.pdf}}
	\end{subfigure}
	
	\begin{subfigure}[t]{0.24\textwidth}
		\includegraphics[width=\textwidth,page=65]{{figures/mach3/selection/2017b_nominal_withdebug_forthesis_ND280_nom.pdf}}
	\end{subfigure}
	\begin{subfigure}[t]{0.24\textwidth}
		\includegraphics[width=\textwidth,page=67]{{figures/mach3/selection/2017b_nominal_withdebug_forthesis_ND280_nom.pdf}}
	\end{subfigure}
	\begin{subfigure}[t]{0.24\textwidth}
		\includegraphics[width=\textwidth,page=69]{{figures/mach3/selection/2017b_nominal_withdebug_forthesis_ND280_nom.pdf}}
	\end{subfigure}
	\begin{subfigure}[t]{0.24\textwidth}
		\includegraphics[width=\textwidth,page=71]{{figures/mach3/selection/2017b_nominal_withdebug_forthesis_ND280_nom.pdf}}
	\end{subfigure}
	\caption{Data and nominal MC distributions selections projected onto \cosmu, showing contributions by interaction mode. Bin content is normalised to bin width.}
	\label{fig:nominal1D_cosmu}
\end{figure}

The \pmu and \cosmu mode distributions in \autoref{fig:nominal1D_pmu} and \autoref{fig:nominal1D_cosmu} after applying the weights scaling show no noticeable differences to the raw MC distributions in \autoref{fig:nominal_mcpmu} and \autoref{fig:nominal_mccosmu}. Comparing the sample breakdown in \autoref{tab:nominal_mode_afterscale} to \autoref{tab:nominal_mode} shows the largest effect being CCQE 1.3\% units for FGD1 0$\pi$ and CCDIS for FGD1 and FGD2 (also impacting the NTrk selections).

\begin{table}[h]
	\centering
  \begin{tabular}{l | c c c c c c c }
    \hline
    \hline
      Sample	      & CCQE & 2p2h & CC1$\pi^{\pm,0}$ 	& CC coh 	& CC multi-$\pi$ & CC DIS  	& NC \\
      \hline
      FGD1 $0\pi$     & \textbf{56.7} & \textbf{10.0} & 19.8 & 0.3 & 4.5 & 5.1 & 3.6 \\
      FGD2 0$\pi$     & \textbf{55.2} & \textbf{9.4} & 21.5 & 0.3 & 4.9 & 5.2 & 3.4 \\
      \hline
      FGD1 $1\pi$     & 5.5 & 0.8 & \textbf{49.2} & \textbf{2.7} & \textbf{18.1} & 17.2 & 6.6 \\
      FGD2 1$\pi$     & 5.6 & 0.7 & \textbf{48.5} & \textbf{2.8} & \textbf{18.3} & 17.6 & 6.4 \\
      \hline
      FGD1 Other      & 4.7 & 1.0 & 14.9 & 0.4 & \textbf{26.1} & \textbf{45.0} & 8.1 \\
      FGD2 Other      & 4.9 & 1.0 & 15.5 & 0.3 & \textbf{25.5} & \textbf{44.9} & 7.9 \\
      \hline
      FGD1 1Trk     & \textbf{64.0} & \textbf{10.0} & 14.7 & 0.7 & 2.8 & 2.5 & 5.1 \\
      FGD2 1Trk     & \textbf{64.4} & \textbf{9.9} & 14.7 & 0.7 & 2.8 & 2.6 & 4.9 \\
      \hline
      FGD1 NTrk     & 7.6 & 2.6 & \textbf{28.4} & \textbf{3.3} & \textbf{19.8} & \textbf{26.9} & 11.3 \\
      FGD2 NTrk     & 8.3 & 2.7 & \textbf{28.3} & \textbf{3.2} & \textbf{19.8} & \textbf{26.2} & 11.6 \\
      \hline
      FGD1 1Trk \numu & \textbf{43.5} & \textbf{8.2} & 25.2 & 0.9 & 7.2 & 6.6 & 8.5 \\
      FGD2 1Trk \numu & \textbf{43.3} & \textbf{8.2} & 25.2 & 0.8 & 7.8 & 7.1 & 7.7 \\
      \hline
      FGD1 NTrk \numu & 12.1 & 3.1 & \textbf{28.5} & \textbf{1.8} & \textbf{20.8} & \textbf{26.6} & 7.0 \\
      FGD2 1Trk \numu & 11.9 & 2.7 & \textbf{29.8} & \textbf{1.8} & \textbf{21.1} & \textbf{26.4} & 6.4 \\
      \hline
      \hline
  \end{tabular}
\caption{Percentage mode breakdown for the binned nominal \textbf{scaled} Monte-Carlo samples, \textbf{boldface} indicates interactions targeted by specific selections. The distributions are \textbf{not} bin-width normalised. Compare to \autoref{tab:nominal_mode} for effect of weights.}
\label{tab:nominal_mode_afterscale}
\end{table}

\section{Asimov results}
\label{sec:asimov}
To internally validate the implementation and evaluate the effectiveness of the fitting framework we perform studies in which the nominal Monte-Carlo predictions presented in \autoref{sec:nom_model} are set to be the data.

\subsection{Log-likelihood scan}
\label{sec:llh_scan}
For the log-likelihood scans each parameter is set to the values recommended by the priors: the same parameter set which produces the nominal model in \autoref{sec:nom_model}. The parameters are then varied one at a time from -2$\sigma$ to +2$\sigma$, where $\sigma = \sqrt{\mathbf{V}_{i,i}}$, where $\mathbf{V}_{i,i}$ is the $i^{\text{th}}$ diagonal entry of each group of parameters' covariance matrix\footnote{So the bounds are unaffected by the correlation when getting the lower and upper bounds of the scan}. The minimum of the test-statistic occurs when the parameter is equal to the prior, as this sets prior constraint term to zero and produces a MC sample which perfectly matches the Asimov set, giving a zero contribution from the sample too. When a parameter has been scanned it is reset to the prior value. It is expected that most parameters produces a Gaussian response since the prior probability density function is set to such, and few parameters produce asymmetric responses in the event rate of a given bin\footnote{Although there are some, mentioned later.}.

The scan splits the likelihood into each of the individual contributions presented in \autoref{eq:test_stat} and shows the total likelihood. For any given likelihood scan, there should only be contributions from the likelihood terms that are being varied: the sample likelihood (since the MC event $\lambda=\lambda(\vec{b},\vec{x},\vec{d})$) and the group of systematics which the parameter belongs to (e.g. the flux likelihood should vary when any of the $\vec{b}$ parameters are varied).

In practice, the estimated constraints from the one dimensional log-likelihood are not accurate due to the large correlations in the beam and ND280 parameters in the prior covariance matrices; hence $\chi^2\sim1$ does not indicate the typical $1\sigma$ sensitivity. For sensitivity estimates, it is more appropriate to vary the correlated parameters simultaneously, as is done in the Asimov fit presented later in \autoref{sec:asimov_fit}. The likelihood scans are considered more of a closure test to inspect implementation.

\begin{figure}[!h]
	\centering
\begin{subfigure}[t]{0.32\textwidth}
	\includegraphics[width=\textwidth, trim={0mm 0mm 0mm 11mm}, clip,page=5]{figures/mach3/Asimov/Full_LLHscan_18July_BeRPA_U_ND280logL_scan}
	\caption{ND280 FHC \numu 0.6-0.7 GeV}
\end{subfigure}
\begin{subfigure}[t]{0.32\textwidth}
\includegraphics[width=\textwidth, trim={0mm 0mm 0mm 11mm}, clip,page=13]{figures/mach3/Asimov/Full_LLHscan_18July_BeRPA_U_ND280logL_scan}
	\caption{ND280 FHC \numubar 0.7-1.0 GeV}
\end{subfigure}
\begin{subfigure}[t]{0.32\textwidth}
	\includegraphics[width=\textwidth, trim={0mm 0mm 0mm 11mm}, clip,page=30]{figures/mach3/Asimov/Full_LLHscan_18July_BeRPA_U_ND280logL_scan}
	\caption{ND280 RHC \numubar 0.5-0.6 GeV}
\end{subfigure}

\begin{subfigure}[t]{0.32\textwidth}
	\includegraphics[width=\textwidth, trim={0mm 0mm 0mm 11mm}, clip,page=18]{figures/mach3/Asimov/Full_LLHscan_18July_BeRPA_U_ND280logL_scan}
	\caption{ND280 FHC \nue 0.5-0.7 GeV}
\end{subfigure}
\begin{subfigure}[t]{0.32\textwidth}
	\includegraphics[width=\textwidth, trim={0mm 0mm 0mm 11mm}, clip,page=45]{figures/mach3/Asimov/Full_LLHscan_18July_BeRPA_U_ND280logL_scan}
	\caption{ND280 RHC \nuebar 0.5-0.7 GeV}
\end{subfigure}
\begin{subfigure}[t]{0.32\textwidth}
	\includegraphics[width=\textwidth, trim={0mm 0mm 0mm 11mm}, clip,page=55]{figures/mach3/Asimov/Full_LLHscan_18July_BeRPA_U_ND280logL_scan}
	\caption{SK FHC \numu 0.6-0.7 GeV}
\end{subfigure}
\caption{Asimov likelihood scans for selected beam parameters in the fit}
\label{fig:beam_asimov}
\end{figure}

\autoref{fig:beam_asimov} shows a selected number of beam parameters. Using the Asimov data set, it's clear that the prior term is dominant, even for the high-statistics ND280 FHC \numu $E_\nu = 0.6-0.7\text{ GeV}$ parameter. Many parameters see barely any constraint from the samples---e.g. \numubar in FHC running (due to very low wrong-sign events in \numu running) and \nue in FHC (due to no dedicated \nue selection being included). As expected, the SK flux parameters are only being constrained by the prior and see no direction contribution from any ND280 samples.
\begin{figure}[!h]
	\centering
	\begin{subfigure}[t]{0.32\textwidth}
		\includegraphics[width=\textwidth, trim={0mm 0mm 0mm 11mm}, clip,page=107]{figures/mach3/Asimov/Full_LLHscan_18July_BeRPA_U_ND280logL_scan}
		\caption{$M_A^{QE}$}
	\end{subfigure}
	\begin{subfigure}[t]{0.32\textwidth}
		\includegraphics[width=\textwidth, trim={0mm 0mm 0mm 11mm}, clip,page=110]{figures/mach3/Asimov/Full_LLHscan_18July_BeRPA_U_ND280logL_scan}
		\caption{2p2h norm $\nu$}
	\end{subfigure}
	\begin{subfigure}[t]{0.32\textwidth}
		\includegraphics[width=\textwidth, trim={0mm 0mm 0mm 11mm}, clip,page=113]{figures/mach3/Asimov/Full_LLHscan_18July_BeRPA_U_ND280logL_scan}
		\caption{2p2h shape C}
	\end{subfigure}
	
	\begin{subfigure}[t]{0.32\textwidth}
		\includegraphics[width=\textwidth, trim={0mm 0mm 0mm 11mm}, clip,page=118]{figures/mach3/Asimov/Full_LLHscan_18July_BeRPA_U_ND280logL_scan}
		\caption{BeRPA E}
	\end{subfigure}
	\begin{subfigure}[t]{0.32\textwidth}
		\includegraphics[width=\textwidth, trim={0mm 0mm 0mm 11mm}, clip,page=122]{figures/mach3/Asimov/Full_LLHscan_18July_BeRPA_U_ND280logL_scan}
		\caption{$I_{1/2}^\text{bkg} 	$}
	\end{subfigure}
	\begin{subfigure}[t]{0.32\textwidth}
		\includegraphics[width=\textwidth, trim={0mm 0mm 0mm 11mm}, clip,page=105]{figures/mach3/Asimov/Full_LLHscan_18July_BeRPA_U_ND280logL_scan}
		\caption{FSI CEX LO}
	\end{subfigure}
	\caption{Asimov likelihood scans for selected cross-section parameters in the fit}
	\label{fig:xsec_asimov}
\end{figure}

\autoref{fig:xsec_asimov} shows the likelihood scans for a selected few cross-section parameters. $M_A^{QE}$ and 2p2h norm $\nu$ are both fit without a prior, so only sees constraint from the sample likelihood, and 2p2h shape C has an almost flat prior likelihood, largely constrained by the samples. Some cross-section parameters, like BeRPA E, have a weaker constraint from the MC samples than from the prior, largely due to its effect being limited to high $Q^2$, of which ND280 have few. We also observe some non-Gaussian responses, such as the pion final-state-interaction charge exchange at low pion momentum (FSI CEX LO) and the single pion production non-resonant background parameter, $I_{1/2}^\text{bkg}$.
\begin{figure}[!h]
	\centering
	\begin{subfigure}[t]{0.32\textwidth}
		\includegraphics[width=\textwidth, trim={0mm 0mm 0mm 11mm}, clip,page=132]{figures/mach3/Asimov/Full_LLHscan_18July_BeRPA_U_ND280logL_scan}
		\caption{FGD1 CC0$\pi$ 	\{0-1 GeV, 0.6-0.7\}}
	\end{subfigure}
	\begin{subfigure}[t]{0.32\textwidth}
		\includegraphics[width=\textwidth, trim={0mm 0mm 0mm 11mm}, clip,page=239]{figures/mach3/Asimov/Full_LLHscan_18July_BeRPA_U_ND280logL_scan}
		\caption{FGD1 CCOther \{1.5-2 GeV, 0.8-0.85\}}
	\end{subfigure}
	\begin{subfigure}[t]{0.32\textwidth}
		\includegraphics[width=\textwidth, trim={0mm 0mm 0mm 11mm}, clip,page=264]{figures/mach3/Asimov/Full_LLHscan_18July_BeRPA_U_ND280logL_scan}
		\caption{FGD1 \numubar CC1Trk \{0.4-0.9 GeV, 0.6-0.7\}}
	\end{subfigure}

\begin{subfigure}[t]{0.32\textwidth}
	\includegraphics[width=\textwidth, trim={0mm 0mm 0mm 11mm}, clip,page=410]{figures/mach3/Asimov/Full_LLHscan_18July_BeRPA_U_ND280logL_scan}
	\caption{FGD2 CC0$\pi$ 	\{0-1 GeV, 0.6-0.7\}}
\end{subfigure}
\begin{subfigure}[t]{0.32\textwidth}
	\includegraphics[width=\textwidth, trim={0mm 0mm 0mm 11mm}, clip,page=517]{figures/mach3/Asimov/Full_LLHscan_18July_BeRPA_U_ND280logL_scan}
	\caption{FGD2 CCOther \{1.5-2 GeV, 0.8-0.85\}}
\end{subfigure}
\begin{subfigure}[t]{0.32\textwidth}
	\includegraphics[width=\textwidth, trim={0mm 0mm 0mm 11mm}, clip,page=542]{figures/mach3/Asimov/Full_LLHscan_18July_BeRPA_U_ND280logL_scan}
	\caption{FGD2 \numubar CC1Trk \{0.4-0.9 GeV, 0.6-0.7\}}
\end{subfigure}
\caption{Asimov likelihood scans for selected ND280 parameters in the fit}
\label{fig:nd280_asimov}
\end{figure}

\autoref{fig:nd280_asimov} shows a selected number of the ND280 parameters for FGD1 and FGD2. Generally, the ND280 parameters are more balanced between prior and sample likelihoods. The effect is by design, since the underlying MC events that are being varied when making the detector covariance are the same as those being varied in the fit: the only difference is the fit binning and the binning used to make the ND280 covariance matrix, covered in \autoref{subsec:syst_nd280}.

As expected, the ND280 parameters covering high statistics samples and regions of phase space---such as CC0$\pi$, $0<p_\mu<1.0\text{ GeV}$, $0.6 < \cos\theta_\mu < 0.7$---have higher constraints than low ones---such as CCOther $1.5 < p_\mu < 2.0\text{ GeV}$, $0.8 < \cos\theta_\mu < 0.85$.

Comparing top and bottom panels, the responses for equivalent FGD1 and FGD2 parameters are compatible and have similar strength.

\subsection{Parameter variations}
\label{sec:sigmavar}
To finally inspect the effects of the parameterisation we vary the parameters one at a time over one unit of $\sigma$, where $\sigma = \sqrt{\mathbf{V}_{i,i}}$ as before, now looking at the impact on the event distributions for each selection.

The largest effects of the variations on the event rates in each sample is shown in \autoref{tab:onesigma}. For the CC0$\pi$ or 1 track selections the 2p2h normalisation parameters has the largest effect. This is expected because of the uncertainty applied is conservative at $\pm100\%$, so the lower event rate (15139.1 for FGD1 0$\pi$) is the event rate without any 2p2h $\nu$ events. For CC1$\pi$ selections, the largest effect is the $C_5^A$ parameter, which controls the single pion production model. For the CCOther selections the CC DIS parameter has the largest effect. For the RHC NTrack selections the $M_A^{RES}$ parameter is instead dominant, controlling the single pion production model of which the NTrack selections is dominated by (e.g. FGD1 NTrack 28.4\% CC1$\pi^{\pm,0}$ vs 26.9\% CCDIS in \autoref{tab:nominal_mode_afterscale}). The only selection which is does not have an interaction parameter as its largest uncertainity on event rate is FGD1 NTrk \numu, where the 29th beam parameter (RHC \numubar, $E_\nu = 0.7-1.0 \text{ GeV}$)--- the second largest effect is from $M_A^{RES}$, which predicts event rates from +1$\sigma$:1291.84 to -1$\sigma$: 1432.51. 

\begin{table}[h]
	\centering
	\begin{tabular}{l | c | c c c }
		\hline
		\hline
		Sample & Parameter & +1$\sigma$ & Nominal & -1$\sigma$ \\
		\hline
		FGD1 0$\pi$ & 2p2h norm $\nu$ & 15139.1 & 16723.8 & 18308.6 \\
		FGD2 0$\pi$ & 2p2h norm $\nu$ & 15420.4 & 16959.3 & 18498.2 \\
		FGD1 1$\pi$ & $C^A_5$ & 4056.67 & 4381.47 & 4746.83 \\
		FGD2 1$\pi$ & $C^A_5$ & 3307.92 & 3564.23 & 3852.44 \\
		FGD1 Other & CC DIS & 3691.18 & 3943.95 & 4196.72 \\
		FGD2 Other & CC DIS & 3343.41 & 3570.94 & 3798.47 \\
		\hline
		FGD1 1Trk & 2p2h norm $\bar{\nu}$ & 3245.72 & 3587.77 & 3929.83 \\ 
		FGD2 1Trk & 2p2h norm $\bar{\nu}$ & 3272.86 & 3618.29 & 3963.73 \\
		FGD1 NTrk & $M_A^{RES}$ & 1019.96 & 1066.91 & 1126.26 \\
		FGD2 NTrk & $M_A^{RES}$ & 1028.7 & 1077.24 & 1138.62 \\
		
		\hline
		FGD1 1Trk \numu & 2p2h norm $\nu$ & 1178.79 & 1272.17 & 1365.55 \\
		FGD2 1Trk \numu & 2p2h norm $\nu$ & 1170.3 & 1262.63 & 1354.97 \\
		FGD1 NTrk \numu & b29 & 1282.3 & 1357.45 & 1432.61 \\
		FGD2 NTrk \numu & $M_A^{RES}$ & 1184.42 & 1246.71 & 1317.25 \\
		\hline
		\hline
	\end{tabular}
	\caption{The largest effect of the 1-$\sigma$ variations on each sample on the event selections}
	\label{tab:onesigma}
\end{table}

The results are expected and entirely compatible with the likelihood scans in \autoref{sec:llh_scan}, where we saw very strong constraints on many parameters (e.g. 2p2h norm $\nu$), which primarily came from sensitivity in the sample distributions. It is therefore expected that those parameters have a large impact on the event rates from the 1$\sigma$ variations.

The impact on the \pmu \cosmu distributions for each sample and their respective ``highest impact parameters'' presented in \autoref{tab:onesigma} are shown in \autoref{fig:onesigma_fhc} and \autoref{fig:onesigma_rhc}. We note similar responses in both FGDs and across samples.

\begin{figure}[!h]
	\begin{subfigure}[t]{0.24\textwidth}
	\includegraphics[width=\textwidth,page=1]{figures/mach3/sigmavar/Full_1sigmaVar_18July_BeRPA_U_ND280_sigmavar_highest_all}
	\end{subfigure}
	\begin{subfigure}[t]{0.24\textwidth}
	\includegraphics[width=\textwidth,page=2]{figures/mach3/sigmavar/Full_1sigmaVar_18July_BeRPA_U_ND280_sigmavar_highest_all}
	\end{subfigure}
	\begin{subfigure}[t]{0.24\textwidth}
	\includegraphics[width=\textwidth,page=3]{figures/mach3/sigmavar/Full_1sigmaVar_18July_BeRPA_U_ND280_sigmavar_highest_all}
	\end{subfigure}
	\begin{subfigure}[t]{0.24\textwidth}
	\includegraphics[width=\textwidth,page=4]{figures/mach3/sigmavar/Full_1sigmaVar_18July_BeRPA_U_ND280_sigmavar_highest_all}
	\end{subfigure}

	\begin{subfigure}[t]{0.24\textwidth}
	\includegraphics[width=\textwidth,page=7]{figures/mach3/sigmavar/Full_1sigmaVar_18July_BeRPA_U_ND280_sigmavar_highest_all}
	\end{subfigure}
	\begin{subfigure}[t]{0.24\textwidth}
	\includegraphics[width=\textwidth,page=8]{figures/mach3/sigmavar/Full_1sigmaVar_18July_BeRPA_U_ND280_sigmavar_highest_all}
	\end{subfigure}
	\begin{subfigure}[t]{0.24\textwidth}
		\includegraphics[width=\textwidth,page=9]{figures/mach3/sigmavar/Full_1sigmaVar_18July_BeRPA_U_ND280_sigmavar_highest_all}
	\end{subfigure}
	\begin{subfigure}[t]{0.24\textwidth}
		\includegraphics[width=\textwidth,page=10]{figures/mach3/sigmavar/Full_1sigmaVar_18July_BeRPA_U_ND280_sigmavar_highest_all}
	\end{subfigure}

\begin{subfigure}[t]{0.24\textwidth}
	\includegraphics[width=\textwidth,page=5]{figures/mach3/sigmavar/Full_1sigmaVar_18July_BeRPA_U_ND280_sigmavar_highest_all}
\end{subfigure}
\begin{subfigure}[t]{0.24\textwidth}
	\includegraphics[width=\textwidth,page=6]{figures/mach3/sigmavar/Full_1sigmaVar_18July_BeRPA_U_ND280_sigmavar_highest_all}
\end{subfigure}
\begin{subfigure}[t]{0.24\textwidth}
\includegraphics[width=\textwidth,page=11]{figures/mach3/sigmavar/Full_1sigmaVar_18July_BeRPA_U_ND280_sigmavar_highest_all}
\end{subfigure}
\begin{subfigure}[t]{0.24\textwidth}
\includegraphics[width=\textwidth,page=12]{figures/mach3/sigmavar/Full_1sigmaVar_18July_BeRPA_U_ND280_sigmavar_highest_all}
\end{subfigure}
\caption{The largest effect of the 1-$\sigma$ variations on FHC selections' \pmu \cosmu}
\label{fig:onesigma_fhc}
\end{figure}

\begin{figure}[h]
\begin{subfigure}[t]{0.24\textwidth}
	\includegraphics[width=\textwidth,page=13]{figures/mach3/sigmavar/Full_1sigmaVar_18July_BeRPA_U_ND280_sigmavar_highest_all}
\end{subfigure}
\begin{subfigure}[t]{0.24\textwidth}
	\includegraphics[width=\textwidth,page=14]{figures/mach3/sigmavar/Full_1sigmaVar_18July_BeRPA_U_ND280_sigmavar_highest_all}
	\end{subfigure}
\begin{subfigure}[t]{0.24\textwidth}
\includegraphics[width=\textwidth,page=15]{figures/mach3/sigmavar/Full_1sigmaVar_18July_BeRPA_U_ND280_sigmavar_highest_all}
\end{subfigure}
\begin{subfigure}[t]{0.24\textwidth}
\includegraphics[width=\textwidth,page=16]{figures/mach3/sigmavar/Full_1sigmaVar_18July_BeRPA_U_ND280_sigmavar_highest_all}
\end{subfigure}

\begin{subfigure}[t]{0.24\textwidth}
	\includegraphics[width=\textwidth,page=17]{figures/mach3/sigmavar/Full_1sigmaVar_18July_BeRPA_U_ND280_sigmavar_highest_all}
\end{subfigure}
\begin{subfigure}[t]{0.24\textwidth}
	\includegraphics[width=\textwidth,page=18]{figures/mach3/sigmavar/Full_1sigmaVar_18July_BeRPA_U_ND280_sigmavar_highest_all}
	\end{subfigure}
\begin{subfigure}[t]{0.24\textwidth}
\includegraphics[width=\textwidth,page=19]{figures/mach3/sigmavar/Full_1sigmaVar_18July_BeRPA_U_ND280_sigmavar_highest_all}
\end{subfigure}
\begin{subfigure}[t]{0.24\textwidth}
\includegraphics[width=\textwidth,page=20]{figures/mach3/sigmavar/Full_1sigmaVar_18July_BeRPA_U_ND280_sigmavar_highest_all}
\end{subfigure}

\begin{subfigure}[t]{0.24\textwidth}
	\includegraphics[width=\textwidth,page=21]{figures/mach3/sigmavar/Full_1sigmaVar_18July_BeRPA_U_ND280_sigmavar_highest_all}
\end{subfigure}
\begin{subfigure}[t]{0.24\textwidth}
	\includegraphics[width=\textwidth,page=22]{figures/mach3/sigmavar/Full_1sigmaVar_18July_BeRPA_U_ND280_sigmavar_highest_all}
\end{subfigure}
\begin{subfigure}[t]{0.24\textwidth}
\includegraphics[width=\textwidth,page=23]{figures/mach3/sigmavar/Full_1sigmaVar_18July_BeRPA_U_ND280_sigmavar_highest_all}
\end{subfigure}
\begin{subfigure}[t]{0.24\textwidth}
\includegraphics[width=\textwidth,page=24]{figures/mach3/sigmavar/Full_1sigmaVar_18July_BeRPA_U_ND280_sigmavar_highest_all}
\end{subfigure}

\begin{subfigure}[t]{0.24\textwidth}
	\includegraphics[width=\textwidth,page=25]{figures/mach3/sigmavar/Full_1sigmaVar_18July_BeRPA_U_ND280_sigmavar_highest_all}
\end{subfigure}
\begin{subfigure}[t]{0.24\textwidth}
	\includegraphics[width=\textwidth,page=26]{figures/mach3/sigmavar/Full_1sigmaVar_18July_BeRPA_U_ND280_sigmavar_highest_all}
\end{subfigure}
\begin{subfigure}[t]{0.24\textwidth}
	\includegraphics[width=\textwidth,page=27]{figures/mach3/sigmavar/Full_1sigmaVar_18July_BeRPA_U_ND280_sigmavar_highest_all}
\end{subfigure}
\begin{subfigure}[t]{0.24\textwidth}
	\includegraphics[width=\textwidth,page=28]{figures/mach3/sigmavar/Full_1sigmaVar_18July_BeRPA_U_ND280_sigmavar_highest_all}
\end{subfigure}

\caption{The largest effect of the 1-$\sigma$ variations on RHC selections' \pmu \cosmu}
\label{fig:onesigma_rhc}
\end{figure}

%beam+ND280+xsec
%Largest difference for FGD1_numuCC_0pi: 1584.73 (FGD1_numuCC_0pi_2p2h_norm_nu)
%Largest difference for FGD1_numuCC_1pi: 365.358 (FGD1_numuCC_1pi_CA5)
%Largest difference for FGD1_numuCC_other: 252.769 (FGD1_numuCC_other_CC_DIS)
%Largest difference for FGD2_numuCC_0pi: 1538.91 (FGD2_numuCC_0pi_2p2h_norm_nu)
%Largest difference for FGD2_numuCC_1pi: 288.216 (FGD2_numuCC_1pi_CA5)
%Largest difference for FGD2_numuCC_other: 227.53 (FGD2_numuCC_other_CC_DIS)
%Largest difference for FGD1_anti-numuCC_QE: 342.054 (FGD1_anti-numuCC_QE_2p2h_norm_nubar)
%Largest difference for FGD1_anti-numuCC_nQE: 59.3502 (FGD1_anti-numuCC_nQE_MARES)
%Largest difference for FGD2_anti-numuCC_1_track: 345.436 (FGD2_anti-numuCC_1_track_2p2h_norm_nubar)
%Largest difference for FGD2_anti-numuCC_N_tracks: 61.3828 (FGD2_anti-numuCC_N_tracks_MARES)
%Largest difference for FGD1_NuMuBkg_CCQE_in_AntiNu_Mode_: 93.3766 (FGD1_NuMuBkg_CCQE_in_AntiNu_Mode__2p2h_norm_nu)
%Largest difference for FGD1_NuMuBkg_CCnQE_in_AntiNu_Mode: 75.1545 (FGD1_NuMuBkg_CCnQE_in_AntiNu_Mode_b_29)
%Largest difference for FGD2_NuMuBkg_CCQE_in_AntiNu_Mode_: 92.3373 (FGD2_NuMuBkg_CCQE_in_AntiNu_Mode__2p2h_norm_nu)
%Largest difference for FGD2_NuMuBkg_CCnQE_in_AntiNu_Mode: 70.5455 (FGD2_NuMuBkg_CCnQE_in_AntiNu_Mode_MARES)

%beam+ND280
%Largest difference for FGD1_numuCC_0pi: 419.944 (FGD1_numuCC_0pi_b_4)
%Largest difference for FGD1_numuCC_1pi: 64.8746 (FGD1_numuCC_1pi_b_8)
%Largest difference for FGD1_numuCC_other: 104.927 (FGD1_numuCC_other_b_10)
%Largest difference for FGD2_numuCC_0pi: 428.013 (FGD2_numuCC_0pi_b_4)
%Largest difference for FGD2_numuCC_1pi: 54.6989 (FGD2_numuCC_1pi_b_8)
%Largest difference for FGD2_numuCC_other: 91.9717 (FGD2_numuCC_other_b_10)
%Largest difference for FGD1_anti-numuCC_QE: 95.0237 (FGD1_anti-numuCC_QE_b_34)
%Largest difference for FGD1_anti-numuCC_nQE: 29.915 (FGD1_anti-numuCC_nQE_b_29)
%Largest difference for FGD2_anti-numuCC_1_track: 95.1625 (FGD2_anti-numuCC_1_track_b_34)
%Largest difference for FGD2_anti-numuCC_N_tracks: 29.4032 (FGD2_anti-numuCC_N_tracks_b_29)
%Largest difference for FGD1_NuMuBkg_CCQE_in_AntiNu_Mode_: 37.1494 (FGD1_NuMuBkg_CCQE_in_AntiNu_Mode__b_29)
%Largest difference for FGD1_NuMuBkg_CCnQE_in_AntiNu_Mode: 75.1545 (FGD1_NuMuBkg_CCnQE_in_AntiNu_Mode_b_29)
%Largest difference for FGD2_NuMuBkg_CCQE_in_AntiNu_Mode_: 35.9022 (FGD2_NuMuBkg_CCQE_in_AntiNu_Mode__b_29)
%Largest difference for FGD2_NuMuBkg_CCnQE_in_AntiNu_Mode: 68.9212 (FGD2_NuMuBkg_CCnQE_in_AntiNu_Mode_b_29)

%ND280
%Largest difference for FGD1_numuCC_0pi: 306.549 (FGD1_numuCC_0pi_ndd_0)
%Largest difference for FGD1_numuCC_1pi: 31.2325 (FGD1_numuCC_1pi_ndd_50)
%Largest difference for FGD1_numuCC_other: 88.7887 (FGD1_numuCC_other_ndd_82)
%Largest difference for FGD2_numuCC_0pi: 354.861 (FGD2_numuCC_0pi_ndd_278)
%Largest difference for FGD2_numuCC_1pi: 31.2562 (FGD2_numuCC_1pi_ndd_328)
%Largest difference for FGD2_numuCC_other: 89.0011 (FGD2_numuCC_other_ndd_360)
%Largest difference for FGD1_anti-numuCC_QE: 43.0114 (FGD1_anti-numuCC_QE_ndd_133)
%Largest difference for FGD1_anti-numuCC_nQE: 12.409 (FGD1_anti-numuCC_nQE_ndd_183)
%Largest difference for FGD2_anti-numuCC_1_track: 46.6356 (FGD2_anti-numuCC_1_track_ndd_411)
%Largest difference for FGD2_anti-numuCC_N_tracks: 11.855 (FGD2_anti-numuCC_N_tracks_ndd_467)
%Largest difference for FGD1_NuMuBkg_CCQE_in_AntiNu_Mode_: 11.0608 (FGD1_NuMuBkg_CCQE_in_AntiNu_Mode__ndd_198)
%Largest difference for FGD1_NuMuBkg_CCnQE_in_AntiNu_Mode: 14.1052 (FGD1_NuMuBkg_CCnQE_in_AntiNu_Mode_ndd_239)
%Largest difference for FGD2_NuMuBkg_CCQE_in_AntiNu_Mode_: 10.5504 (FGD2_NuMuBkg_CCQE_in_AntiNu_Mode__ndd_476)
%Largest difference for FGD2_NuMuBkg_CCnQE_in_AntiNu_Mode: 15.3709 (FGD2_NuMuBkg_CCnQE_in_AntiNu_Mode_ndd_554)

\subsection{Prior predictive spectrum}
\label{sec:Asimov_prior}
\red{insert citation to Box, J.R. Statist. Soc. A 1980, 143, Part 4, pp. 383-430}
The first statistical test of the Asimov is to investigate the prior predictive spectrum and resulting p-values. This is meant to reflect the compatibility of the prior with the Asimov data. The p-value is expected to be about 50\% since we're throwing the parameters either side of the central value which created the Asimov data set. However, we are performing correlated throws, so offsets from 50\% is to be expected.

Here 20000 correlated throws of the parameters are performed, using the supplied covariance matrices. For each throw we
\begin{itemize}
	\item Reweight the Monte-Carlo using the new parameter set, applying an updated weight for each event
	\item Count the number of events in each \pmu \cosmu bin for each selection and save it
	\item Statistically fluctuate each bin according to its bin contents\footnote{By using a Poisson probability distribution function in which $\lambda$ is the un-fluctuated number of events in the bin}
	\item Calculate the test-statistic between the fluctuated histogram from the throw and the histogram from the throw, $\chi^2_{\text{Draw, Draw Fluc}}$
	\item Calculate the test-statistic between the data and the histogram from the throw, $\chi^2_{\text{Data, Draw}}$.
	\item Fill a two dimensional histogram of the two test-statistics
\end{itemize}
Over the 20000 throws builds a predictive distribution in each \pmu \cosmu bin, in which the z-axis is the number of calculated events in each bin. We then calculate the mean and error for each bin using the arithmetic method and by fitting a Gaussian and extracting the parameters from there. The predictive is formed by taking the mean of each bin as the number of events in the bin, and the error in the bin is calculated in the same way.

We then perform another closure test using the predictive spectrum and fluctuations of the predictive spectrum in which we
\begin{itemize}
	\item Take the predictive spectrum calculated above
	\item Statistically fluctuate the content of each \pmu \cosmu bin 
	\item Calculate the test-statistic between the fluctuated predictive histogram and the predictive histogram, $\chi^2_{\text{Pred, Pred Fluc}}$
	\item Calculate the test-statistic between the fluctuated predictive histogram and the data, $\chi^2_{\text{Data, Pred Fluc}}$.
	\item Fill a two dimensional histogram of the two test-statistics
\end{itemize}

The two two-dimensional histograms with test-statistics are formed and p-values can then be calculated, which are referred to as ``prior predictive p-values''.  These p-values reflect the priors' influence and compatibility on the Asimov data set. The p-values are simply defined as\red{check these please}

\begin{equation}
p = \frac{N\left(\chi^2_{\text{Data, Draw}} > \chi^2_{\text{Draw, Draw Fluc}}\right)}{N\left(\text{Total}\right)}
\end{equation}

\begin{equation}
p = \frac{N\left(\chi^2_{\text{Data, Pred Fluc}} > \chi^2_{\text{Pred, Pred Fluc}}\right)}{N\left(\text{Total}\right)}
\end{equation}

which reflects how compatible the prior model is with the Asimov data set. In the case of the Asimov data set it is expected that the two p-values are virtually identical, since we expect the fluctuations of the predictive spectrum to be very similar to that of a throw. \red{explain why?} 

The two p-values are shown in \autoref{fig:prior_predictive_asimov} and lay around the expected value of 50\% as previously asserted.

\begin{figure}[!h]
	\begin{subfigure}[t]{0.49\textwidth}
		\includegraphics[width=\textwidth, trim={0mm 0mm 0mm 11mm}, clip,page=1]{figures/mach3/Asimov/2017b_NewDet_3Xsec_4Det_5Flux_NewXSecTune_Asimov_merge_PriorPred_procs}
	\end{subfigure}
	\begin{subfigure}[t]{0.49\textwidth}
		\includegraphics[width=\textwidth, trim={0mm 0mm 0mm 11mm}, clip,page=2]{figures/mach3/Asimov/2017b_NewDet_3Xsec_4Det_5Flux_NewXSecTune_Asimov_merge_PriorPred_procs}
	\end{subfigure}
\caption{Prior predictive p-values for the Asimov data}
\label{fig:prior_predictive_asimov}
\end{figure}

\subsection{Asimov fit}
\label{sec:asimov_fit}
After the closure tests are complete we fit the model to the Asimov data set. Three different MCMCs were run, each one being a tuned version of the previous, in regards to the autocorrelations present in the parameters throughout the chain.\red{show autocorrelations?} \autoref{tab:asimov_mcmc_versions} shows the different chains completed for the Asimov study. In each case the burn-in is 1/4 of the total chain length.
\begin{table}[h]
	\centering
	\begin{tabular}{l | c c c}
		\hline
		\hline
							& Chain length  & Acceptance rate & Accepted steps\\
							\hline
		Low acceptance		& 800,000		& 6.9\% 		  & 55,200 \\
		Mid acceptance		& 800,000		& 15.0\% 		  & 120,000\\
		High acceptance		& 1,400,000		& 23.0\%  		  & 322,000 \\
		\hline
		\hline
	\end{tabular}
\caption{Different MCMC run configurations for the Asimov fit}
\label{tab:asimov_mcmc_versions}
\end{table}

Since MCMC methods aren't designed to find a global minimum of the test-statistic, it is less trivial to define the ``best-fit'' parameter set than with a conventional minimiser. The most straightforward method is to select the MCMC step with the highest likelihood and draw the parameter set from that step, although there is no guarantee whatsoever that this is the global minimum. For the one-dimensional parameter plots we instead opt to marginalise over all parameters except the ``parameter of interest''. Since we're dealing with a very high-dimensional, highly-correlated fit, the outcome is likely to display some ``marginalisation effects'': the parameter value which maximises the $n+1$ dimensional marginal likelihood does not maximise the $n$ dimensional marginal likelihood.

\autoref{fig:mcmc_asimov_loglsteps} demonstrates that the lowest test-statistic is indeed obtained at the very beginning of the chain, and the chain progresses to move around the minimum at $\chi^2 \sim 300-400$. The lowest test-statistic after the first 1/4 of the chain has been discarded is found at step 423471/800000, which has $\chi^2 = 284.49$.

\begin{figure}[h]
	\includegraphics[width=0.45\textwidth]{figures/mach3/mcmc/2017b_NewDet_3Xsec_4Det_5Flux_NewXSecTune_Asimov_merge}
	\caption{Markov Chain behaviour for the ``mid acceptance'' MCMC, showing intended behaviour of moving around minimum}
	\label{fig:mcmc_asimov_loglsteps}
\end{figure}

\autoref{fig:flux_asimov_fhc} (FHC) and \autoref{fig:flux_asimov_rhc} (RHC) shows the result of the Asimov fit for the ND280 flux parameters. There is a consistent ``bias'' in estimating the parameter set which generated the Asimov, whose residuals (bottom panel) appear to have a shape, going from over-estimated to under-estimated in parameter value as $E_\nu$ increases. The highest acceptance chain has less ``bias'' than the shorter chains, and all the chains are consistent with each other. Many of the parameters with large biases are barely constrained by the ND280 data (e.g. ND280 FHC $\nu_\mu$ at $E_\nu > 4\text{ GeV}$ and ND280 RHC $\bar{\nu}_e$) and only get their constraints from the prior covariance matrix. The ND280 to SK correlation is working as expected and the SK flux parameters show almost identical constraint to their ND280 equivalents.

We also note improved constraints on all flux parameters from the prior uncertainty. This is emphasises around the flux peak at $E_\nu \sim 0.6\text{ GeV}$, where the flux parameter uncertainty reduces by 50\%.
\begin{figure}[h]
	\begin{subfigure}[t]{0.10\textwidth}
		\includegraphics[width=\textwidth, trim={0mm 0mm 0mm 0mm}, clip,page=1]{figures/mach3/Asimov/2017_NewDet_Asimov_actually_0_2017b_NewDet_3Xsec_4Det_5Flux_NewXSecTune_Asimov_0_2017b_NewDet_NewData_Asimov_Long_0}
	\end{subfigure}

\begin{subfigure}[t]{0.24\textwidth}
	\includegraphics[width=\textwidth, trim={0mm 0mm 0mm 0mm}, clip,page=2]{figures/mach3/Asimov/2017_NewDet_Asimov_actually_0_2017b_NewDet_3Xsec_4Det_5Flux_NewXSecTune_Asimov_0_2017b_NewDet_NewData_Asimov_Long_0}
\end{subfigure}
\begin{subfigure}[t]{0.24\textwidth}
	\includegraphics[width=\textwidth, trim={0mm 0mm 0mm 0mm}, clip,page=3]{figures/mach3/Asimov/2017_NewDet_Asimov_actually_0_2017b_NewDet_3Xsec_4Det_5Flux_NewXSecTune_Asimov_0_2017b_NewDet_NewData_Asimov_Long_0}
\end{subfigure}
\begin{subfigure}[t]{0.24\textwidth}
	\includegraphics[width=\textwidth, trim={0mm 0mm 0mm 0mm}, clip,page=4]{figures/mach3/Asimov/2017_NewDet_Asimov_actually_0_2017b_NewDet_3Xsec_4Det_5Flux_NewXSecTune_Asimov_0_2017b_NewDet_NewData_Asimov_Long_0}
\end{subfigure}
\begin{subfigure}[t]{0.24\textwidth}
	\includegraphics[width=\textwidth, trim={0mm 0mm 0mm 0mm}, clip,page=5]{figures/mach3/Asimov/2017_NewDet_Asimov_actually_0_2017b_NewDet_3Xsec_4Det_5Flux_NewXSecTune_Asimov_0_2017b_NewDet_NewData_Asimov_Long_0}
\end{subfigure}

\begin{subfigure}[t]{0.24\textwidth}
	\includegraphics[width=\textwidth, trim={0mm 0mm 0mm 0mm}, clip,page=10]{figures/mach3/Asimov/2017_NewDet_Asimov_actually_0_2017b_NewDet_3Xsec_4Det_5Flux_NewXSecTune_Asimov_0_2017b_NewDet_NewData_Asimov_Long_0}
\end{subfigure}
\begin{subfigure}[t]{0.24\textwidth}
	\includegraphics[width=\textwidth, trim={0mm 0mm 0mm 0mm}, clip,page=11]{figures/mach3/Asimov/2017_NewDet_Asimov_actually_0_2017b_NewDet_3Xsec_4Det_5Flux_NewXSecTune_Asimov_0_2017b_NewDet_NewData_Asimov_Long_0}
\end{subfigure}
\begin{subfigure}[t]{0.24\textwidth}
	\includegraphics[width=\textwidth, trim={0mm 0mm 0mm 0mm}, clip,page=12]{figures/mach3/Asimov/2017_NewDet_Asimov_actually_0_2017b_NewDet_3Xsec_4Det_5Flux_NewXSecTune_Asimov_0_2017b_NewDet_NewData_Asimov_Long_0}
\end{subfigure}
\begin{subfigure}[t]{0.24\textwidth}
	\includegraphics[width=\textwidth, trim={0mm 0mm 0mm 0mm}, clip,page=13]{figures/mach3/Asimov/2017_NewDet_Asimov_actually_0_2017b_NewDet_3Xsec_4Det_5Flux_NewXSecTune_Asimov_0_2017b_NewDet_NewData_Asimov_Long_0}
\end{subfigure}
\caption{ND280 and SK FHC flux parameters after the Asimov fit for different MCMC chains}
\label{fig:flux_asimov_fhc}
\end{figure}

\begin{figure}[h]
	\begin{subfigure}[t]{0.24\textwidth}
		\includegraphics[width=\textwidth, trim={0mm 0mm 0mm 0mm}, clip,page=6]{figures/mach3/Asimov/2017_NewDet_Asimov_actually_0_2017b_NewDet_3Xsec_4Det_5Flux_NewXSecTune_Asimov_0_2017b_NewDet_NewData_Asimov_Long_0}
	\end{subfigure}
	\begin{subfigure}[t]{0.24\textwidth}
		\includegraphics[width=\textwidth, trim={0mm 0mm 0mm 0mm}, clip,page=7]{figures/mach3/Asimov/2017_NewDet_Asimov_actually_0_2017b_NewDet_3Xsec_4Det_5Flux_NewXSecTune_Asimov_0_2017b_NewDet_NewData_Asimov_Long_0}
	\end{subfigure}
	\begin{subfigure}[t]{0.24\textwidth}
		\includegraphics[width=\textwidth, trim={0mm 0mm 0mm 0mm}, clip,page=8]{figures/mach3/Asimov/2017_NewDet_Asimov_actually_0_2017b_NewDet_3Xsec_4Det_5Flux_NewXSecTune_Asimov_0_2017b_NewDet_NewData_Asimov_Long_0}
	\end{subfigure}
	\begin{subfigure}[t]{0.24\textwidth}
		\includegraphics[width=\textwidth, trim={0mm 0mm 0mm 0mm}, clip,page=9]{figures/mach3/Asimov/2017_NewDet_Asimov_actually_0_2017b_NewDet_3Xsec_4Det_5Flux_NewXSecTune_Asimov_0_2017b_NewDet_NewData_Asimov_Long_0}
		\end{subfigure}
		
	\begin{subfigure}[t]{0.24\textwidth}
		\includegraphics[width=\textwidth, trim={0mm 0mm 0mm 0mm}, clip,page=14]{figures/mach3/Asimov/2017_NewDet_Asimov_actually_0_2017b_NewDet_3Xsec_4Det_5Flux_NewXSecTune_Asimov_0_2017b_NewDet_NewData_Asimov_Long_0}
	\end{subfigure}
	\begin{subfigure}[t]{0.24\textwidth}
		\includegraphics[width=\textwidth, trim={0mm 0mm 0mm 0mm}, clip,page=15]{figures/mach3/Asimov/2017_NewDet_Asimov_actually_0_2017b_NewDet_3Xsec_4Det_5Flux_NewXSecTune_Asimov_0_2017b_NewDet_NewData_Asimov_Long_0}
	\end{subfigure}
	\begin{subfigure}[t]{0.24\textwidth}
		\includegraphics[width=\textwidth, trim={0mm 0mm 0mm 0mm}, clip,page=16]{figures/mach3/Asimov/2017_NewDet_Asimov_actually_0_2017b_NewDet_3Xsec_4Det_5Flux_NewXSecTune_Asimov_0_2017b_NewDet_NewData_Asimov_Long_0}
	\end{subfigure}
	\begin{subfigure}[t]{0.24\textwidth}
		\includegraphics[width=\textwidth, trim={0mm 0mm 0mm 0mm}, clip,page=17]{figures/mach3/Asimov/2017_NewDet_Asimov_actually_0_2017b_NewDet_3Xsec_4Det_5Flux_NewXSecTune_Asimov_0_2017b_NewDet_NewData_Asimov_Long_0}
	\end{subfigure}
	\caption{ND280 and SK RHC flux parameters after the Asimov fit for different MCMC chains}
	\label{fig:flux_asimov_rhc}
\end{figure}
\autoref{fig:xsec_asimov} shows interaction parameters after the fit to Asimov data. As for the flux parameters, some parameters are ``biased'', notably the Fermi momentum parameters ($p_F$), 2p2h norm $\bar{\nu}$, BeRPA A, CC DIS and NC$1\gamma$. Again, the chains are very compatible in their estimates of the parameter values, and the reduction of the uncertainties are significant for many parameters. $p_F$, 2p2h norm C/O, BeRPA E, $\nu_e$/$\nu_\mu$, NC coherent, NC$1\gamma$ and NC oth. SK barely improve, due to the fit lacking events which constrain these parameters, or the parameter has very little ``strength'' to change the interaction cross-section.

\begin{figure}[h]
	\begin{subfigure}[t]{0.49\textwidth}
		\includegraphics[width=\textwidth, trim={0mm 0mm 0mm 0mm}, clip,page=18]{figures/mach3/Asimov/2017_NewDet_Asimov_actually_0_2017b_NewDet_3Xsec_4Det_5Flux_NewXSecTune_Asimov_0_2017b_NewDet_NewData_Asimov_Long_0}
	\end{subfigure}
	\begin{subfigure}[t]{0.49\textwidth}
		\includegraphics[width=\textwidth, trim={0mm 0mm 0mm 0mm}, clip,page=19]{figures/mach3/Asimov/2017_NewDet_Asimov_actually_0_2017b_NewDet_3Xsec_4Det_5Flux_NewXSecTune_Asimov_0_2017b_NewDet_NewData_Asimov_Long_0}
	\end{subfigure}

	\begin{subfigure}[t]{0.49\textwidth}
		\includegraphics[width=\textwidth, trim={0mm 0mm 0mm 0mm}, clip,page=20]{figures/mach3/Asimov/2017_NewDet_Asimov_actually_0_2017b_NewDet_3Xsec_4Det_5Flux_NewXSecTune_Asimov_0_2017b_NewDet_NewData_Asimov_Long_0}
	\end{subfigure}
	\begin{subfigure}[t]{0.49\textwidth}
		\includegraphics[width=\textwidth, trim={0mm 0mm 0mm 0mm}, clip,page=21]{figures/mach3/Asimov/2017_NewDet_Asimov_actually_0_2017b_NewDet_3Xsec_4Det_5Flux_NewXSecTune_Asimov_0_2017b_NewDet_NewData_Asimov_Long_0}
	\end{subfigure}
	\caption{Interaction parameters after the Asimov fit for different MCMC chains}
	\label{fig:xsec_asimov}
\end{figure}

\red{Show some posteriors?}
\subsubsection{Asimov fit, flux parameters only}
To investigate if correlations and marginalisation are to blame for the ``bias'' in \autoref{sec:asimov_fit}, a fit to the Asimov data is done keeping the cross-section and ND280 parameters fixed at their Asimov values, and only the flux parameters are varied. \autoref{fig:flux_asimov_nd280_fluxonly} shows the parameter values for the ND280 flux parameters using different methods of obtaining the central values. The biases have vanished for all methods, and we conclude that the Asimov parameters are being correctly found when flux parameters alone are being fit.

\begin{figure}[h]
	\begin{subfigure}[t]{0.10\textwidth}
		\includegraphics[width=\textwidth, trim={0mm 0mm 0mm 0mm}, clip,page=1]{figures/mach3/Asimov/2017b_Asimov_July2017_FixND280_FixXsec_0_drawPar}
	\end{subfigure}
	
	\begin{subfigure}[t]{0.24\textwidth}
		\includegraphics[width=\textwidth, trim={0mm 0mm 0mm 0mm}, clip,page=2]{figures/mach3/Asimov/2017b_Asimov_July2017_FixND280_FixXsec_0_drawPar}
	\end{subfigure}
	\begin{subfigure}[t]{0.24\textwidth}
		\includegraphics[width=\textwidth, trim={0mm 0mm 0mm 0mm}, clip,page=3]{figures/mach3/Asimov/2017b_Asimov_July2017_FixND280_FixXsec_0_drawPar}
	\end{subfigure}
	\begin{subfigure}[t]{0.24\textwidth}
		\includegraphics[width=\textwidth, trim={0mm 0mm 0mm 0mm}, clip,page=4]{figures/mach3/Asimov/2017b_Asimov_July2017_FixND280_FixXsec_0_drawPar}
	\end{subfigure}
	\begin{subfigure}[t]{0.24\textwidth}
		\includegraphics[width=\textwidth, trim={0mm 0mm 0mm 0mm}, clip,page=5]{figures/mach3/Asimov/2017b_Asimov_July2017_FixND280_FixXsec_0_drawPar}
	\end{subfigure}
	
	\begin{subfigure}[t]{0.24\textwidth}
		\includegraphics[width=\textwidth, trim={0mm 0mm 0mm 0mm}, clip,page=6]{figures/mach3/Asimov/2017b_Asimov_July2017_FixND280_FixXsec_0_drawPar}
	\end{subfigure}
	\begin{subfigure}[t]{0.24\textwidth}
		\includegraphics[width=\textwidth, trim={0mm 0mm 0mm 0mm}, clip,page=7]{figures/mach3/Asimov/2017b_Asimov_July2017_FixND280_FixXsec_0_drawPar}
	\end{subfigure}
	\begin{subfigure}[t]{0.24\textwidth}
		\includegraphics[width=\textwidth, trim={0mm 0mm 0mm 0mm}, clip,page=8]{figures/mach3/Asimov/2017b_Asimov_July2017_FixND280_FixXsec_0_drawPar}
	\end{subfigure}
	\begin{subfigure}[t]{0.24\textwidth}
		\includegraphics[width=\textwidth, trim={0mm 0mm 0mm 0mm}, clip,page=9]{figures/mach3/Asimov/2017b_Asimov_July2017_FixND280_FixXsec_0_drawPar}
	\end{subfigure}
	
	\caption{ND280 flux parameters after the Asimov fit, fitting flux only}
	\label{fig:flux_asimov_nd280_fluxonly}
\end{figure}

\subsubsection{Marginalisation effects}
\label{sec:marginalisation}
In \autoref{sec:asimov_fit} it was noted that many parameters appeared ``biased'' in their one-dimensional marginal posterior not agreeing with the input Asimov value. This is expected for parameters which correlate strongly with non-Gaussian parameters, of which the fit contains some. Here we investigate this further by looking at the two-dimensional marginal posterior for a few correlated parameters and comparing them to the one-dimensional marginal posterior. For the central value parameter estimates we use the highest posterior density point. We look at three selected parameters that showed bias: $p_F^{C}$, 2p2h norm $\bar{\nu}$ and the 30th beam parameter (``$b_29$'', or the ND280 RHC $\bar{\nu}_\mu$ 0.7-1.0 GeV).

\paragraph{$p_F^C$ bias}
\autoref{fig:marginalisation_pf} shows the two dimensional marginal posterior of $p_F^C$ with $p_F^O$ and BeRPA D. The $p_F$ parameters are problematic in that the parameterisation \red{show splines?} allows for non-Gaussian behaviour and additionally has hard cut-offs near the prior value--notably at the lower limit. Marginalising over such parameters causes shifts in the highest posterior densities. For the two-dimensional posterior of $p_F^C$ and $p_F^O$ the highest density is indeed very close to the prior input value of 1.0 for both parameters (within one bin-width), whereas marginalising over the other parameter causes the posterior density to shift from $p_F^{C}$ = 1.008 to 1.062. The second inset shows the opposite effect when marginalising $p_F^{C}$ over BeRPA D, in which the two-dimensional marginal posterior has its highest density at $p_F^{C} = 1.077$ which shifts to $p_F^{C} = 1.054$ after the marginalisation.

\begin{figure}[h]
	\begin{subfigure}[t]{0.49\textwidth}
		\includegraphics[width=\textwidth, trim={0mm 0mm 0mm 0mm}, clip,page=15]{figures/mach3/mcmc/2017b_NewDet_3Xsec_4Det_5Flux_NewXSecTune_Asimov_merge_marg_xsec}
	\end{subfigure}
	\begin{subfigure}[t]{0.49\textwidth}
		\includegraphics[width=\textwidth, trim={0mm 0mm 0mm 0mm}, clip,page=23]{figures/mach3/mcmc/2017b_NewDet_3Xsec_4Det_5Flux_NewXSecTune_Asimov_merge_marg_xsec}
	\end{subfigure}
\caption{Selected two-dimensional marginal posteriors for $p_F^C$ and 2p2h shape O and BeRPA B, showing the resulting one-dimensional marginal posterior}
\label{fig:marginalisation_pf}
\end{figure}

\paragraph{2p2h norm $\bar{\nu}_\mu$ bias}
\autoref{fig:marginalisation_2p2h_norm_nubar} shows the same marginalisation plots for 2p2h norm $\bar{\nu}$ with 2p2h shape C and BeRPA E. The case of the 2p2h normalisation parameters are slightly different to the $p_F$ parameters: the normalisation is well-behaved across the phase space and looks very Gaussian and does not have any hard cut-offs near the prior (cut-off is when normalisation = 0). The marginsaliation effect happens instead with parameters that 2p2h normalisations correlate heavily with which may have non-Gaussian posteriors. The first example is with 2p2h shape C, in which we notice a tail at low 2p2h shape C and low 2p2h norm $\bar{\nu}$. Marginalising over the parameter bring the posterior density from 2p2h norm $\bar{\nu}$ = 1.061 to 0.842, which is the main cause of the ``bias''. Again, other parameters have the opposite effect: marginalising over BeRPA E causes the marginal posterior to shift from 2p2h norm $\bar{\nu}$ from 0.769 to 0.842.
\begin{figure}[h]
	\begin{subfigure}[t]{0.49\textwidth}
		\includegraphics[width=\textwidth, trim={0mm 0mm 0mm 0mm}, clip,page=55]{figures/mach3/mcmc/2017b_NewDet_3Xsec_4Det_5Flux_NewXSecTune_Asimov_merge_marg_xsec}
	\end{subfigure}
	\begin{subfigure}[t]{0.49\textwidth}
		\includegraphics[width=\textwidth, trim={0mm 0mm 0mm 0mm}, clip,page=60]{figures/mach3/mcmc/2017b_NewDet_3Xsec_4Det_5Flux_NewXSecTune_Asimov_merge_marg_xsec}
	\end{subfigure}
	\caption{Selected two-dimensional marginal posteriors for 2p2h norm $\bar{\nu}$ with 2p2h shape C and BeRPA E showing the resulting one-dimensional marginal posterior}
	\label{fig:marginalisation_2p2h_norm_nubar}
\end{figure}

\paragraph{$b\_29$ bias}
The 30th beam parameter (``$b\_29$'', or the ND280 RHC $\bar{\nu}_\mu$ 0.7-1.0 GeV) sees similar effects from a number of parameters:
``$b\_24$'' (ND280 FHC $\bar{\nu}_e$ 2.5-30 GeV), ``$b\_46$'' (ND280 RHC $\nu_\mu$ 1.5-2.5 GeV), ``$b\_49$'' (ND280 RHC $\nu_e$ 2.5-30 GeV) and BeRPA D. The marginalisation plots are shown in \autoref{fig:marginalisation_b29} and all the above parameters have identical shifts: $b\_29$ moves from 0.988 to 0.974. It is noteworthy that the above parameters are all only weakly constrained by ND280 data: there is no dedicated $\nu_e$ selection (b\_46 and b\_49) and high $Q^2$ CCQE events are sparse (BeRPA D). This can cause non-Gaussianity in that the chain may ``wander'' the space and explore a largely flat likelihood in one parameter.

\begin{figure}[h]
	\begin{subfigure}[t]{0.24\textwidth}
		\includegraphics[width=\textwidth, trim={0mm 0mm 0mm 0mm}, clip,page=25]{figures/mach3/mcmc/2017b_NewDet_3Xsec_4Det_5Flux_NewXSecTune_Asimov_merge_b29}
	\end{subfigure}
	\begin{subfigure}[t]{0.24\textwidth}
		\includegraphics[width=\textwidth, trim={0mm 0mm 0mm 0mm}, clip,page=47]{figures/mach3/mcmc/2017b_NewDet_3Xsec_4Det_5Flux_NewXSecTune_Asimov_merge_b29}
	\end{subfigure}
	\begin{subfigure}[t]{0.24\textwidth}
		\includegraphics[width=\textwidth, trim={0mm 0mm 0mm 0mm}, clip,page=50]{figures/mach3/mcmc/2017b_NewDet_3Xsec_4Det_5Flux_NewXSecTune_Asimov_merge_b29}
	\end{subfigure}
	\begin{subfigure}[t]{0.24\textwidth}
		\includegraphics[width=\textwidth, trim={0mm 0mm 0mm 0mm}, clip,page=61]{figures/mach3/mcmc/2017b_NewDet_3Xsec_4Det_5Flux_NewXSecTune_Asimov_merge_b29}
	\end{subfigure}
	\caption{Selected two-dimensional marginal posteriors for ``$b\_29$'' (ND280 RHC $\bar{\nu}_\mu$ 0.7-1.0 GeV)}
	\label{fig:marginalisation_b29}
\end{figure}
In conclusion, the biases present in the Asmiov fit study seems likely to be due to marginlisation effects over parameters that are non-Gaussian, have hard cut-offs or are poorly constrained in the fit. These patterns are expected to arise again when fitting against real data.

\subsection{Posterior predictive spectrum}
\red{Put in reference to the posterior predictive calculations}
The next statistical closure test we can perform on the Asimov data fit is to calculate the posterior predictive spectra and p-values. The procedure and aim of the tests is similar to that of the prior predictive spectrum calculation presented in \autoref{sec:Asimov_prior}, but the parameter variations no longer come form the pre-fit covariance matrices. Instead the parameter sets come from the Markov Chain steps after the burn-in has been removed, so the p-value reflects the future predictability of the model in light of the fit data, were more data to be observed.\red{clarify this?} The expectation of this test is to produce a p-value very close to 1.0, since the Asimov data is an exact parameter set of the model.

\autoref{tab:asimov_posterior_pred} shows the event rates from the posterior predictive spectrum. We note a consistently low likelihood contribution from all samples ($\sim0.18-0.42$), totalling at 4.43. This is small compared to the number of bins being fit (1624), and the contributions enter primarily in low statistics areas (high \pmu, low \cosmu), where non-Gaussian behaviour is expected. The 2D \pmu \cosmu distributions and the bin-by-bin likelihood contributions for FGD1 and FGD2 0$\pi$ is shown in \autoref{fig:posterior_predictive_2d_asimov}.
\begin{table}[h]
	\centering
  \begin{tabular}{l c c c}
\hline
\hline
    Sample & Nominal & Pos. Pred & -2$\log\mathcal{L}_s$ \\ 
\hline
 FGD1 0$\pi$ & 16723.8 &  16730.9 &  0.42 \\
 FGD1 1$\pi$ & 4381.47 &  4371.67 &  0.29 \\
 FGD1 Other & 3943.95 &  3957.1  & 0.28\\
 FGD2 0$\pi$ & 16959.3 &  16952.8 &  0.31 \\
 FGD2 1$\pi$ & 3564.23 &  3565.16 &  0.36 \\
 FGD2 Other & 3570.94 &  3571.43 &  0.34 \\
    \hline
 FGD1 1Trk & 3587.77 &  3586.45 &  0.28 \\
 FGD1 NTrk & 1066.91 &  1075.13 &  0.35  \\
 FGD2 1Trk & 3618.29 &  3612.19 &  0.36 \\
 FGD2 NTrk & 1077.24 &  1084.68 &  0.20 \\
    \hline
 FGD1 \numu 1 Trk & 1272.17 &  1267.04 &  0.18 \\
 FGD1 \numu NTrk & 1357.45 &  1357.19 &  0.38 \\
 FGD2 \numu 1Trk & 1262.63 &  1259.48 &  0.35 \\
 FGD2 \numu NTrk & 1246.71 &  1246.61 &  0.33 \\
\hline
Total & 63632.86 & 63637.83 & 4.43 \\
\hline
\hline
  \end{tabular}
\caption{Event rates broken down by sample after the posterior predictive spectrum for the fit to Asimov data}
\label{tab:asimov_posterior_pred}
\end{table}

\begin{figure}[h]
	\begin{subfigure}[t]{\textwidth}
	\begin{subfigure}[t]{0.49\textwidth}
		\includegraphics[width=\textwidth, trim={0mm 10mm 0mm 11mm}, clip,page=5]{figures/mach3/Asimov/2017b_NewDet_3Xsec_4Det_5Flux_NewXSecTune_Asimov_merge_PostPred_procs}
	\end{subfigure}
	\begin{subfigure}[t]{0.49\textwidth}
		\includegraphics[width=\textwidth, trim={0mm 10mm 0mm 11mm}, clip,page=7]{figures/mach3/Asimov/2017b_NewDet_3Xsec_4Det_5Flux_NewXSecTune_Asimov_merge_PostPred_procs}
	\end{subfigure}
\caption{FGD1 CC0$\pi$}
\end{subfigure}

\begin{subfigure}[t]{\textwidth}
\begin{subfigure}[t]{0.49\textwidth}
	\includegraphics[width=\textwidth, trim={0mm 10mm 0mm 11mm}, clip,page=23]{figures/mach3/Asimov/2017b_NewDet_3Xsec_4Det_5Flux_NewXSecTune_Asimov_merge_PostPred_procs}
\end{subfigure}
\begin{subfigure}[t]{0.49\textwidth}
	\includegraphics[width=\textwidth, trim={0mm 10mm 0mm 11mm}, clip,page=25]{figures/mach3/Asimov/2017b_NewDet_3Xsec_4Det_5Flux_NewXSecTune_Asimov_merge_PostPred_procs}
\end{subfigure}
\caption{FGD2 CC0$\pi$}
\end{subfigure}
	\caption{Posterior predictive \pmu \cosmu spectrum data/post-fit ratios and bin-by-bin likelihood contributions for the fit to Asimov data}
	\label{fig:posterior_predictive_2d_asimov}
\end{figure}

The MCMC was run on the Asimov data fit, presented in \autoref{sec:asimov_fit}. The longest chain in \autoref{tab:asimov_mcmc_versions} was selected for the calculation of these spectra. The parameter sets are drawn randomly from the MCMC after burn-in (1/4, or 350,000 steps), which was chosen after stability was observed in the parameters and likelihoods, and the autocorrelations tended to zero. 

The result is shown in \autoref{fig:posterior_predictive_asimov}, and as expected the p-value is 1.0. Comparing the y-axis ($-2\mathcal{L}_{\text{Sample}} \text{(Data, Draw)}$) to the prior predictive spectrum in \autoref{fig:prior_predictive_asimov}, the difference is almost one order of magnitude, essentially reflecting the much tighter model constraint before and after using ND280 data.
\begin{figure}[h]
	\begin{subfigure}[t]{0.49\textwidth}
		\includegraphics[width=\textwidth, trim={0mm 0mm 0mm 11mm}, clip,page=1]{figures/mach3/Asimov/2017b_NewDet_3Xsec_4Det_5Flux_NewXSecTune_Asimov_merge_PostPred_procs}
	\end{subfigure}
	\begin{subfigure}[t]{0.49\textwidth}
		\includegraphics[width=\textwidth, trim={0mm 0mm 0mm 11mm}, clip,page=2]{figures/mach3/Asimov/2017b_NewDet_3Xsec_4Det_5Flux_NewXSecTune_Asimov_merge_PostPred_procs}
	\end{subfigure}
\caption{Posterior predictive p-values for the fit to Asimov data}
\label{fig:posterior_predictive_asimov}
\end{figure}

\subsection{Covariance matrix}
\label{sec:covariance_asimov}
The last consistency check is to inspect the covariance matrix after the fitting to the Asimov data. The expected outcome is heavily correlated flux parameters---roughly retaining the covariances from the input covariance matrix---, interaction parameters correlating internally for parameters that affect the same modes and topologies (e.g. $M_A^{QE}$ and BeRPA), and correlations between the flux parameters and cross-section parameters, especially for normalisation parameters.

In \autoref{fig:asimov_full_corr} we present the full flux and cross-section parameter square root covariance and correlation matrix. The red square in the bottom left corner is the block of 100 flux parameters, all heavily internally correlated. The upper right corner is occupied by the cross-section parameters, which has some internal correlations, but far less than the fluxes. Many cross-section parameters correlate with the flux parameters, notably CC DIS---which is parameterised as $0.4/E_\nu$, so is essentially a flux normalisation weight for DIS events---, $C_5^A$---which largely controls the CC$1\pi$ nucleon level cross-section normalisation, and the BeRPA parameters---which control the $Q^2$ correction of CCQE events due to the RPA effects.
\begin{figure}[h]
	\begin{subfigure}[t]{0.49\textwidth}
		\includegraphics[width=\textwidth, trim={0mm 0mm 0mm 0mm}, clip,page=2]{figures/mach3/Asimov/2017b_NewDet_NewData_Asimov_Long_0_drawCorr.pdf}
		\caption{$\sqrt{\mathbf{V}_{i,j}}$}
	\end{subfigure}
	\begin{subfigure}[t]{0.49\textwidth}
		\includegraphics[width=\textwidth, trim={0mm 0mm 0mm 0mm}, clip,page=3]{figures/mach3/Asimov/2017b_NewDet_NewData_Asimov_Long_0_drawCorr.pdf}
		\caption{$\rho_{i,j}$}
	\end{subfigure}
	\caption{$\sqrt{\mathbf{V}_{i,j}}$ and correlation matrix for the Asimov post-fit, showing the full flux and cross-section parameters}
	\label{fig:asimov_full_corr}
\end{figure}

A more digestible version of \autoref{fig:asimov_full_corr} is shown in \autoref{fig:asimov_nd_corr}, which excludes the SK flux parameters in the matrices. Here we spot the largest post-fit flux uncertainty from the \nue flux parameters, and that the high energy flux parameters correlate only weakly, as expected from the neutrino production parents at high energies compared to low energies. We also observe very strong correlations for the low energy FHC $\nu_\mu$ and RHC $\bar{\nu}_\mu$ parameters. We also see the low energy flux parameters correlating with all the 2p2h parameters.
\begin{figure}[h]
	\begin{subfigure}[t]{0.49\textwidth}
		\includegraphics[width=\textwidth, trim={0mm 0mm 0mm 0mm}, clip,page=5]{figures/mach3/Asimov/2017b_NewDet_NewData_Asimov_Long_0_drawCorr.pdf}
	\end{subfigure}
	\begin{subfigure}[t]{0.49\textwidth}
		\includegraphics[width=\textwidth, trim={0mm 0mm 0mm 0mm}, clip,page=6]{figures/mach3/Asimov/2017b_NewDet_NewData_Asimov_Long_0_drawCorr.pdf}
	\end{subfigure}
	\caption{$\sqrt{\mathbf{V}_{i,j}}$ and correlation matrix for the Asimov post-fit, showing ND280 flux and cross-section parameters}
	\label{fig:asimov_nd_corr}
\end{figure}

\autoref{fig:asimov_flux_corr} shows the flux parameters with the input prior covariance matrix and the output post-fit Asimov covariance matrix. The correlations are largely the same, and we notice the smaller covariance values post-fit compared to the pre-fit, reflecting the tighter constraints on the flux parameters in the Asimov fit.
\begin{figure}[h]
	\begin{subfigure}[t]{\textwidth}
	\begin{subfigure}[t]{0.49\textwidth}
		\includegraphics[width=\textwidth, trim={0mm 0mm 0mm 0mm}, clip,page=8]{figures/mach3/Asimov/2017b_NewDet_NewData_Asimov_Long_0_drawCorr.pdf}
	\end{subfigure}
	\begin{subfigure}[t]{0.49\textwidth}
		\includegraphics[width=\textwidth, trim={0mm 0mm 0mm 0mm}, clip,page=9]{figures/mach3/Asimov/2017b_NewDet_NewData_Asimov_Long_0_drawCorr.pdf}
	\end{subfigure}
\caption{Post-fit}
\end{subfigure}

\begin{subfigure}[t]{\textwidth}
\begin{subfigure}[t]{0.49\textwidth}
	\includegraphics[width=\textwidth, trim={0mm 0mm 0mm 0mm}, clip,page=2]{figures/mach3/inputs/flux_covariance_banff_13av2.pdf}
\end{subfigure}
\begin{subfigure}[t]{0.49\textwidth}
	\includegraphics[width=\textwidth, trim={0mm 0mm 0mm 0mm}, clip,page=3]{figures/mach3/inputs/flux_covariance_banff_13av2.pdf}
\end{subfigure}
\caption{Pre-fit}
\end{subfigure}
	\caption{$\sqrt{\mathbf{V}_{i,j}}$ and correlation matrix for the flux parameters pre and post-fit to Asimov data}
	\label{fig:asimov_flux_corr}
\end{figure}

\autoref{fig:asimov_xsec_corr} shows the covariance and correlation matrices zoomed in on the cross-section parameters (the upper right corner of \autoref{fig:asimov_full_corr} and \autoref{fig:asimov_nd_corr}). As expected we observe correlations between C and O parameters (e.g. CC Coh C and CC Coh O), $\nu$ and $\bar{\nu}$ parameters (e.g. 2p2h norm). The CC0$\pi$ parameters (lower left block) correlate $M_A^{QE}$, $p_F$ and the BeRPA parameters, which all affect the CCQE interaction. We also see strong internal correlations in the BeRPA parameters, as expected. The single pion parameter block (middle) roughly maintain the prior correlation, as does the FSI block (top right corner). The single pion parameter $C_5^A$, $M_A^{RES}$ and non-resonant $I_{1/2}$ parameters correlate with the CC coherent parameters---which produce a 1$\pi^\pm$ final state---and the CC DIS parameter---all because they populate the CC1$\pi$ selection (and to a less extent the CCOther). We also note slight correlations between the CCQE and single pion parameters, due to the 20\% single pion events seen in \autoref{tab:nominal_mode_afterscale} in the CC0$\pi$ selection from final state pion interactions.
\begin{figure}[h]
	\begin{subfigure}[t]{0.49\textwidth}
		\includegraphics[width=\textwidth, trim={0mm 0mm 0mm 0mm}, clip,page=11]{figures/mach3/Asimov/2017b_NewDet_NewData_Asimov_Long_0_drawCorr.pdf}
	\end{subfigure}
	\begin{subfigure}[t]{0.49\textwidth}
		\includegraphics[width=\textwidth, trim={0mm 0mm 0mm 0mm}, clip,page=12]{figures/mach3/Asimov/2017b_NewDet_NewData_Asimov_Long_0_drawCorr.pdf}
	\end{subfigure}
	\caption{$\sqrt{\mathbf{V}_{i,j}}$ and correlation matrix for the Asimov post-fit, showing cross-section parameters}
	\label{fig:asimov_xsec_corr}
\end{figure}
