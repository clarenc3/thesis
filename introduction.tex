\chapter{Introduction}
\label{chap:intro}
Neutrino oscillation physics has entered the precision era with the later stages of T2K\cite{t2k_2017} and NO$\nu$A\cite{nova_2018}, building towards the high statistics Hyper-Kamiokande\cite{hyperk} and DUNE\cite{dune} experiments. Measurements in the future aim to reveal if neutrinos oscillate differently to anti-neutrinos, to explore the ordering of the neutrino mass states, and the completeness of the PMNS parameterisation\cite{p1,p2,mns}. Thanks to excellent beam performance with increasing power, both T2K and NO$\nu$A have successfully run in neutrino and anti-neutrino dominated beams and see consistent results in the latest analyses\cite{t2k_2017, nova_2018}, with updates presented in Heidelberg at Neutrino 2018\cite{t2k_neutrino2018, nova_neutrino2018}. As the statistics amount, there is a concerted effort towards percent-level precision in the community, and a future combination of T2K and NO$\nu$A results has been agreed\cite{t2k_nova, t2k_nova_meet} to fully exploit the strengths of each experiment.

Hand-in-hand with measuring precision neutrino mixing parameters, detailed study of the $\nu_\mu \rightarrow \nu_e$ process at T2K shares parameters with short baseline reactor experiments such as Daya-Bay\cite{daya_bay} and RENO\cite{reno}, measuring $\bar{\nu}_e\rightarrow \bar{\nu}_e$. Differences in $\theta_{13}$ and $\Delta m^2_{32}$ from the two neutrino sources is a probe of unmodelled physics. Similar is true for the IceCube\cite{icecube} and Super-Kamiokande\cite{superk} neutrino observatories measuring atmospheric neutrino mixing, primarily from $\nu_\mu \rightarrow \nu_\mu$.

This thesis concerns the use of external and internal neutrino scattering data at T2K to aid the precision measurements of neutrino oscillations. By carefully studying systematics present at both ND280 and Super-Kamiokande, the uncertainties on the event rates at SK are decreased from 12-14\% to 2-4\%. Such stringent model constraints enable world-leading results on $\sin^2\theta_{23}$, $\Delta m^2_{32}$ and $\delta_{CP}$ and competitive results on $\sin^2\theta_{13}$ from T2K\cite{pdg_2017}, and opens the door to the first observation of $\bar{\nu}_\mu \rightarrow \bar{\nu}_e$, with strong indications from NO$\nu$A at Neutrino 2018\cite{nova_neutrino2018}.

The studies also shine light on degeneracies and weaknesses of the T2K simulation from systematic sources, notably the neutrino flux and interaction modelling. By focusing effort on highlighted systematics, uncertainties can be driven down further. Improving model selection and availability is also crucial to searches at T2K involving ND280 only, such as sterile neutrinos and rare neutrino-matter interaction searches.

This thesis is organised to first present a historical overview alongside the theoretical framework in \autoref{chap:theory}. The fitting procedure using Markov Chain Monte Carlo methods is introduced in \autoref{chap:mcmc}. It then introduces T2K on the stage of neutrino oscillation physics in \autoref{chap:detectors}, giving an overview of the near detectors---ND280 and INGRID---and the far detector---Super-Kamiokande. An overview of the T2K oscillation analysis chain is given and the importance of using ND280 data is highlighted in \autoref{chap:ND280}. The selections at ND280 are presented, the parameterisation of systematics is detailed, and is followed by a fit to Asimov data and then real data, rounded off by a discussion of the impact on T2K oscillation analysis. The results from that analysis chapter were used in the 2017 official results\cite{t2k_2017}, and are being used for the 2018 official results presented at Neutrino 2018\cite{t2k_neutrino2018}.

The second analysis chapter in \autoref{chap:nd280_2018} concerns updating the fit to ND280 data using new selections and systematics, with an almost twofold increase in statistic in both neutrinos and anti-neutrinos. These constraints are to be used beyond the 2018 analysis and present stronger constraints yet.

The thesis finally closes with a summary and concluding remarks.
