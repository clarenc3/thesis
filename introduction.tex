\chapter{Introduction}
\label{chap:intro}
Neutrino oscillation physics is entering the precision era with the later stages of T2K\cite{t2k_2017} and NO$\nu$A\cite{nova_2018}, building towards the high statistics Hyper-Kamiokande\cite{hyperk} and DUNE\cite{dune} experiments. Nature will soon reveal if neutrinos oscillate differently to anti-neutrinos, the ordering of the neutrino mass states, and the completeness of the PMNS parameterisation\cite{p1,p2,mns}. Thanks to excellent beam performance with increasing power, both T2K and NO$\nu$A have successfully run in neutrino and anti-neutrino dominated beams and see consistent results in the latest analyses, with fresh updates presented in Heidelberg at Neutrino 2018\cite{t2k_neutrino2018, nova_neutrino2018}. As the statistics mount, there is a concerted effort towards percent-level systematics in the community, and a future combination of T2K and NO$\nu$A results has been agreed to fully exploit the strengths of each experiment\cite{t2k_nova, t2k_nova_meet}. Reaching small uncertainties on oscillation parameters in the GeV region requires significantly improved interaction and flux modelling over the prior\cite{t2k_ii, dune_exp}.

Hand-in-hand with detailed study of $\nu_e$ appearance at T2K and NO$\nu$A, the channel shares oscillation parameters with short baseline reactor experiments such as Daya-Bay\cite{daya_bay}, measuring $\bar{\nu}_e\rightarrow \bar{\nu}_e$. Differences in oscillation parameters from the two neutrino sources is a probe of un-modelled physics. Similar is true for the IceCube\cite{icecube} and Super-Kamiokande\cite{superk} neutrino observatories, measuring solar and atmospheric neutrinos.

This work concerns the use of external and internal neutrino scattering data at T2K to aid precision measurements of neutrino oscillations. By carefully studying systematics present at both ND280 and Super-Kamiokande, the uncertainties on event rates at SK are decreased from 12-14\% to 2-4\%. Such stringent model constraints enable world-leading oscillation measurements from T2K\cite{pdg_2017}, and opens the door to the first observation of $\bar{\nu}_\mu \rightarrow \bar{\nu}_e$, with strong indications from NO$\nu$A presented at Neutrino 2018\cite{nova_neutrino2018}.

The studies also shine light on degeneracies and weaknesses of the T2K simulation from systematic sources, notably the neutrino flux and interaction modelling. By focusing effort on highlighted systematics, uncertainties can be driven down further. Narrowing down model selection is also crucial to searches at T2K involving ND280 only, such as sterile neutrinos and rare neutrino interaction searches.

This thesis is organised to first present a historical overview alongside theory in \autoref{chap:theory}. It then introduces T2K on the stage of neutrino oscillation physics in \autoref{chap:detectors}, giving an overview of the beamline, the ND280 and INGRID near detectors and the Super-Kamiokande far detector. The fitting procedure using Markov Chain Monte Carlo is introduced in \autoref{chap:mcmc}, followed by an overview of the T2K oscillation analysis chain and the importance of ND280 data in \autoref{chap:ND280}. The selections at ND280 are presented, the parameterisation of systematics is detailed, and a fit to Asimov data and then real data is made, rounded off by a discussion of the impact on T2K oscillation analysis.

The second analysis chapter in \autoref{chap:nd280_2018} concerns updating the fit to ND280 data using new selections and systematics, with an almost twofold increase in statistics for both neutrinos and anti-neutrinos. The model constraints are to be used beyond the 2018 analysis, further reducing uncertainties on expected rates at Super-Kamiokande. The thesis finally closes with a summary and concluding remarks.

The presented results are used in the official T2K 2017\cite{t2k_2017} and upcoming 2018 publications, including those presented at Neutrino 2018\cite{t2k_neutrino2018}.