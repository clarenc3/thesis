\chapter{Updating the ND280 fit for 2018+}
The official 2017 analysis presented in \autoref{chap:ND280} used data from T2K runs 2 through 6, which was collected between 2010 and 2015. It was largely developed for the 2015 analyses, with an inconsistent 0$\pi$, 1$\pi$ and Other selection for \numu in FHC, and 1Track and NTrack for \numubar and \numu in RHC. Primarily due to low statistics and the 1Trk/NTrk selection, the posterior predictive p-values for the anti-neutrino samples were generally uninformative---centered around 0.5---implying not much could be said about the health of T2K anti-neutrino modelling.

For the upcoming 2018+ analyses, the near-detector fit has several updates which further pushes the uncertainties down for T2K-SK analysers, which will be presented in this chapter.

\section{Adding run 7 and 8 data}
Adding to the run 2 to 6 data, T2K has been collecting POT steadily since 2015, making it possible to refine the selections and update the binning for increased parameter sensitivity. As previously shown in \autoref{fig:t2k_pot}, almost double the amount of protons-on-target were accumulated in run 7 (RHC) and 8 (FHC) due to an increasing beam power, culminating at nearly 500 kW.

\autoref{tab:pot_2018} shows the run-by-run breakdown of the data and generated Monte-Carlo POT, directly comparable to the 2017 equivalent for run 2 to 6 in \autoref{tab:pot_2017}. The amount of FHC data increased by 99\% and RHC data by 63\%.
\begin{table}[h]
	\centering
	\begin{tabular}{ l c c c }
		\hline
		\hline
		Run &  \multicolumn{3}{c}{POT (E+19)} \\
		    & 	Data & MC & Sand \\
		\hline
		2a  & 3.59337    & 92.3937     & 3.7132  \\
		2w  & 4.33765    & 120.341     & 4.00035 \\
		\hline
		3b  & 2.1705     & 44.7864     & 2.35053 \\
		3c  & 13.6398    & 263.227     & 13.1337 \\
		\hline
		4a  & 17.8271    & 349.96      & 17.4125 \\
		4w  & 16.4277    & 226.216     & 15.9801 \\
		\hline
		5   & 4.3468     & 229.627     & 9.07403 \\
		\hline
		6b  & 12.7301    & 141.74      & 25.9187 \\
		6c  & 5.07819    & 52.7562     & 10.4626 \\
		6d  & 7.75302    & 68.83       & 15.8059 \\
		6e  & 8.51429    & 85.9439     & 17.2691 \\
		\hline
		7   & 24.3683	 & 337.059     & 50.3961 \\
		\hline
		8a  & 41.4909	 & 363.054	   & 40.1875 \\
		8w  & 15.8053    & 264.115 	   & 16.1263 \\
		\hline
		Total FHC & 115.29232 & 1724.0931 &  112.09418 \\
		Total RHC & 62.791 	  & 915.9561  &  127.92643 \\
		\hline
		\hline
		Total 	  & 178.08332 & 2640.0492 &  240.83061 \\
		\hline
		\hline
		Total FHC x2017 & 1.9876 & --- & --- \\
		Total RHC x2017 & 1.6276 & --- & --- \\
		Total x2017 	& 1.8438 & --- & --- \\
		\hline
		\hline
	\end{tabular}
	\caption{Counted and generated proton-on-targets for the T2K ND280 2018+ analysis}
	\label{tab:pot_2018}
\end{table}

\section{Selections}
For the selection update we only change the RHC sample from 1Track/NTrack to 0$\pi$, 1$\pi$ and Other. The FHC selection remains identical to as was presented in \autoref{sec:numu_sel}. The pion counting for the RHC sample matches that of \autoref{sec:numu_sel}, using either the TPC, FGD Michel electron or FGD isolated track reconstruction for the pion tag. 

In summary, the TPC pion PID requires that the track (with $p_{reco}<500\text{ MeV}$) has a MIP likelihood of $\mathcal{L}_{MIP} > 0.8$ (\autoref{eq:tpc_track_mip}) and a pion likelihood of $\mathcal{L}_\pi > 0.3$. The FGD is also used to search for pion-like deposits using a Michel electron tag which searches for a time-delayed FGD hit cluster, and a FGD reconstructed pion tag using an optimised pion pull cut for forwards-going events ($|\cos\theta_{\pi,\nu}|>0.3$).

The only difference in the likelihood cuts from \autoref{sec:ND280:sel} is for the $\mu^+$ selection, which in \autoref{sec:numubar_sel} required $0.1 < \mathcal{L}_\mu < 0.7$. The upper bound at 0.7 was present to reject low energy $\mu^-$ from \numu RHC interactions, which was discarded for this analysis after it gave a 4\% increase in efficiency with negligible purity change. The $\mu^-$ criteria for the \numu in RHC selection did not change and still requires a MIP-like track with $0.1 < \mathcal{L}_\mu < 0.8$.

Here we study the efficiencies and purities in the same way as \autoref{sec:ND280:sel}, using unweighted raw Monte-Carlo events.

\subsection{\numu in FHC}
Since the FHC selection is unchanged to the 2017 analysis, the efficiency and purities are very similar and here we only compare FGD1 CC0$\pi$ in 2018 to 2017 in \autoref{fig:fgd1_cc0pi_eff_2017_2018}, and refer to \autoref{tab:eff_pur_summary_2018} for the summary.
\begin{figure}[h]
	\centering
	\caption*{2018 analysis}
	\begin{subfigure}[t]{\textwidth}
		\centering
	\begin{subfigure}[t]{0.4\textwidth}
		\includegraphics[width=\textwidth,page=1, trim={0mm 0mm 0mm 9mm}, clip]{figures/mach3/2018/Selection/2018_RedNDmatrix_rebin_verbose_may_noweights_diagnostics}
		\caption{Efficiency}
	\end{subfigure}
	\begin{subfigure}[t]{0.4\textwidth}
		\includegraphics[width=\textwidth,page=2, trim={0mm 0mm 0mm 9mm}, clip]{figures/mach3/2018/Selection/2018_RedNDmatrix_rebin_verbose_may_noweights_diagnostics}
		\caption{Purity}
	\end{subfigure}
\end{subfigure}

	\begin{subfigure}[t]{\textwidth}
		\centering
		\caption*{2017 analysis}
	\begin{subfigure}[t]{0.4\textwidth}
	\includegraphics[width=\textwidth,page=1, trim={0mm 0mm 0mm 9mm}, clip]{figures/mach3/selection/2017b_Diag_WithSelection}
	\caption{Efficiency}
\end{subfigure}
\begin{subfigure}[t]{0.4\textwidth}
	\includegraphics[width=\textwidth,page=2, trim={0mm 0mm 0mm 9mm}, clip]{figures/mach3/selection/2017b_Diag_WithSelection}
	\caption{Purity}
\end{subfigure}
\end{subfigure}
\caption{FGD1 0$\pi$ efficiencies and purities for 2017 and 2018 analyses}
\label{fig:fgd1_cc0pi_eff_2017_2018}
\end{figure}

\subsection{\numubar in RHC}
The CC0$\pi$ RHC selections are much the same as the 2017 1Track selection, and as such the purity in \autoref{fig:numubar_cc0pi_topology_2018} is almost identical to \autoref{fig:ccnubar1trk_topology}. Both FGDs have similar purities and the largest contamination is right-sign single events in which the pion is missed, at about 9.5\%. The NC contribution is 5\% and the total wrong-sign contribution is $\sim5\%$, similar to the 1Trk selection. The difference comes primarily from the removed upper bound on the muon likelihood cut. Moving up in momentum, the purity reduces to about 60\% from 85\% at the event peak.
\begin{figure}[h]
	\begin{subfigure}[t]{0.49\textwidth}
		\includegraphics[width=\textwidth,page=13, trim={0mm 0mm 0mm 9mm}, clip]{figures/mach3/2018/Selection/2018_RedNDmatrix_rebin_verbose_may_noweights_diagnostics}
		\caption{FGD1}
	\end{subfigure}
	\begin{subfigure}[t]{0.49\textwidth}
		\includegraphics[width=\textwidth,page=19, trim={0mm 0mm 0mm 9mm}, clip]{figures/mach3/2018/Selection/2018_RedNDmatrix_rebin_verbose_may_noweights_diagnostics}
		\caption{FGD2}
	\end{subfigure}
	\caption{Breakdown of \numubar CC0$\pi$ selection events' true event topology for FGD1 and FGD2 }
	\label{fig:numubar_cc0pi_topology_2018}
\end{figure}

\autoref{fig:numubar_cc0pi_muon_2018} shows the muon tagging efficiency, which again is comparable to the 1Trk equivalent in \autoref{fig:ccnubar1trk_muon}. The largest mis-id comes from $\pi^+$ being reconstructed as the muon, and the wrong-sign contamination is small. At $\sim1.5\text{ GeV}$ we see the characteristic proton bump---which makes up 3\% of the total---in which the dE/dx of a proton is very similar to that of a muon, causing it to be the selected highest momentum positive track with a muon likelihood.
\begin{figure}[h]
	\begin{subfigure}[t]{0.49\textwidth}
		\includegraphics[width=\textwidth,page=14, trim={0mm 0mm 0mm 9mm}, clip]{figures/mach3/2018/Selection/2018_RedNDmatrix_rebin_verbose_may_noweights_diagnostics}
		\caption{FGD1}
	\end{subfigure}
	\begin{subfigure}[t]{0.49\textwidth}
		\includegraphics[width=\textwidth,page=20, trim={0mm 0mm 0mm 9mm}, clip]{figures/mach3/2018/Selection/2018_RedNDmatrix_rebin_verbose_may_noweights_diagnostics}
		\caption{FGD2}
	\end{subfigure}
	\caption{Breakdown of \numubar CC0$\pi$ selection events' true lepton candidate for FGD1 and FGD2}
	\label{fig:numubar_cc0pi_muon_2018}
\end{figure}

Moving to the 1$\pi$ \numubar selection, the purity is shown in \autoref{fig:numubar_cc1pi_topology_2018}. The \numubar RHC selection sees similar performance to the \numu FHC equivalent in \autoref{fig:cc1pi_topology_2018}, reaching an overall purity of $50-54\%$, with FGD1 being the higher. The wrong-sign contamination is significant at 25-30\%, coming from a $\pi^+$ in a CC1$\pi^+$ or CC DIS event being reconstructed as the muon candidate, and the $\mu^-$ reconstructed as the $\pi^-$. The right-sign CCOther contamination is about 10\%, owing mostly to one or several missed $\pi$. We again see the CC0$\pi$ \numu peak at 1.5 GeV, where the proton (likely from a CCQE or 2p2h interaction) is reconstructed as the $\mu^+$.
\begin{figure}[h]
	\begin{subfigure}[t]{0.49\textwidth}
		\includegraphics[width=\textwidth,page=15, trim={0mm 0mm 0mm 9mm}, clip]{figures/mach3/2018/Selection/2018_RedNDmatrix_rebin_verbose_may_noweights_diagnostics}
		\caption{FGD1}
	\end{subfigure}
	\begin{subfigure}[t]{0.49\textwidth}
		\includegraphics[width=\textwidth,page=21, trim={0mm 0mm 0mm 9mm}, clip]{figures/mach3/2018/Selection/2018_RedNDmatrix_rebin_verbose_may_noweights_diagnostics}
		\caption{FGD2}
	\end{subfigure}
	\caption{Breakdown of \numubar CC1$\pi$ selection events' true event topology for FGD1 and FGD2 }
	\label{fig:numubar_cc1pi_topology_2018}
\end{figure}

\autoref{fig:numubar_cc1pi_muon_2018} shows the muon tagging efficiency, which in the event peak sits at 80\% and decreases to 50\% with increasing $p_{\text{reco}}$. Tagging the right-sign pion as the lepton candidate happens 20\% of the time, and protons at $p\sim1.5\text{ GeV}$ about 15\% of the time, making up a large fraction of mis-identification. However, the 1$\pi$ selection is still almost 10\% more efficient and pure than the old CCNTrack selection.
\begin{figure}[h]
	\begin{subfigure}[t]{0.49\textwidth}
		\includegraphics[width=\textwidth,page=16, trim={0mm 0mm 0mm 9mm}, clip]{figures/mach3/2018/Selection/2018_RedNDmatrix_rebin_verbose_may_noweights_diagnostics}
		\caption{FGD1}
	\end{subfigure}
	\begin{subfigure}[t]{0.49\textwidth}
		\includegraphics[width=\textwidth,page=22, trim={0mm 0mm 0mm 9mm}, clip]{figures/mach3/2018/Selection/2018_RedNDmatrix_rebin_verbose_may_noweights_diagnostics}
		\caption{FGD2}
	\end{subfigure}
	\caption{Breakdown of \numubar CC1$\pi$ selection events' true lepton candidate for FGD1 and FGD2}
	\label{fig:numubar_cc1pi_muon_2018}
\end{figure}

Finally \autoref{fig:numubar_ccOth_topology_2018} shows the purity of the \numubar CCOther selection, which collects all \numubar CC candidates that weren't classified as CC0$\pi$ or CC1$\pi$. As with the equivalent \numu selection, the sample suffers from low purity due to broken tracks and secondary interactions, leading to a mis-reconstructed number of pions in the event. The selection has an almost equal efficiency for \numubar CCOther events as it does for \numu CCOther events, and in FGD2 it's indeed more pure of wrong-sign events. It has a high NC contamination due to collecting high pion multiplicity events, causing a pion to be reconstructed as a muon in the TPC.  At low momentum, the purity is close to zero, being swamped by wrong-sign 0$\pi$ events in which the low momentum $\mu^-$ is identified as a $\mu^+$, owing to the changed likelihood cut which in the 2017 analysis was present to remove such events. The wrong-sign 0$\pi$ and 1$\pi$ contributions largely vanish above 500 MeV and the wrong sign component is almost exclusively \numu CCOther.
\begin{figure}[h]
	\begin{subfigure}[t]{0.49\textwidth}
		\includegraphics[width=\textwidth,page=17, trim={0mm 0mm 0mm 9mm}, clip]{figures/mach3/2018/Selection/2018_RedNDmatrix_rebin_verbose_may_noweights_diagnostics}
		\caption{FGD1}
	\end{subfigure}
	\begin{subfigure}[t]{0.49\textwidth}
		\includegraphics[width=\textwidth,page=23, trim={0mm 0mm 0mm 9mm}, clip]{figures/mach3/2018/Selection/2018_RedNDmatrix_rebin_verbose_may_noweights_diagnostics}
		\caption{FGD2}
	\end{subfigure}
	\caption{Breakdown of \numubar CCOther selection events' true event topology for FGD1 and FGD2 }
	\label{fig:numubar_ccOth_topology_2018}
\end{figure}

The muon tagging efficiency of the \numubar CCOther selection is shown in \autoref{fig:numubar_ccOth_muon_2018}, which echoes the conclusions above. The efficiency is below 50\% and has almost equal parts proton tagging and $\pi^+$ tagging as contaminants. The wrong sign tag happens primarily at low momentum, in which the charge is reconstructed in the magnetic field. The proton bump at 1.5 GeV is especially present in this selection.
\begin{figure}[h]
	\begin{subfigure}[t]{0.49\textwidth}
		\includegraphics[width=\textwidth,page=18, trim={0mm 0mm 0mm 9mm}, clip]{figures/mach3/2018/Selection/2018_RedNDmatrix_rebin_verbose_may_noweights_diagnostics}
		\caption{FGD1}
	\end{subfigure}
	\begin{subfigure}[t]{0.49\textwidth}
		\includegraphics[width=\textwidth,page=24, trim={0mm 0mm 0mm 9mm}, clip]{figures/mach3/2018/Selection/2018_RedNDmatrix_rebin_verbose_may_noweights_diagnostics}
		\caption{FGD2}
	\end{subfigure}
	\caption{Breakdown of \numubar CCOther selection events' true lepton candidate for FGD1 and FGD2}
	\label{fig:numubar_ccOth_muon_2018}
\end{figure}

\subsection{\numu in RHC}
As with the \numubar CC0$\pi$ selection, the \numu RHC CC0$\pi$ selection is largely identical to the 1Track equivalent in the 2017 analysis. The purity in \autoref{fig:numurhc_cc0pi_topology_2018} is above 53\%, with large contamination from right-sign 1$\pi$ and Other interactions, and the wrong-sign background making up 8\%, slightly less than the 1Track case. The NC contamination is almost identical to the 1Track selection at 9\%. We note the purity above 600 MeV stabilises at about 60\%.
\begin{figure}[h]
	\begin{subfigure}[t]{0.49\textwidth}
		\includegraphics[width=\textwidth,page=25, trim={0mm 0mm 0mm 9mm}, clip]{figures/mach3/2018/Selection/2018_RedNDmatrix_rebin_verbose_may_noweights_diagnostics}
		\caption{FGD1}
	\end{subfigure}
	\begin{subfigure}[t]{0.49\textwidth}
		\includegraphics[width=\textwidth,page=31, trim={0mm 0mm 0mm 9mm}, clip]{figures/mach3/2018/Selection/2018_RedNDmatrix_rebin_verbose_may_noweights_diagnostics}
		\caption{FGD2}
	\end{subfigure}
	\caption{Breakdown of \numu RHC CC0$\pi$ selection events' true event topology for FGD1 and FGD2 }
	\label{fig:numurhc_cc0pi_topology_2018}
\end{figure}

The muon tagging efficiency is shown in \autoref{fig:numurhc_cc0pi_muon_2018}, where we note 90\% above 1 GeV. At the event peak the efficiency sits at 55\%, leading to overall 78\%. In and below the event peak the main contamination is from $\pi^-$ (13\%) and as we go down in momentum the wrong-sign contributions increase due to wrongly reconstructing the charge in the magnet. At low momentum the wrong-sign component is $\times10$ larger than the right-sign.
\begin{figure}[h]
	\begin{subfigure}[t]{0.49\textwidth}
		\includegraphics[width=\textwidth,page=26, trim={0mm 0mm 0mm 9mm}, clip]{figures/mach3/2018/Selection/2018_RedNDmatrix_rebin_verbose_may_noweights_diagnostics}
		\caption{FGD1}
	\end{subfigure}
	\begin{subfigure}[t]{0.49\textwidth}
		\includegraphics[width=\textwidth,page=32, trim={0mm 0mm 0mm 9mm}, clip]{figures/mach3/2018/Selection/2018_RedNDmatrix_rebin_verbose_may_noweights_diagnostics}
		\caption{FGD2}
	\end{subfigure}
	\caption{Breakdown of \numu RHC CC0$\pi$ selection events' true lepton candidate for FGD1 and FGD2}
	\label{fig:numurhc_cc0pi_muon_2018}
\end{figure}

The CC1$\pi$ purity is shown in \autoref{fig:numurhc_cc1pi_topology_2018}, where we again see a large wrong-sign contribution at low momentum, primarily from \numubar 1$\pi$ events. The purity is 43\% overall, and a meagre 20\% in the event peak. The right-sign CCOther amount is constant with momentum, making up almost 1/3 at higher momentum.
\begin{figure}[h]
	\begin{subfigure}[t]{0.49\textwidth}
		\includegraphics[width=\textwidth,page=27, trim={0mm 0mm 0mm 9mm}, clip]{figures/mach3/2018/Selection/2018_RedNDmatrix_rebin_verbose_may_noweights_diagnostics}
		\caption{FGD1}
	\end{subfigure}
	\begin{subfigure}[t]{0.49\textwidth}
		\includegraphics[width=\textwidth,page=33, trim={0mm 0mm 0mm 9mm}, clip]{figures/mach3/2018/Selection/2018_RedNDmatrix_rebin_verbose_may_noweights_diagnostics}
		\caption{FGD2}
	\end{subfigure}
	\caption{Breakdown of \numu RHC CC1$\pi$ selection events' true event topology for FGD1 and FGD2 }
	\label{fig:numurhc_cc1pi_topology_2018}
\end{figure}

The muon tagging efficiency in \autoref{fig:numurhc_cc1pi_muon_2018} performs similarly to the NTrack selection at 65\%. At the event peak the efficiency is barely 20\%--the rest split almost equally amongst $\pi^-$, $\pi^+$ and $\mu^+$---but increases steadily to 85\% at higher momentum, where to wrong-sign component vanishes. The total wrong-sign contribution is 12\% but is dominant at low momentum. The $\pi^-$ contribution is sizeable at 22\%, which dies off at higher momentum.
\begin{figure}[h]
	\begin{subfigure}[t]{0.49\textwidth}
		\includegraphics[width=\textwidth,page=28, trim={0mm 0mm 0mm 9mm}, clip]{figures/mach3/2018/Selection/2018_RedNDmatrix_rebin_verbose_may_noweights_diagnostics}
		\caption{FGD1}
	\end{subfigure}
	\begin{subfigure}[t]{0.49\textwidth}
		\includegraphics[width=\textwidth,page=34, trim={0mm 0mm 0mm 9mm}, clip]{figures/mach3/2018/Selection/2018_RedNDmatrix_rebin_verbose_may_noweights_diagnostics}
		\caption{FGD2}
	\end{subfigure}
	\caption{Breakdown of \numu RHC CC1$\pi$ selection events' true lepton candidate for FGD1 and FGD2}
	\label{fig:numurhc_cc1pi_muon_2018}
\end{figure}

The \numu RHC CCOther selection's purity seen in \autoref{fig:numurhc_ccOth_topology_2018} is relatively high compared to other CC Other selection; overall 61\%. At low momentum the NC contribution is the largest, which is also the largest background overall at 13\%. Interestingly, the right-sign 0$\pi$ selection contaminates the sample 10\% and is the second largest contamination. Since the sign selection looks for a negative track for \numu selections, the CC0$\pi$ contribution can not come from a proton track being the muon candidate, and must be broken tracks being reconstructed as multiple pions.
\begin{figure}[h]
	\begin{subfigure}[t]{0.49\textwidth}
		\includegraphics[width=\textwidth,page=29, trim={0mm 0mm 0mm 9mm}, clip]{figures/mach3/2018/Selection/2018_RedNDmatrix_rebin_verbose_may_noweights_diagnostics}
		\caption{FGD1}
	\end{subfigure}
	\begin{subfigure}[t]{0.49\textwidth}
		\includegraphics[width=\textwidth,page=35, trim={0mm 0mm 0mm 9mm}, clip]{figures/mach3/2018/Selection/2018_RedNDmatrix_rebin_verbose_may_noweights_diagnostics}
		\caption{FGD2}
	\end{subfigure}
	\caption{Breakdown of \numu RHC CCOther selection events' true event topology for FGD1 and FGD2 }
	\label{fig:numurhc_ccOth_topology_2018}
\end{figure}

The corresponding muon efficiency is shown in \autoref{fig:numurhc_ccOth_muon_2018}, where we see close to zero efficiency at low momentum. In this region the electron is the principal muon candidate, but dies down above 200 MeV. After that the $\pi^-$ is the only competing background at $\sim20\%$. The overall efficiency is 68\% and stabilises at 1 GeV, coinciding with the event distribution peak.
\begin{figure}[h]
	\begin{subfigure}[t]{0.49\textwidth}
		\includegraphics[width=\textwidth,page=30, trim={0mm 0mm 0mm 9mm}, clip]{figures/mach3/2018/Selection/2018_RedNDmatrix_rebin_verbose_may_noweights_diagnostics}
		\caption{FGD1}
	\end{subfigure}
	\begin{subfigure}[t]{0.49\textwidth}
		\includegraphics[width=\textwidth,page=36, trim={0mm 0mm 0mm 9mm}, clip]{figures/mach3/2018/Selection/2018_RedNDmatrix_rebin_verbose_may_noweights_diagnostics}
		\caption{FGD2}
	\end{subfigure}
	\caption{Breakdown of \numu RHC CCOther selection events' true lepton candidate for FGD1 and FGD2}
	\label{fig:numurhc_ccOth_muon_2018}
\end{figure}

A summary of the 2018 analysis' selection efficiencies and purities is shown in \autoref{tab:eff_pur_summary_2018}.
\begin{table}[h]
	\centering
	\begin{tabular}{ l | c c }
		\hline
		\hline
		Selection 					   & Efficiency (\%) & Purity (\%) \\ 
		\hline
		\FGDCCNoPi{1}{\numu}           & 93.7  & 75.4  \\% \hline
		\FGDCCNoPi{2}{\numu}           & 93.0  & 73.3  \\% \hline
		\hline
		\FGDCCOnePi{1}{\numu}          & 83.4  & 57.3  \\% \hline
		\FGDCCOnePi{2}{\numu}          & 83.0  & 56.7  \\% \hline
		\hline
		\FGDCCOther{1}{\numu}          & 73.0  & 64.9  \\% \hline
		\FGDCCOther{2}{\numu}          & 73.4  & 64.9  \\% \hline
		\hline
		\FGDCCNoPi{1}{\numubar}           & 89.4  & 75.2  \\% \hline
		\FGDCCNoPi{2}{\numubar}           & 87.8  & 74.0  \\% \hline
		\hline
		\FGDCCOnePi{1}{\numubar}          & 65.0  & 53.5  \\% \hline
		\FGDCCOnePi{2}{\numubar}          & 61.0  & 49.6  \\% \hline
		\hline
		\FGDCCOther{1}{\numubar}          & 44.1  & 24.6  \\% \hline
		\FGDCCOther{2}{\numubar}          & 41.6  & 23.6  \\% \hline
		\hline		
		\FGDCCNoPi{1}{\numu RHC}           & 79.7  & 55.6  \\% \hline
		\FGDCCNoPi{2}{\numu RHC}           & 77.8  & 53.0 \\% \hline
		\hline
		\FGDCCOnePi{1}{\numu RHC}          & 65.7  & 43.4  \\% \hline
		\FGDCCOnePi{2}{\numu RHC}          & 66.8  & 43.1  \\% \hline
		\hline
		\FGDCCOther{1}{\numu RHC}          & 68.8  & 61.0  \\% \hline
		\FGDCCOther{2}{\numu RHC}          & 69.0  & 60.8  \\% \hline
		\hline
		\hline
	\end{tabular}
	\caption{Efficiency and purity summary for all selections with the range $0 < p_{reco} < 3\text{ GeV/c}$, directly comparable to \autoref{tab:eff_pur_summary}}
	\label{tab:eff_pur_summary_2018}
\end{table}

\section{Rebinning}
With the large increase in statistics from using run 7 and 8 data, a re-binning of the two reconstructed event observables \pmu and \cosmu is due. As was the case in \autoref{sec:binning_2017}, we base the binning on requiring $\sim20$ raw MC events per bin, which is roughly equivalent to 1-2 data events. We keep in mind the approximate momentum resolution of ND280 is $\sim50\text{ MeV}$ and the angular resolution $\sim2\degree$.

Here we re-bin all selections for 2018 analysis using the above criteria, leading to a drastic increase in the number of bins from the 2017 analysis, which was 1624. The total number of bins is now 4130, of which 2942 are FHC (six selections) and 1188 are RHC (12 selections). We note the FGD1 and 2 CC0$\pi$ binning alone (1682) is more bins than was present in total for 2017 (1624).

\begin{itemize}
	\item FGD1+2 CC0$\pi$: 841 fit bins\\
	$p_\mu$ (MeV/c) = 0, 200, 300, 400, 450, 500, 550, 600, 650, 700, 750, 800, 850, 900, 950, 1000, 1050, 1100, 1200, 1300, 1400, 1500, 1600, 1700, 1800, 2000, 2500, 3000, 5000, 30000.\\
	$\cos\theta_\mu$ = -1, 0.5, 0.6, 0.7, 0.76, 0.78, 0.8, 0.83, 0.85, 0.88, 0.89, 0.9, 0.91, 0.92, 0.925, 0.93, 0.935, 0.94, 0.945, 0.95, 0.955, 0.96, 0.965, 0.97, 0.975, 0.98, 0.985, 0.99, 0.995, 1.
	
	\item FGD1+2 CC1$\pi$: 288 fit bins\\
	$p_\mu$ (MeV/c) = 0, 300, 350, 400, 500, 600, 650, 700, 750, 800, 900, 1000, 1100, 1200, 1500, 2000, 3000, 5000, 30000.\\
	$\cos\theta_\mu$ = -1, 0.6, 0.7, 0.8, 0.85, 0.88, 0.9, 0.92, 0.93, 0.94, 0.95, 0.96, 0.97, 0.98, 0.99, 0.995, 1.
	
	\item FGD1+2 CCOther: 342 fit bins\\
	$p_\mu$ (MeV/c) = 0, 300, 400, 500, 600, 650, 700, 750, 800, 900, 1000, 1100, 1250, 1500, 1750, 2000, 3000, 5000, 30000.\\
	$\cos\theta_\mu$ = -1, 0.6, 0.7, 0.76, 0.8, 0.85, 0.88, 0.89, 0.9, 0.91, 0.92, 0.93, 0.94, 0.95, 0.96, 0.97, 0.98, 0.99, 0.995, 1.
	
	\item FGD1+2 CC0$\pi$ RHC: 306 fit bins\\
	$p_\mu$ (MeV/c) = 0, 300, 400, 500, 550, 600, 650, 700, 750, 800, 900, 1000, 1100, 1200, 1500, 2000, 4000, 30000.\\
	$\cos\theta_\mu$ = -1, 0.6, 0.7, 0.8, 0.85, 0.9, 0.92, 0.93, 0.94, 0.95, 0.96, 0.965, 0.97, 0.975, 0.98, 0.985, 0.99, 0.995, 1.
	
	\item FGD1+2 CC1$\pi$ RHC: 48 fit bins \\
	$p_\mu$ (MeV/c) = 0, 500, 700, 900, 1300, 2500, 30000.\\
	$\cos\theta_\mu$ = -1, 0.7, 0.8, 0.9, 0.94, 0.96, 0.98, 0.99, 1
	
	\item FGD1+2 CCOther RHC: 80 fit bins \\
	$p_\mu$ (MeV/c) = 0, 600, 800, 1000, 1250, 1500, 2000, 4000, 30000.\\
	$\cos\theta_\mu$ = -1, 0.7, 0.8, 0.85, 0.9, 0.93, 0.95, 0.97, 0.98, 0.99, 1.
	
	\item FGD1+2 CC0$\pi$ $\nu$ RHC: 120 fit bins \\
	$p_\mu$ (MeV/c) = 0, 300, 500, 700, 800, 900, 1250, 1500, 2000, 4000, 30000.\\
	$\cos\theta_\mu$ = -1, 0.7, 0.8, 0.85, 0.88, 0.9, 0.92, 0.94, 0.96, 0.97, 0.98, 0.99, 1.
	
	\item FGD1+2 CC1$\pi$ $\nu$ RHC: 40 fit bins \\
	$p_\mu$ (MeV/c) = 0, 600, 800, 1500, 30000.\\
	$\cos\theta_\mu$ = -1, 0.7, 0.8, 0.86, 0.9, 0.94, 0.96, 0.97, 0.98, 0.99, 1
	
	\item FGD1+2 CCOther $\nu$ RHC: 80 fit bins \\
	$p_\mu$ (MeV/c) = 0, 600, 1000, 1250, 2000, 4000, 30000.\\
	$\cos\theta_\mu$ = -1, 0.7, 0.8, 0.86, 0.9, 0.93, 0.95, 0.97, 0.99, 1.
\end{itemize}

\red{Put in the plots when they're done? Plot overload?}

\section{Systematics}
\subsection{Flux}
No change

\subsection{Cross-section}
Only new thing is FSI central values

and the CC norm nu nubar

\subsection{Making new detector systematics}
Also added proton SI

Deciding on detector binning: 
Check how similar percentage effect on systematics: if the percentage change is more than 5\% and the percentage effect of the systematics on the bin is greater than 5\% and the bin content in the bin is at least one and the effect of the systematics on the bin changes the number of events in that bin by at least one.
Will favour high-statistics bins and disfavour low-statistics bins.

VERIFIED IN 17 MAY AFTER BUG v7
\begin{itemize}
	\item FGD1 and FGD2 CC0$\pi$: 272 detector bins (841) \\
	$p_\mu$ (GeV/c): 0, 200, 300, 400, 450, 550, 600, 650, 700, 750, 800, 850, 900, 950, 1000, 1400, 5000, 30000\\
	$\cos\theta_\mu$: -1, 0.5, 0.6, 0.7, 0.76, 0.8, 0.83, 0.85, 0.88, 0.965, 0.97, 0.975, 0.98, 0.985, 0.99, 0.995, 1
	
	\item FGD1 and FGD2 CC1$\pi$: 110 detector bins (288) \\
	$p_\mu$ (GeV/c): 0, 300, 350, 400, 500, 600, 650, 700, 1100, 3000, 5000, 30000\\
	$\cos\theta_\mu$: -1, 0.6, 0.7, 0.8, 0.85, 0.88, 0.9, 0.92, 0.93, 0.94, 1
	
	\item FGD1 and FGD2 CCOther: 72 detector bins (342) \\
	$p_\mu$ (GeV/c): 0, 300, 400, 600, 650, 1750, 2000, 5000, 30000\\
	$\cos\theta_\mu$: -1, 0.6, 0.93, 0.94, 0.95, 0.96, 0.98, 0.99, 0.995, 1
	
	\item FGD1 and FGD2 CC0$\pi$ RHC: 49 detector bins (306) \\
	$p_\mu$ (GeV/c): 0, 300, 400, 500, 550, 2000, 4000, 30000\\
	$\cos\theta_\mu$: -1, 0.6, 0.7, 0.8, 0.85, 0.9, 0.96, 1 
	
	\item FGD1 and FGD2 CC1$\pi$ RHC: 4 detector bins (48) \\
	$p_\mu$ (GeV/c): 0, 500, 30000.\\
	$\cos\theta_\mu$: -1, 0.7, 1
	
	\item FGD1 and FGD2 CCOther RHC: 6 detector bins (80) \\
	$p_\mu$ (GeV/c): 0, 600, 800, 30000.\\
	$\cos\theta_\mu$: -1, 0.7, 1.
	
	\item FGD1 and FGD2 CC0$\pi$ RHC $\nu$: 15 detector bins (120) \\
	$p_\mu$ (GeV/c): 0, 300, 500, 700, 800, 30000.\\
	$\cos\theta_\mu$: -1, 0.7, 0.8, 1.
	
	\item FGD1 and FGD2 CC1$\pi$ RHC $\nu$: 6 detector bins (40) \\
	$p_\mu$ (GeV/c): 0, 600, 800, 30000.\\
	$\cos\theta_\mu$: -1, 0.7, 1
	
	\item FGD1 and FGD2 CCOther RHC $\nu$: 4 detector bins (54)\\
	$p_\mu$ (GeV/c): 0, 600, 30000.\\
	$\cos\theta_\mu$: -1, 0.7, 1.
\end{itemize}

show raw 4238 plots, binning choices for data

try to rebin with justification, show ndof etc, some canvases

\section{Nominal model}
Using the nominal model, applying the multiplicative nominal weights, the event rates in \autoref{tab:detailed_eventrate_2018} are obtained. 

As expected from the large increase in POT detailed in \autoref{tab:pot_2018}, run 7 and 8 almost doubles the data for the FHC and RHC selections. There are now 67,000 FHC \numu, 13,000 RHC \numubar, and 5,000 RHC \numu CC0$\pi$ events, totalling at 82,000. The FGD1 and FGD2 equivalent selections are consistent, and we generally note CC0$\pi$ selections underestimated by 0-6\%, CC1$\pi$ overestimated by the same amount, and CCOther underestimated by 9-26\% for all FGDs and beam running modes. Averaging over all the selections the nominal model underestimates the data by 6\%.

% Full matrix
%\begin{table}[h]
%	\centering
%	\begin{tabular}{ l c c c }
%		\hline
 %       \hline
%		Sample & Data & Nominal MC & Data/MC \\
%		\hline
%		FGD1 0$\pi$       & 33553 & 31530.5  & 1.06 \\
%		FGD1 1$\pi$       & 7757  & 7997.96  & 0.97 \\
%		FGD1 Other        & 8068  & 6793.11  & 1.18 \\
%		\hline
%		FGD2 0$\pi$       & 33462 & 31736.7 & 1.05\\
%		FGD2 1$\pi$       & 6133  & 6419.23 & 0.96 \\
%		FGD2 Other        & 7664  & 6563.14 & 1.17 \\
%		\hline
%		FGD1 \numubar 0$\pi$       & 6368 & 6371.09  & 1.00 \\
%		FGD1 \numubar 1$\pi$       & 535  & 533.187  & 1.00 \\
%		FGD1 \numubar Other        & 1032 & 1023.2  & 1.01 \\
%		\hline
%		FGD2 \numubar 0$\pi$       & 6451 & 6284.65  & 1.03  \\
%		FGD2 \numubar 1$\pi$       & 465  & 483.469  & 0.96 \\
%		FGD2 \numubar Other        & 1032 & 944.175 & 1.09 \\
%		\hline
%		FGD1 \numu RHC 0$\pi$       & 2707 & 2497.71 & 1.08 \\
%		FGD1 \numu RHC 1$\pi$       & 847  & 860.675 & 0.98 \\
%		FGD1 \numu RHC Other        & 1015 & 797.499 & 1.27 \\
%		\hline
%		FGD2 \numu RHC 0$\pi$       & 2648 & 2553.51 & 1.04 \\
%		FGD2 \numu RHC 1$\pi$       & 693  & 679.99  & 1.02 \\
%		FGD2 \numu RHC Other        & 932  & 792.166 & 1.18 \\
 %               \hline
%		Total & 121432 & 114862 & 1.06 \\
%		Total x2017 & 1.87 & 1.80 & \\
%		\hline
%		\hline
%	\end{tabular}
 %       \caption{Based on full matrix hAdNOT heppc205 8 Apr and DsHVqI on heppc105}
%	\label{tab:detailed_eventrate_2018}
%\end{table}

% Reduced matrix,  fLrRiy on 205
\begin{table}[h]
  \begin{tabular}{l c c c }
  	\hline
  	\hline
  	Sample & Data & Nominal MC & Data/MC \\
  	\hline
    FGD1 0$\pi$          & 33553     & 31529.3 & 1.06  \\
    FGD1 1$\pi$          & 7757      & 7998.1  & 0.97 \\
    FGD1 other           & 8068      & 6793.68 & 1.18 \\
    \hline
    FGD2 0$\pi$          & 33462     & 31734   & 1.05 \\
    FGD2 1$\pi$          & 6133      & 6419.04 & 0.96 \\
    FGD2 other           & 7664      & 6562.75 & 1.17 \\
    \hline
    FGD1 \numubar 0$\pi$       & 6368      & 6371.34 & 1.00 \\
    FGD1 \numubar 1$\pi$       & 535       & 533.253 & 1.00 \\
    FGD1 \numubar other        & 1102      & 1023.36 & 1.08 \\
    \hline
    FGD2 \numubar 0$\pi$       & 6451      & 6283.35 & 1.03\\
    FGD2 \numubar 1$\pi$       & 465       & 483.508 & 0.96 \\
    FGD2 \numubar other        & 1032      & 943.956 & 1.09 \\
    \hline
    FGD1 \numu RHC 0$\pi$ 	   & 2707      & 2485.51 & 1.09 \\
    FGD1 \numu RHC 1$\pi$		& 847      & 855.911 & 0.99 \\
    FGD1 \numu RHC other 	   & 1015      & 804.647 & 1.26\\
    \hline
    FGD2 \numu RHC 0$\pi$ 		& 2648      & 2553.51 & 1.04 \\
    FGD2 \numu RHC 1$\pi$ 		& 693       & 679.99  & 1.02 \\
    FGD2 \numu RHC other 		& 932       & 792.166 & 1.18 \\
    \hline
    Total                       & 121432  	& 114847 & 1.06 \\
    Total x2017					& 1.87 		& 1.80 \\
    \hline
    \hline
  \end{tabular}
  \caption{Observed and predicted event rates for the different ND280 selections for the 2018 analysis}
  \label{tab:detailed_eventrate_2018}
\end{table}

\autoref{fig:nominal2D_FGD1numu_2018} shows the data and nominal model prediction for the FGD1 FHC selections. We see in the restricted plotting region (excluding highest momentum and most backward bins, normalising to bin width), the data is consistently higher than the prediction. The CC0$\pi$ selection looks overestimated at higher momentum between \cosmu=0.8-0.95. We also see some clear underestimates along lines of constant $Q^2$, between 0.07 and 0.15 $\text{GeV}^2$. For the CC1$\pi$ selection we see a similar overestimate at high \pmu and \cosmu=0.8-0.95. For the CCOther selection we see a clear underestimate in almost all bins below $Q^2=0.2\text{ GeV}^2$. In general, the distributions are compatible and similar to the 2017 equivalents in \autoref{fig:nominal2D_FGD1numu}.
\begin{figure}[h]
	\begin{subfigure}[t]{0.32\textwidth}
		\includegraphics[width=\textwidth,page=1]{{figures/mach3/2018/Selection/2018_RedNDmatrix_rebin_verbose_may_noweights_ND280_nom}}
	\end{subfigure}
	\begin{subfigure}[t]{0.32\textwidth}
		\includegraphics[width=\textwidth,page=2]{{figures/mach3/2018/Selection/2018_RedNDmatrix_rebin_verbose_may_noweights_ND280_nom}}
	\end{subfigure}
	\begin{subfigure}[t]{0.32\textwidth}
		\includegraphics[width=\textwidth,page=3]{{figures/mach3/2018/Selection/2018_RedNDmatrix_rebin_verbose_may_noweights_ND280_nom}}
	\end{subfigure}
	
	\begin{subfigure}[t]{0.32\textwidth}
		\includegraphics[width=\textwidth,page=4]{{figures/mach3/2018/Selection/2018_RedNDmatrix_rebin_verbose_may_noweights_ND280_nom}}
	\end{subfigure}
	\begin{subfigure}[t]{0.32\textwidth}
		\includegraphics[width=\textwidth,page=5]{{figures/mach3/2018/Selection/2018_RedNDmatrix_rebin_verbose_may_noweights_ND280_nom}}
	\end{subfigure}
	\begin{subfigure}[t]{0.32\textwidth}
		\includegraphics[width=\textwidth,page=6]{{figures/mach3/2018/Selection/2018_RedNDmatrix_rebin_verbose_may_noweights_ND280_nom}}
	\end{subfigure}
	
	\begin{subfigure}[t]{0.32\textwidth}
		\includegraphics[width=\textwidth,page=7]{{figures/mach3/2018/Selection/2018_RedNDmatrix_rebin_verbose_may_noweights_ND280_nom}}
	\end{subfigure}
	\begin{subfigure}[t]{0.32\textwidth}
		\includegraphics[width=\textwidth,page=8]{{figures/mach3/2018/Selection/2018_RedNDmatrix_rebin_verbose_may_noweights_ND280_nom}}
	\end{subfigure}
	\begin{subfigure}[t]{0.32\textwidth}
		\includegraphics[width=\textwidth,page=9]{{figures/mach3/2018/Selection/2018_RedNDmatrix_rebin_verbose_may_noweights_ND280_nom}}
	\end{subfigure}
	
	\caption{Data and nominal MC distributions and the Data/MC ratio for FGD1 FHC selections. Lines of constant $Q^2_\text{reco}$ are shown. Bin content is normalised to bin width.}
	\label{fig:nominal2D_FGD1numu_2018}
\end{figure}

\autoref{fig:nominal2D_FGD2numu_2018} shows the nominal FHC \numu distributions for FGD2, with very similar behaviour to the FGD1 distributions.
\begin{figure}[h]
	\begin{subfigure}[t]{0.32\textwidth}
		\includegraphics[width=\textwidth,page=10]{{figures/mach3/2018/Selection/2018_RedNDmatrix_rebin_verbose_may_noweights_ND280_nom}}
	\end{subfigure}
	\begin{subfigure}[t]{0.32\textwidth}
		\includegraphics[width=\textwidth,page=11]{{figures/mach3/2018/Selection/2018_RedNDmatrix_rebin_verbose_may_noweights_ND280_nom}}
	\end{subfigure}
	\begin{subfigure}[t]{0.32\textwidth}
		\includegraphics[width=\textwidth,page=12]{{figures/mach3/2018/Selection/2018_RedNDmatrix_rebin_verbose_may_noweights_ND280_nom}}
	\end{subfigure}
	
	\begin{subfigure}[t]{0.32\textwidth}
		\includegraphics[width=\textwidth,page=13]{{figures/mach3/2018/Selection/2018_RedNDmatrix_rebin_verbose_may_noweights_ND280_nom}}
	\end{subfigure}
	\begin{subfigure}[t]{0.32\textwidth}
		\includegraphics[width=\textwidth,page=14]{{figures/mach3/2018/Selection/2018_RedNDmatrix_rebin_verbose_may_noweights_ND280_nom}}
	\end{subfigure}
	\begin{subfigure}[t]{0.32\textwidth}
		\includegraphics[width=\textwidth,page=15]{{figures/mach3/2018/Selection/2018_RedNDmatrix_rebin_verbose_may_noweights_ND280_nom}}
	\end{subfigure}
	
	\begin{subfigure}[t]{0.32\textwidth}
		\includegraphics[width=\textwidth,page=16]{{figures/mach3/2018/Selection/2018_RedNDmatrix_rebin_verbose_may_noweights_ND280_nom}}
	\end{subfigure}
	\begin{subfigure}[t]{0.32\textwidth}
		\includegraphics[width=\textwidth,page=17]{{figures/mach3/2018/Selection/2018_RedNDmatrix_rebin_verbose_may_noweights_ND280_nom}}
	\end{subfigure}
	\begin{subfigure}[t]{0.32\textwidth}
		\includegraphics[width=\textwidth,page=18]{{figures/mach3/2018/Selection/2018_RedNDmatrix_rebin_verbose_may_noweights_ND280_nom}}
	\end{subfigure}
	
	\caption{Data and nominal MC distributions and the Data/MC ratio for FGD2 FHC selections. Lines of constant $Q^2_\text{reco}$ are shown. Bin content is normalised to bin width.}
	\label{fig:nominal2D_FGD2numu_2018}
\end{figure}

\autoref{fig:nominal2D_FGD1numubar_2018} shows the first light for the new RHC CC0$\pi$, CC1$\pi$ and CCNTrack selections. The CC0$\pi$ selection is slightly over-estimated, but the nominal prediction looks more compatible with data than the FHC \numu distributions. The Data/MC ratio also doesn't appear to contain the same deficiency in $Q^2$. The CC1$\pi$ distribution is consistently underestimated in the most forward bin, and hints at an overestimation at low $Q^2$. The CCOther distribution looks similar to the FHC equivalents in that it is almost consistently underestimated.
\begin{figure}
	\begin{subfigure}[t]{0.32\textwidth}
		\includegraphics[width=\textwidth,page=19]{{figures/mach3/2018/Selection/2018_RedNDmatrix_rebin_verbose_may_noweights_ND280_nom}}
	\end{subfigure}
	\begin{subfigure}[t]{0.32\textwidth}
		\includegraphics[width=\textwidth,page=20]{{figures/mach3/2018/Selection/2018_RedNDmatrix_rebin_verbose_may_noweights_ND280_nom}}
	\end{subfigure}
	\begin{subfigure}[t]{0.32\textwidth}
		\includegraphics[width=\textwidth,page=21]{{figures/mach3/2018/Selection/2018_RedNDmatrix_rebin_verbose_may_noweights_ND280_nom}}
	\end{subfigure}
	
	\begin{subfigure}[t]{0.32\textwidth}
		\includegraphics[width=\textwidth,page=22]{{figures/mach3/2018/Selection/2018_RedNDmatrix_rebin_verbose_may_noweights_ND280_nom}}
	\end{subfigure}
	\begin{subfigure}[t]{0.32\textwidth}
		\includegraphics[width=\textwidth,page=23]{{figures/mach3/2018/Selection/2018_RedNDmatrix_rebin_verbose_may_noweights_ND280_nom}}
	\end{subfigure}
	\begin{subfigure}[t]{0.32\textwidth}
		\includegraphics[width=\textwidth,page=24]{{figures/mach3/2018/Selection/2018_RedNDmatrix_rebin_verbose_may_noweights_ND280_nom}}
	\end{subfigure}
	
	\begin{subfigure}[t]{0.32\textwidth}
		\includegraphics[width=\textwidth,page=25]{{figures/mach3/2018/Selection/2018_RedNDmatrix_rebin_verbose_may_noweights_ND280_nom}}
	\end{subfigure}
	\begin{subfigure}[t]{0.32\textwidth}
		\includegraphics[width=\textwidth,page=26]{{figures/mach3/2018/Selection/2018_RedNDmatrix_rebin_verbose_may_noweights_ND280_nom}}
	\end{subfigure}
	\begin{subfigure}[t]{0.32\textwidth}
		\includegraphics[width=\textwidth,page=27]{{figures/mach3/2018/Selection/2018_RedNDmatrix_rebin_verbose_may_noweights_ND280_nom}}
	\end{subfigure}
	\caption{Data and nominal MC distributions and the Data/MC ratio for FGD1 \numubar selections. Lines of constant $Q^2_\text{reco}$ are shown. Bin content is normalised to bin width.}
	\label{fig:nominal2D_FGD1numubar_2018}
\end{figure}

\autoref{fig:nominal2D_FGD2numubar_2018} shows the new RHC selections for FGD2. The CC0$\pi$ distribution is underestimated, in contrast to FGD1, and looks more similar to the FHC selections with patters of underestimation looking roughly constant in $Q^2$. The CC1$\pi$ distribution appears to underestimate in $Q^2$ rather than overestimate as was the case for FGD1. The CCOther distribution however largely looks compatible with the FGD1 case and is underestimated in general.
\begin{figure}
	\begin{subfigure}[t]{0.32\textwidth}
		\includegraphics[width=\textwidth,page=28]{{figures/mach3/2018/Selection/2018_RedNDmatrix_rebin_verbose_may_noweights_ND280_nom}}
	\end{subfigure}
	\begin{subfigure}[t]{0.32\textwidth}
		\includegraphics[width=\textwidth,page=29]{{figures/mach3/2018/Selection/2018_RedNDmatrix_rebin_verbose_may_noweights_ND280_nom}}
	\end{subfigure}
	\begin{subfigure}[t]{0.32\textwidth}
		\includegraphics[width=\textwidth,page=30]{{figures/mach3/2018/Selection/2018_RedNDmatrix_rebin_verbose_may_noweights_ND280_nom}}
	\end{subfigure}

	\begin{subfigure}[t]{0.32\textwidth}
		\includegraphics[width=\textwidth,page=31]{{figures/mach3/2018/Selection/2018_RedNDmatrix_rebin_verbose_may_noweights_ND280_nom}}
	\end{subfigure}
	\begin{subfigure}[t]{0.32\textwidth}
		\includegraphics[width=\textwidth,page=32]{{figures/mach3/2018/Selection/2018_RedNDmatrix_rebin_verbose_may_noweights_ND280_nom}}
	\end{subfigure}
	\begin{subfigure}[t]{0.32\textwidth}
		\includegraphics[width=\textwidth,page=33]{{figures/mach3/2018/Selection/2018_RedNDmatrix_rebin_verbose_may_noweights_ND280_nom}}
	\end{subfigure}
	
	\begin{subfigure}[t]{0.32\textwidth}
		\includegraphics[width=\textwidth,page=34]{{figures/mach3/2018/Selection/2018_RedNDmatrix_rebin_verbose_may_noweights_ND280_nom}}
	\end{subfigure}
	\begin{subfigure}[t]{0.32\textwidth}
		\includegraphics[width=\textwidth,page=35]{{figures/mach3/2018/Selection/2018_RedNDmatrix_rebin_verbose_may_noweights_ND280_nom}}
	\end{subfigure}
	\begin{subfigure}[t]{0.32\textwidth}
		\includegraphics[width=\textwidth,page=36]{{figures/mach3/2018/Selection/2018_RedNDmatrix_rebin_verbose_may_noweights_ND280_nom}}
	\end{subfigure}	
\caption{Data and nominal MC distributions and the Data/MC ratio for FGD2 \numubar selections. Lines of constant $Q^2_\text{reco}$ are shown. Bin content is normalised to bin width.}
\label{fig:nominal2D_FGD2numubar_2018}
\end{figure}

\autoref{fig:nominal2D_FGD1numurhc_2018} shows the new RHC \numu selections for FGD1. As expected, they are concentrated at higher $p_\mu$, owing to the neutrino parents producing neutrinos of higher $E_\nu$. The distributions are consistently underestimated, although the shapes are fairly well reproduced. The CC0$\pi$ distribution again looks underestimated in constant $Q^2$, similar to the FHC \numu and RHC \numubar selections. The CC1$\pi$ selection is also similar to the RHC \numubar equivalent and is underestimated at high \cosmu and high \pmu. The CCOther selection agrees well with the previous CCOther selections, being even more underestimated than for the others.
\begin{figure}[h]
	\begin{subfigure}[t]{0.32\textwidth}
		\includegraphics[width=\textwidth,page=37]{{figures/mach3/2018/Selection/2018_RedNDmatrix_rebin_verbose_may_noweights_ND280_nom}}
	\end{subfigure}
	\begin{subfigure}[t]{0.32\textwidth}
		\includegraphics[width=\textwidth,page=38]{{figures/mach3/2018/Selection/2018_RedNDmatrix_rebin_verbose_may_noweights_ND280_nom}}
	\end{subfigure}
	\begin{subfigure}[t]{0.32\textwidth}
		\includegraphics[width=\textwidth,page=39]{{figures/mach3/2018/Selection/2018_RedNDmatrix_rebin_verbose_may_noweights_ND280_nom}}
	\end{subfigure}

	\begin{subfigure}[t]{0.32\textwidth}
		\includegraphics[width=\textwidth,page=40]{{figures/mach3/2018/Selection/2018_RedNDmatrix_rebin_verbose_may_noweights_ND280_nom}}
	\end{subfigure}
	\begin{subfigure}[t]{0.32\textwidth}
		\includegraphics[width=\textwidth,page=41]{{figures/mach3/2018/Selection/2018_RedNDmatrix_rebin_verbose_may_noweights_ND280_nom}}
	\end{subfigure}
	\begin{subfigure}[t]{0.32\textwidth}
		\includegraphics[width=\textwidth,page=42]{{figures/mach3/2018/Selection/2018_RedNDmatrix_rebin_verbose_may_noweights_ND280_nom}}
	\end{subfigure}

	\begin{subfigure}[t]{0.32\textwidth}
		\includegraphics[width=\textwidth,page=43]{{figures/mach3/2018/Selection/2018_RedNDmatrix_rebin_verbose_may_noweights_ND280_nom}}
	\end{subfigure}
	\begin{subfigure}[t]{0.32\textwidth}
		\includegraphics[width=\textwidth,page=44]{{figures/mach3/2018/Selection/2018_RedNDmatrix_rebin_verbose_may_noweights_ND280_nom}}
	\end{subfigure}
	\begin{subfigure}[t]{0.32\textwidth}
		\includegraphics[width=\textwidth,page=45]{{figures/mach3/2018/Selection/2018_RedNDmatrix_rebin_verbose_may_noweights_ND280_nom}}
	\end{subfigure}
\caption{Data and nominal MC distributions and the Data/MC ratio for FGD1 \numu RHC selections. Lines of constant $Q^2_\text{reco}$ are shown. Bin content is normalised to bin width.}
\label{fig:nominal2D_FGD1numurhc_2018}
\end{figure}

The FGD2 \numu RHC distributions in \autoref{fig:nominal2D_FGD2numurhc_2018} are similarly underestimated throughout. For CC0$\pi$ there is consistent underestimation at high \pmu and \cosmu, which agrees with the FGD1 distribution. The data appears shifted towards higher momentum to the prediction, and have a larger spread. The CC1$\pi$ selection is similarly patchy and is difficult to draw conclusions from. CCOther is consistent with FGD1, being mostly underestimated, although overestimated at low momentum.
\begin{figure}[h]
	\begin{subfigure}[t]{0.32\textwidth}
		\includegraphics[width=\textwidth,page=46]{{figures/mach3/2018/Selection/2018_RedNDmatrix_rebin_verbose_may_noweights_ND280_nom}}
	\end{subfigure}
	\begin{subfigure}[t]{0.32\textwidth}
		\includegraphics[width=\textwidth,page=47]{{figures/mach3/2018/Selection/2018_RedNDmatrix_rebin_verbose_may_noweights_ND280_nom}}
	\end{subfigure}
	\begin{subfigure}[t]{0.32\textwidth}
		\includegraphics[width=\textwidth,page=48]{{figures/mach3/2018/Selection/2018_RedNDmatrix_rebin_verbose_may_noweights_ND280_nom}}
	\end{subfigure}
	
	\begin{subfigure}[t]{0.32\textwidth}
		\includegraphics[width=\textwidth,page=49]{{figures/mach3/2018/Selection/2018_RedNDmatrix_rebin_verbose_may_noweights_ND280_nom}}
	\end{subfigure}
	\begin{subfigure}[t]{0.32\textwidth}
		\includegraphics[width=\textwidth,page=50]{{figures/mach3/2018/Selection/2018_RedNDmatrix_rebin_verbose_may_noweights_ND280_nom}}
	\end{subfigure}
	\begin{subfigure}[t]{0.32\textwidth}
		\includegraphics[width=\textwidth,page=51]{{figures/mach3/2018/Selection/2018_RedNDmatrix_rebin_verbose_may_noweights_ND280_nom}}
	\end{subfigure}
	
	\begin{subfigure}[t]{0.32\textwidth}
		\includegraphics[width=\textwidth,page=52]{{figures/mach3/2018/Selection/2018_RedNDmatrix_rebin_verbose_may_noweights_ND280_nom}}
	\end{subfigure}
	\begin{subfigure}[t]{0.32\textwidth}
		\includegraphics[width=\textwidth,page=53]{{figures/mach3/2018/Selection/2018_RedNDmatrix_rebin_verbose_may_noweights_ND280_nom}}
	\end{subfigure}
	\begin{subfigure}[t]{0.32\textwidth}
		\includegraphics[width=\textwidth,page=54]{{figures/mach3/2018/Selection/2018_RedNDmatrix_rebin_verbose_may_noweights_ND280_nom}}
	\end{subfigure}
	\caption{Data and nominal MC distributions and the Data/MC ratio for FGD2 \numu RHC selections. Lines of constant $Q^2_\text{reco}$ are shown. Bin content is normalised to bin width.}
	\label{fig:nominal2D_FGD2numurhc_2018}
\end{figure}

Looking closer at the nominal distributions, we project onto \pmu and \cosmu separately and look at the mode contributions in the nominal model. The distributions for FGD1 FHC \numu selections are shown in \autoref{fig:nominal1D_pmu_fhc_2018}. For CC0$\pi$ we see the same behaviour as in 2017 (\autoref{fig:nominal1D_pmu}), with the low momentum on underestimated up until the event peak at 500 MeV, which after 1 GeV is mostly well modelled. The CC1$\pi$ distributions look marginally more consistent with data compared to 2017, but there is still a consistent overestimation in $0.5 < p_\mu < 1\text{ GeV}$, with a slight underestimation at the event peak. The CCOther distributions are again grossly underestimated throughout $p_\mu$ between 10 and 20\%. FGD2 looks particularly like a normalisation effect.
\begin{figure}[h]
	\begin{subfigure}[t]{0.20\textwidth}
		\includegraphics[width=\textwidth,page=55]{{figures/mach3/2018/Selection/2018_RedNDmatrix_rebin_verbose_may_noweights_ND280_nom}}
	\end{subfigure}

	\begin{subfigure}[t]{0.32\textwidth}
		\includegraphics[width=\textwidth,page=56]{{figures/mach3/2018/Selection/2018_RedNDmatrix_rebin_verbose_may_noweights_ND280_nom}}
	\end{subfigure}
	\begin{subfigure}[t]{0.32\textwidth}
		\includegraphics[width=\textwidth,page=58]{{figures/mach3/2018/Selection/2018_RedNDmatrix_rebin_verbose_may_noweights_ND280_nom}}
	\end{subfigure}
	\begin{subfigure}[t]{0.32\textwidth}
		\includegraphics[width=\textwidth,page=60]{{figures/mach3/2018/Selection/2018_RedNDmatrix_rebin_verbose_may_noweights_ND280_nom}}
	\end{subfigure}
	
	\begin{subfigure}[t]{0.32\textwidth}
		\includegraphics[width=\textwidth,page=62]{{figures/mach3/2018/Selection/2018_RedNDmatrix_rebin_verbose_may_noweights_ND280_nom}}
	\end{subfigure}
	\begin{subfigure}[t]{0.32\textwidth}
		\includegraphics[width=\textwidth,page=64]{{figures/mach3/2018/Selection/2018_RedNDmatrix_rebin_verbose_may_noweights_ND280_nom}}
	\end{subfigure}
	\begin{subfigure}[t]{0.32\textwidth}
		\includegraphics[width=\textwidth,page=66]{{figures/mach3/2018/Selection/2018_RedNDmatrix_rebin_verbose_may_noweights_ND280_nom}}
	\end{subfigure}

\caption{Data and nominal MC distributions for FHC \numu selections projected onto \pmu, showing contributions by interaction mode. Bin content is normalised to bin width.}
\label{fig:nominal1D_pmu_fhc_2018}
\end{figure}

The new RHC \numubar distributions' projections are shown in \autoref{fig:nominal1D_pmu_rhc_numubar_2018}, where the 0$\pi$ selection shows a similar pattern to the FHC \numu projections above and the 1Trk distributions from 2017. Namely, there is an underestimation at low $p_\mu$, which goes to an overestimation at $\sim0.5\text{ GeV}$, which returns to a satisfactory prediction above 1 GeV. FGD1 appears to see this more than FGD2, as was the case in 2017. For the 1$\pi$ selections, the prediction is mostly within 1$\sigma$ of the data (excluding systematic errors on the Monte Carlo). For the CCOther distributions there are hints of a consistent underestimation for both FGDs---particularly at lower momentum---but the majority of the prediction is still within statistical error of the data.
\begin{figure}[h]
	
	\begin{subfigure}[t]{0.32\textwidth}
		\includegraphics[width=\textwidth,page=68]{{figures/mach3/2018/Selection/2018_RedNDmatrix_rebin_verbose_may_noweights_ND280_nom}}
	\end{subfigure}
	\begin{subfigure}[t]{0.32\textwidth}
		\includegraphics[width=\textwidth,page=70]{{figures/mach3/2018/Selection/2018_RedNDmatrix_rebin_verbose_may_noweights_ND280_nom}}
	\end{subfigure}
	\begin{subfigure}[t]{0.32\textwidth}
		\includegraphics[width=\textwidth,page=72]{{figures/mach3/2018/Selection/2018_RedNDmatrix_rebin_verbose_may_noweights_ND280_nom}}
	\end{subfigure}

	\begin{subfigure}[t]{0.32\textwidth}
		\includegraphics[width=\textwidth,page=74]{{figures/mach3/2018/Selection/2018_RedNDmatrix_rebin_verbose_may_noweights_ND280_nom}}
	\end{subfigure}
	\begin{subfigure}[t]{0.32\textwidth}
		\includegraphics[width=\textwidth,page=76]{{figures/mach3/2018/Selection/2018_RedNDmatrix_rebin_verbose_may_noweights_ND280_nom}}
	\end{subfigure}
	\begin{subfigure}[t]{0.32\textwidth}
		\includegraphics[width=\textwidth,page=78]{{figures/mach3/2018/Selection/2018_RedNDmatrix_rebin_verbose_may_noweights_ND280_nom}}
	\end{subfigure}
\caption{Data and nominal MC distributions for RHC \numubar selections projected onto \pmu, showing contributions by interaction mode. Bin content is normalised to bin width.}
\label{fig:nominal1D_pmu_rhc_numubar_2018}
\end{figure}

Turning to the \pmu projections of the RHC \numu distributions in \autoref{fig:nominal1D_pmu_rhc_numu_2018}, we see a consistent underestimation of the FGD1 CC0$\pi$ selection, with FGD2 overestimation the lowest momentum bin and two adjacent bins at 700 MeV. The CC1$\pi$ distributions are compatible across the two FGDs, where the low momentum bin is slightly overestimated and the two next bins underestimated, which isn't compatible with the \numubar 1$\pi$ selections. However, the CCQE and 2p2h contributions are high at low momentum for the \numu selection, and barely present for the \numubar selection. The CCOther selection---which we noted contains more \numubar CCOther than \numu CCOther---verifies the consistent picture of grossly underestimating the data, constant with \pmu at 10-25\%.
\begin{figure}
\begin{subfigure}[t]{0.32\textwidth}
	\includegraphics[width=\textwidth,page=80]{{figures/mach3/2018/Selection/2018_RedNDmatrix_rebin_verbose_may_noweights_ND280_nom}}
\end{subfigure}
\begin{subfigure}[t]{0.32\textwidth}
	\includegraphics[width=\textwidth,page=82]{{figures/mach3/2018/Selection/2018_RedNDmatrix_rebin_verbose_may_noweights_ND280_nom}}
\end{subfigure}
\begin{subfigure}[t]{0.32\textwidth}
	\includegraphics[width=\textwidth,page=84]{{figures/mach3/2018/Selection/2018_RedNDmatrix_rebin_verbose_may_noweights_ND280_nom}}
\end{subfigure}

\begin{subfigure}[t]{0.32\textwidth}
	\includegraphics[width=\textwidth,page=86]{{figures/mach3/2018/Selection/2018_RedNDmatrix_rebin_verbose_may_noweights_ND280_nom}}
\end{subfigure}
\begin{subfigure}[t]{0.32\textwidth}
	\includegraphics[width=\textwidth,page=88]{{figures/mach3/2018/Selection/2018_RedNDmatrix_rebin_verbose_may_noweights_ND280_nom}}
\end{subfigure}
\begin{subfigure}[t]{0.32\textwidth}
	\includegraphics[width=\textwidth,page=90]{{figures/mach3/2018/Selection/2018_RedNDmatrix_rebin_verbose_may_noweights_ND280_nom}}
\end{subfigure}
\caption{Data and nominal MC distributions for RHC \numu selections projected onto \pmu, showing contributions by interaction mode. Bin content is normalised to bin width.}
\label{fig:nominal1D_pmu_rhc_numu_2018}
\end{figure}

Turning attention to the \cosmu projections of the FHC \numu selections in \autoref{fig:nominal1D_cosmu_fhc_2018}, the additional data and bins compared to 2017 highlights the underestimation of much of the high-angle data: all the way until $\cos\theta_\mu\sim0.85$ for both FGDs. The overestimation is almost present in every bin above that and outside the statistical error of the data. The small overestimation from 2017 when $0.8 < \cos\theta_\mu < 0.93$ appears gone. The 1$\pi$ selections repeats the wavy Data/MC pattern of 2017 in both FGDs, with overestimates at high-angles and underestimates the most forward going angles. As was the case for the \pmu projections, the CCOther \cosmu distributions are grossly underestimated throughout \cosmu, between 10-30\%.
\begin{figure}[h]
	\begin{subfigure}[t]{0.32\textwidth}
		\includegraphics[width=\textwidth,page=57]{{figures/mach3/2018/Selection/2018_RedNDmatrix_rebin_verbose_may_noweights_ND280_nom}}
	\end{subfigure}
	\begin{subfigure}[t]{0.32\textwidth}
		\includegraphics[width=\textwidth,page=59]{{figures/mach3/2018/Selection/2018_RedNDmatrix_rebin_verbose_may_noweights_ND280_nom}}
	\end{subfigure}
	\begin{subfigure}[t]{0.32\textwidth}
		\includegraphics[width=\textwidth,page=61]{{figures/mach3/2018/Selection/2018_RedNDmatrix_rebin_verbose_may_noweights_ND280_nom}}
	\end{subfigure}
	
	\begin{subfigure}[t]{0.32\textwidth}
		\includegraphics[width=\textwidth,page=63]{{figures/mach3/2018/Selection/2018_RedNDmatrix_rebin_verbose_may_noweights_ND280_nom}}
	\end{subfigure}
	\begin{subfigure}[t]{0.32\textwidth}
		\includegraphics[width=\textwidth,page=65]{{figures/mach3/2018/Selection/2018_RedNDmatrix_rebin_verbose_may_noweights_ND280_nom}}
	\end{subfigure}
	\begin{subfigure}[t]{0.32\textwidth}
		\includegraphics[width=\textwidth,page=67]{{figures/mach3/2018/Selection/2018_RedNDmatrix_rebin_verbose_may_noweights_ND280_nom}}
	\end{subfigure}
	
	\caption{Data and nominal MC distributions for FHC \numu selections projected onto \cosmu, showing contributions by interaction mode. Bin content is normalised to bin width.}
	\label{fig:nominal1D_cosmu_fhc_2018}
\end{figure}

The RHC \numubar selection's \cosmu projections are shown in \autoref{fig:nominal1D_cosmu_rhc_numubar_2018}, where the CC0$\pi$ echoes the \numu equivalent: underestimation at high angles up to \cosmu=0.85, above which FGD1 oscillates between over and under estimate, and FGD2 consistently underestimates at 10\%, just within the statistical error of the data. As for the \pmu projection, the 1$\pi$ selections are mostly within statistical error except for the most forward-going bin, which sees an underestimate for FGD1 and an overestimate for FGD2. The CCOther distributions are both almost consistently underestimated with the exception of three bins in total, although each of the three are within statistical error of the data. 
\begin{figure}[h]
	\begin{subfigure}[t]{0.32\textwidth}
		\includegraphics[width=\textwidth,page=69]{{figures/mach3/2018/Selection/2018_RedNDmatrix_rebin_verbose_may_noweights_ND280_nom}}
	\end{subfigure}
	\begin{subfigure}[t]{0.32\textwidth}
		\includegraphics[width=\textwidth,page=71]{{figures/mach3/2018/Selection/2018_RedNDmatrix_rebin_verbose_may_noweights_ND280_nom}}
	\end{subfigure}
	\begin{subfigure}[t]{0.32\textwidth}
		\includegraphics[width=\textwidth,page=73]{{figures/mach3/2018/Selection/2018_RedNDmatrix_rebin_verbose_may_noweights_ND280_nom}}
	\end{subfigure}
	
	\begin{subfigure}[t]{0.32\textwidth}
		\includegraphics[width=\textwidth,page=75]{{figures/mach3/2018/Selection/2018_RedNDmatrix_rebin_verbose_may_noweights_ND280_nom}}
	\end{subfigure}
	\begin{subfigure}[t]{0.32\textwidth}
		\includegraphics[width=\textwidth,page=77]{{figures/mach3/2018/Selection/2018_RedNDmatrix_rebin_verbose_may_noweights_ND280_nom}}
	\end{subfigure}
	\begin{subfigure}[t]{0.32\textwidth}
		\includegraphics[width=\textwidth,page=79]{{figures/mach3/2018/Selection/2018_RedNDmatrix_rebin_verbose_may_noweights_ND280_nom}}
	\end{subfigure}
	\caption{Data and nominal MC distributions for RHC \numubar selections projected onto \cosmu, showing contributions by interaction mode. Bin content is normalised to bin width.}
	\label{fig:nominal1D_cosmu_rhc_numubar_2018}
\end{figure}

Finally the RHC \numu \cosmu projections are shown in \autoref{fig:nominal1D_cosmu_rhc_numu_2018}, where we note a large 1$\pi$ and multi-$\pi$ contribution to the 0$\pi$ selection, especially in the forward-going region. The data is underestimated above \cosmu=0.9, similar to what was seen in the RHC \numubar and FHC \numu distributions, compatible with 2017. The 1$\pi$ selection is harder to draw conclusions from, although both FGDs are mostly consistent in underestimating the most forward-going bins and overestimating the high-angle and backwards. The CCOther distributions are again consistent with the other CCOther selections: constant underestimation of the data between 10-30\%.
\begin{figure}
	\begin{subfigure}[t]{0.32\textwidth}
		\includegraphics[width=\textwidth,page=81]{{figures/mach3/2018/Selection/2018_RedNDmatrix_rebin_verbose_may_noweights_ND280_nom}}
	\end{subfigure}
	\begin{subfigure}[t]{0.32\textwidth}
		\includegraphics[width=\textwidth,page=83]{{figures/mach3/2018/Selection/2018_RedNDmatrix_rebin_verbose_may_noweights_ND280_nom}}
	\end{subfigure}
	\begin{subfigure}[t]{0.32\textwidth}
		\includegraphics[width=\textwidth,page=85]{{figures/mach3/2018/Selection/2018_RedNDmatrix_rebin_verbose_may_noweights_ND280_nom}}
	\end{subfigure}
	
	\begin{subfigure}[t]{0.32\textwidth}
		\includegraphics[width=\textwidth,page=87]{{figures/mach3/2018/Selection/2018_RedNDmatrix_rebin_verbose_may_noweights_ND280_nom}}
	\end{subfigure}
	\begin{subfigure}[t]{0.32\textwidth}
		\includegraphics[width=\textwidth,page=89]{{figures/mach3/2018/Selection/2018_RedNDmatrix_rebin_verbose_may_noweights_ND280_nom}}
	\end{subfigure}
	\begin{subfigure}[t]{0.32\textwidth}
		\includegraphics[width=\textwidth,page=91]{{figures/mach3/2018/Selection/2018_RedNDmatrix_rebin_verbose_may_noweights_ND280_nom}}
	\end{subfigure}
	\caption{Data and nominal MC distributions for RHC \numu selections projected onto \cosmu, showing contributions by interaction mode. Bin content is normalised to bin width.}
	\label{fig:nominal1D_cosmu_rhc_numu_2018}
\end{figure}

Summarising the mode contributions for each selection, we look at \autoref{tab:nominal_mode_afterscale_2018}. The mode contributions are very similar to the 2017 results for FHC \numu selections. The RHC \numubar 0$\pi$ selection has 3\% less CCQE events, likely due to the change in the muon likelihood cut, and has a larger contamination of 1$\pi$, multi-$\pi$ and DIS events. The NC contribution is largest for the new RHC \numubar CCOther selections at 13.5\%. Generally, the selections perform satisfactorily at targeting the appropriate interaction modes, reaching above 55\% throughout.
\begin{table}[h]
	\centering
	\begin{tabular}{l | c c c c c c c }
		\hline
		\hline
		Sample	      & CCQE & 2p2h & CC1$\pi^{\pm,0}$ 	& CC coh 	& CC multi-$\pi$ & CC DIS  	& NC \\
		\hline
                FGD1 $0\pi$     & \textbf{56.7} & \textbf{9.9} & 19.5 & 0.3 & 4.6 & 5.2 & 3.7 \\
                FGD2 0$\pi$     & \textbf{54.8} & \textbf{9.3} & 21.6 & 0.3 & 5.0 & 5.3 & 3.7 \\
		\hline
                FGD1 $1\pi$     & 6.2 & 1.0 & \textbf{48.1} & \textbf{2.8} & \textbf{17.7} & 17.4 & 6.8 \\
                FGD2 1$\pi$     & 5.7 & 0.8 & \textbf{47.7} & \textbf{2.8} & \textbf{18.1} & 18.1 & 6.8 \\
		\hline
                FGD1 Other      & 5.0 & 1.0 & 14.9 & 0.4 & \textbf{25.6} & \textbf{44.8} & 8.3 \\
                FGD2 Other      & 5.1 & 1.0 & 15.4 & 0.4 & \textbf{25.1} & \textbf{45.0} & 8.0 \\
		\hline
                FGD1 \numubar $0\pi$  & \textbf{61.5} & \textbf{10.1} & 15.6 & 0.8 & 3.3 & 3.2 & 5.5 \\
                FGD2 \numubar 0$\pi$  & \textbf{61.7} & \textbf{9.8} & 15.8 & 0.7 & 3.6 & 3.3 & 5.1 \\
		\hline
                FGD1 \numubar $1\pi$  & 3.8 & 0.8 & \textbf{44.4} & \textbf{8.7} & \textbf{15.3} & 18.7 & 8.3 \\
                FGD2 \numubar 1$\pi$  & 5.5 & 0.6 & \textbf{42.2} & \textbf{7.8} & \textbf{15.7} & 18.9 & 9.3 \\
		\hline
                FGD1 \numubar Other   &9.4 & 1.5 & 19.8 & 1.4 & \textbf{22.4} & \textbf{32.0} & 13.5 \\
                FGD2 \numubar Other   &8.7 & 1.5 & 19.4 & 1.4 & \textbf{22.3} & \textbf{33.4} & 13.3 \\
                \hline

                FGD1 \numu RHC $0\pi$  & \textbf{42.3} & \textbf{8.7} & 24.1 & 0.7 & 7.8 & 8.3 & 8.1 \\
                FGD2 \numu RHC 0$\pi$  & \textbf{40.2} & \textbf{8.2} & 26.2 & 0.7 & 8.8 & 8.7 & 7.2 \\
		\hline
                FGD1 \numu RHC $1\pi$  & 6.7 & 1.1 & \textbf{45.1} & \textbf{3.8} & \textbf{20.3} & 15.7 & 7.3 \\
                FGD2 \numu RHC 1$\pi$  &6.4 & 0.8 & \textbf{44.7} & \textbf{3.9} & \textbf{21.5} & 16.8 & 5.9 \\
		\hline
                FGD1 \numu RHC Other   &4.9 & 1.0 & 14.2 & 0.8 & \textbf{26.3} & \textbf{44.7} & 8.1 \\
                FGD2 \numu RHC Other   &4.6 & 0.9 & 15.2 & 0.6 & \textbf{25.9} & \textbf{44.5} & 8.3 \\

		\hline
		\hline
	\end{tabular}
	\caption{Percentage mode breakdown for the binned nominal \textbf{scaled} Monte-Carlo samples, \textbf{boldface} indicates interactions targeted by specific selections. Directly comparable to 2017 results in \autoref{tab:nominal_mode_afterscale}.}
	\label{tab:nominal_mode_afterscale_2018}
\end{table}

\autoref{tab:detailed_eventrate_2018} shows the effect on the overall event rate in applying the different classes of weights.
\begin{sidewaystable}
  \resizebox{\textwidth}{!}{%
    \begin{tabular}{ l | c | c | c | c | c | c | c }
    \hline
    \hline
      Sample  & MC  & POT  & Flux  & Xsec & Det & ND280 Cov  & All \\
      \hline
      FGD1 0$\pi$ & 470176 & 31464.6 & 34153.7 & 30106.3 & 30298 & 31487.9  & 31529.3\\
      FGD1 1$\pi$ & 119835 & 8059.92 & 9106.99 & 7496.3 & 7690.07 & 7959.59  & 7998.1 \\
      FGD1 Other & 92630 & 6224.1 & 7389.73 & 6096.03 & 5917.31 & 6144.95  & 6793.68 \\
      FGD2 0$\pi$ & 471140 & 31215.4 & 33881.4 & 30017.8 & 30361.4 & 31198.7  & 31734 \\
      FGD2 1$\pi$ & 95498 & 6303.05 & 7148.91 & 5919.55 & 6121.38 & 6184.34  & 6419.04 \\
      FGD2 Other & 87931 & 5839.17 & 6933.53 & 5723.31 & 5697.16 & 5773.49  & 6562.75 \\

      FGD1 \numubar 0$\pi$ & 96602 & 6784.14 & 6983.08 & 6247.83 & 6723.77 & 6794.13 & 6371.34 \\
      FGD1 \numubar 1$\pi$ & 9133 & 639.595 & 658.426 & 536.293 & 623.687 & 635.372 & 533.253 \\
      FGD1 \numubar Other & 15046 & 1066.91 & 1113.12 & 1012.08 & 1044.37 & 1055.12 & 1023.36 \\

      FGD2 \numubar 0$\pi$ & 95597 & 6692.82 & 6897.34 & 6185.55 & 6578.7 & 6715.93 & 6283.35 \\
      FGD2 \numubar 1$\pi$ & 8165 & 568.917 & 587.274 & 491.61 & 553.376 & 557.22 & 483.508 \\
      FGD2 \numubar Other & 13849 & 970.796 & 1015.05 & 927.283 & 954.657 & 961.134 & 943.956 \\

      FGD1 \numu RHC 0$\pi$ & 34950 & 2457.88 & 2646.97 & 2392.37 & 2379.95 & 2449.2 & 2485.51 \\
      FGD1 \numu RHC 1$\pi$ & 12352 & 871.792 & 952.669 & 814.909 & 838.532 & 871.141 & 855.911 \\
      FGD1 \numu RHC Other & 10894 & 764.374 & 854.448 & 750.775 & 733.571 & 764.374 & 804.647 \\

      FGD2 \numu RHC 0$\pi$ & 35180 & 2458.01 & 2645.96 & 2408.41 & 2419.07 & 2458.01 & 2553.51 \\
      FGD2 \numu RHC 1$\pi$ & 9714 & 676.737 & 740.521 & 632.743 & 662.563 & 676.737 & 679.99 \\
      FGD2 \numu RHC Other & 10421 & 732.441 & 819.628 & 718.565 & 720.726 & 732.441 & 792.166 \\
      \hline
      Total & 1689110 & 113791 & 124529 & 108478 & 110318 & 113420 & 114847 \\
      \hline
      \hline
    \end{tabular}
        }
        \caption{Event rates broken by type of weight applied}
  \label{tab:detailed_eventrate_2018}
\end{sidewaystable}

\section{Asimov}
\label{sec:asimov_2018}
In this section we use the nominal model predictions outlined above and set it to the data to perform closure tests and expected sensitivity studies, as was done in \autoref{sec:asimov} for the 2017 analysis.

\subsection{Likelihood scan}
For the likelihood scans we set the data to be the nominal model prediction and vary each parameter one at a time, resetting it to the nominal when one parameter scan is complete. It follows the identical method to in the 2017 analysis, but includes the new multi-$\pi$ RHC selections and run 7 and 8 data. Hence we expect relatively large increases in sensitivity for many parameter, primarily those driven by the sample likelihood contribution rather than the prior.

\autoref{fig:beam_asimov_llh_2018} shows the same beam parameters as \autoref{fig:beam_asimov_llh} with the new selections and data. We note a significantly stronger constraint on the sample likelihoods, with no change to the prior, as expected. The ND280 FHC \numu 0.6-0.7 GeV parameter moves from a sample $-2\ln\mathcal{L}$ contribution of $\sim175$ to $\sim350$, and we note a Gaussian response for all the flux parameters.
\begin{figure}[h]
	\centering
	\begin{subfigure}[t]{0.32\textwidth}
		\includegraphics[width=\textwidth, trim={0mm 0mm 0mm 11mm}, clip,page=5]{figures/mach3/2018/llh/tryBinningNumber6_after_fit_asimov_asimov_ND280logL_scan}
		\caption{ND280 FHC \numu 0.6-0.7 GeV}
	\end{subfigure}
	\begin{subfigure}[t]{0.32\textwidth}
		\includegraphics[width=\textwidth, trim={0mm 0mm 0mm 11mm}, clip,page=13]{figures/mach3/2018/llh/tryBinningNumber6_after_fit_asimov_asimov_ND280logL_scan}
		\caption{ND280 FHC \numubar 0.7-1.0 GeV}
	\end{subfigure}
	\begin{subfigure}[t]{0.32\textwidth}
		\includegraphics[width=\textwidth, trim={0mm 0mm 0mm 11mm}, clip,page=30]{figures/mach3/2018/llh/tryBinningNumber6_after_fit_asimov_asimov_ND280logL_scan}
		\caption{ND280 RHC \numubar 0.5-0.6 GeV}
	\end{subfigure}
	
	\begin{subfigure}[t]{0.32\textwidth}
		\includegraphics[width=\textwidth, trim={0mm 0mm 0mm 11mm}, clip,page=18]{figures/mach3/2018/llh/tryBinningNumber6_after_fit_asimov_asimov_ND280logL_scan}
		\caption{ND280 FHC \nue 0.5-0.7 GeV}
	\end{subfigure}
	\begin{subfigure}[t]{0.32\textwidth}
		\includegraphics[width=\textwidth, trim={0mm 0mm 0mm 11mm}, clip,page=45]{figures/mach3/2018/llh/tryBinningNumber6_after_fit_asimov_asimov_ND280logL_scan}
		\caption{ND280 RHC \nuebar 0.5-0.7 GeV}
	\end{subfigure}
	\begin{subfigure}[t]{0.32\textwidth}
		\includegraphics[width=\textwidth, trim={0mm 0mm 0mm 11mm}, clip,page=55]{figures/mach3/2018/llh/tryBinningNumber6_after_fit_asimov_asimov_ND280logL_scan}
		\caption{SK FHC \numu 0.6-0.7 GeV}
	\end{subfigure}
	\caption{Asimov likelihood scans for selected beam parameters}
	\label{fig:beam_asimov_llh_2018}
\end{figure}

\autoref{fig:xsec_asimov_llh_2018} shows the likelihood scans for the same interaction parameters as \autoref{fig:xsec_asimov_llh}, where we again note large increases in the likelihood responses for many parameters. Notably $M_A^{QE}$ almost doubles in sensitivity, as does 2p2h shape and the non-resonant $I_{1/2}$ single pion parameter.
\begin{figure}[h]
	\centering
	\begin{subfigure}[t]{0.32\textwidth}
		\includegraphics[width=\textwidth,page=105, trim={0mm 0mm 0mm 9mm}, clip]{figures/mach3/2018/llh/tryBinningNumber6_after_fit_asimov_asimov_ND280logL_scan}
		\caption{$M_A^{QE}$}
	\end{subfigure}
	\begin{subfigure}[t]{0.32\textwidth}
		\includegraphics[width=\textwidth,page=108, trim={0mm 0mm 0mm 9mm}, clip]{figures/mach3/2018/llh/tryBinningNumber6_after_fit_asimov_asimov_ND280logL_scan}
		\caption{2p2h norm $\nu$}
	\end{subfigure}
	\begin{subfigure}[t]{0.32\textwidth}
		\includegraphics[width=\textwidth,page=111, trim={0mm 0mm 0mm 9mm}, clip]{figures/mach3/2018/llh/tryBinningNumber6_after_fit_asimov_asimov_ND280logL_scan}
		\caption{2p2h shape C}
	\end{subfigure}

	\begin{subfigure}[t]{0.32\textwidth}
		\includegraphics[width=\textwidth,page=116, trim={0mm 0mm 0mm 9mm}, clip]{figures/mach3/2018/llh/tryBinningNumber6_after_fit_asimov_asimov_ND280logL_scan}
		\caption{BeRPA E}
	\end{subfigure}
	\begin{subfigure}[t]{0.32\textwidth}
		\includegraphics[width=\textwidth,page=120, trim={0mm 0mm 0mm 9mm}, clip]{figures/mach3/2018/llh/tryBinningNumber6_after_fit_asimov_asimov_ND280logL_scan}
		\caption{$I_{1/2}^\text{bkg}$}
	\end{subfigure}
	\begin{subfigure}[t]{0.32\textwidth}
		\includegraphics[width=\textwidth,page=104, trim={0mm 0mm 0mm 9mm}, clip]{figures/mach3/2018/llh/tryBinningNumber6_after_fit_asimov_asimov_ND280logL_scan}
		\caption{FSI CEX LO}
	\end{subfigure}
	\caption{Asimov likelihood scans for selected cross-section parameters}
	\label{fig:xsec_asimov_llh_2018}
\end{figure}

To facilitate direct comparisons between 2017 and 2018 analyses, the likelihood response for some of the parameters with the largest improvement are compared in \autoref{fig:llh_2017_vs_2018}. We see that all the listed interaction parameters increase by a factor $1.5\sim2.0$, whereas the flux parameters don't improve more than 25\%.
\begin{figure}[h]
	\centering
	\begin{subfigure}[t]{0.32\textwidth}
		\includegraphics[width=\textwidth,page=1, trim={0mm 0mm 0mm 9mm}, clip]{figures/mach3/2018/llh/MultiPi_vs_MultiTrack_TotalLLH_2017vs2018}
		\caption{ND280 FHC \numu 0.7-1.0 GeV}
	\end{subfigure}
	\begin{subfigure}[t]{0.32\textwidth}
		\includegraphics[width=\textwidth,page=10, trim={0mm 0mm 0mm 9mm}, clip]{figures/mach3/2018/llh/MultiPi_vs_MultiTrack_TotalLLH_2017vs2018}
		\caption{$M_A^{QE}$}
	\end{subfigure}
	\begin{subfigure}[t]{0.32\textwidth}
		\includegraphics[width=\textwidth,page=16, trim={0mm 0mm 0mm 9mm}, clip]{figures/mach3/2018/llh/MultiPi_vs_MultiTrack_TotalLLH_2017vs2018}
		\caption{2p2h shape C}
	\end{subfigure}
	
	\begin{subfigure}[t]{0.32\textwidth}
		\includegraphics[width=\textwidth,page=18, trim={0mm 0mm 0mm 9mm}, clip]{figures/mach3/2018/llh/MultiPi_vs_MultiTrack_TotalLLH_2017vs2018}
		\caption{BeRPA A}
	\end{subfigure}
	\begin{subfigure}[t]{0.32\textwidth}
		\includegraphics[width=\textwidth,page=23, trim={0mm 0mm 0mm 9mm}, clip]{figures/mach3/2018/llh/MultiPi_vs_MultiTrack_TotalLLH_2017vs2018}
		\caption{$C_5^A$}
	\end{subfigure}
	\begin{subfigure}[t]{0.32\textwidth}
		\includegraphics[width=\textwidth,page=26, trim={0mm 0mm 0mm 9mm}, clip]{figures/mach3/2018/llh/MultiPi_vs_MultiTrack_TotalLLH_2017vs2018}
		\caption{CC DIS}
	\end{subfigure}
	\caption{Asimov likelihood scans for 2017 and 2018 analyses}
	\label{fig:llh_2017_vs_2018}
\end{figure}

Finally we compare the effect coming from the new data alone versus the new data with rebinning and new RHC multi-$\pi$ in \autoref{fig:llh_multitrack_vs_multipi}. The flux parameters all varied less than 5\%, whereas many interaction parameters see large changes. 

Most of the parameter change by approximately only a normalisation whereas the pion final state parameters and 2p2h shape parameters change shape. A clear example of this is the pion final-state-interaction parameters in \autoref{fig:llh_multitrack_vs_multipi_fsi}. They all see the largest increase in response, often with a complex shape. This is somewhat expected since the RHC selections in 2017 only had one selection per FGD with reconstructed pions (NTrack) but had two per FGD for the FHC selections (1$\pi$ and Other). Many of the FSI parameters get stronger constraints from using both $\pi^+$ and $\pi^-$ data, which is better satisfied with the new RHC selections.
\begin{figure}[h]
	\centering
	\begin{subfigure}[t]{0.32\textwidth}
		\includegraphics[width=\textwidth,page=1, trim={0mm 0mm 0mm 9mm}, clip]{figures/mach3/2018/llh/MultiPi_vs_MultiTrack_TotalLLH}
		\caption{$M_A^{QE}$}
	\end{subfigure}
	\begin{subfigure}[t]{0.32\textwidth}
		\includegraphics[width=\textwidth,page=4, trim={0mm 0mm 0mm 9mm}, clip]{figures/mach3/2018/llh/MultiPi_vs_MultiTrack_TotalLLH}
		\caption{2p2h shape C}
	\end{subfigure}
	\begin{subfigure}[t]{0.32\textwidth}
		\includegraphics[width=\textwidth,page=6, trim={0mm 0mm 0mm 9mm}, clip]{figures/mach3/2018/llh/MultiPi_vs_MultiTrack_TotalLLH}
		\caption{BeRPA D}
	\end{subfigure}
\caption{Asimov likelihood scans for multi-$\pi$ and rebinned samples versus the unchanged multi-track sample from 2017 with run 2 to 8 statistics}
\label{fig:llh_multitrack_vs_multipi}
\end{figure}

\begin{figure}[h]
	\centering
	\begin{subfigure}[t]{0.32\textwidth}
		\includegraphics[width=\textwidth,page=7, trim={0mm 0mm 0mm 9mm}, clip]{figures/mach3/2018/llh/MultiPi_vs_MultiTrack_TotalLLH}
		\caption{Quasi-elastic}
	\end{subfigure}
	\begin{subfigure}[t]{0.32\textwidth}
		\includegraphics[width=\textwidth,page=9, trim={0mm 0mm 0mm 9mm}, clip]{figures/mach3/2018/llh/MultiPi_vs_MultiTrack_TotalLLH}
		\caption{Inelastic}
	\end{subfigure}
	\begin{subfigure}[t]{0.32\textwidth}
		\includegraphics[width=\textwidth,page=11, trim={0mm 0mm 0mm 9mm}, clip]{figures/mach3/2018/llh/MultiPi_vs_MultiTrack_TotalLLH}
		\caption{Charge exchange}
	\end{subfigure}
	\caption{Asimov likelihood scans for multi-$\pi$ and rebinned samples versus the unchanged multi-track sample from 2017 with run 2 to 8 statistics, for some pion FSI rescaterring parameters}
	\label{fig:llh_multitrack_vs_multipi_fsi}
\end{figure}

\subsection{Prior predictive}
As was done in \autoref{sec:Asimov_prior} we perform a closure test checking the prior's predictive power on the Asimov data set. We expect the predictive spectrum to largely agree with the Asimov data and produce large uncertainties and thereby p-values around 0.5.

The \pmu \cosmu prior predictive spectrum is made using 20,000 correlated throws of the systematics, and the two-dimensional p-value is calculated for each of the 20,000 throws. The p-value compares the test-statistic from the samples of the drawn parameter variation with the Asimov data to the test-statistic of the drawn parameter variation to a statistical fluctuation of the drawn parameter variation. The fraction of draws in which the former is smaller than the latter constitute the p-value.

\autoref{fig:prior_predictive_asimov_2018} shows the resulting two-dimensional p-value, agreeing with the expectation. Comparing to 2017 where the p-value was 0.572, it's clear the p-value increases this year. In the case of skewed correlated parameter throws---coming from missing parameter correlations---we'd expect something similar to this when the case where we increase the statistics. This is because the x-axis in \autoref{fig:prior_predictive_asimov_2018} is likely to be squeeze when the statistics increase, whereas they y-axis would not because the systematics are unchanged changed.
\begin{figure}[h]
	\begin{subfigure}[t]{0.49\textwidth}
		\includegraphics[width=\textwidth, trim={0mm 0mm 0mm 11mm}, clip,page=1]{figures/mach3/2018/asimov/pred/17May_MultiPi_CovFix_Final_pospred_PriorPred_procs.pdf}
	\end{subfigure}
	\begin{subfigure}[t]{0.49\textwidth}
		\includegraphics[width=\textwidth, trim={0mm 0mm 0mm 11mm}, clip,page=2]{figures/mach3/2018/asimov/pred/17May_MultiPi_CovFix_Final_pospred_PriorPred_procs.pdf}
	\end{subfigure}
	\caption{Prior predictive p-values for the Asimov data in 2018}
	\label{fig:prior_predictive_asimov_2018}
\end{figure}

\autoref{tab:asimov_prior_pred_2018} shows the event rates after making correlated throws of the systematics with the test-statistic to Asimov data. Comparing to the 2017 equivalent in \autoref{tab:asimov_prior_pred} we see a similar level of uncertainty, as expected. As in 2017, the prior predictive on Asimov data produces an 11\% uncertainty on the total event rate, with 13\% on the CC0$\pi$, 11\% on CC1$\pi$ and 13\% on CCOther. As in 2017, we see a consistent skew in the prior predictive ``best-fit'' where it overestimates the Asimov data set, likely due to missing covariance matrices between cross-section, flux and ND280 parameters. The contribution to the test-statistic is much larger too compared to 2017, primarily due to many more bins in each selection.
\begin{table}
	\begin{tabular} {l | c c | c}
		\hline
		\hline
		Sample 			& Nominal	& Prior Pred.	& $-2\log\mathcal{L}_S$ \\
		\hline
		FGD1 0$\pi$             & 31529.3   & $32347.8\pm4135.7$  & 26.1   \\
		FGD1 1$\pi$             & 7998.1    & $8106.6\pm8899.4$   & 2.1   \\
		FGD1 Other              & 6793.7   & $6894.1\pm858.4$   & 3.1 \\
		\hline
		FGD2 0$\pi$             & 31734.1  & $32572.8\pm4012.8$  & 26.5   \\
		FGD2 1$\pi$             & 6419.0   & $6511.2\pm702.4$   & 2.0    \\
		FGD2 Other              & 6562.8   & $6652.8\pm786.4$    & 3.0   \\
		\hline
		FGD1 \numubar 0$\pi$    & 6371.3   & $6541.9\pm847.5$   & 7.7   \\
		FGD1 \numubar 1$\pi$    & 533.3   	& $530.5\pm115.5$    & 4.6  \\
		FGD1 \numubar Other     & 1023.4   & $1037.4\pm191.1$    & 0.8   \\
		\hline
		FGD2 \numubar 0$\pi$    & 6283.4   & $6444.7\pm830.5$   & 8.4   \\
		FGD2 \numubar 1$\pi$    & 483.5   	& $486.8\pm102.3$    & 0.6  \\
		FGD2 \numubar Other     & 944.0   	& $953.2\pm204.1$    & 0.6   \\
		\hline
		FGD1 \numu RHC 0$\pi$   & 2485.5   & $2543.8\pm429.5$   & 3.3   \\
		FGD1 \numu RHC 1$\pi$   & 855.9   	& $861.9\pm107.7$     & 0.5  \\
		FGD1 \numu RHC Other    & 804.7   	& $790.9\pm159.8$     & 1.9  \\
		\hline
		FGD2 \numu RHC 0$\pi$   & 2553.5   & $2503.5\pm395.5$   & 6.3  \\
		FGD2 \numu RHC 1$\pi$   & 680.0    & $687.3\pm88.1$     & 0.8  \\
		FGD2 \numu RHC Other    & 792.2   	& $780.4\pm136.5$     & 1.9  \\
		\hline
		Total                   & 114834    & $117541.9\pm12383.0$  & 100.2 \\
		% Updated
		\hline
		\hline
	\end{tabular}
	\caption{Prior predictive event rates for the Asimov data}
	\label{tab:asimov_prior_pred_2018}
\end{table}

\subsection{Asimov fit}
\label{sec:asimov_fit_2018}
We now fit the model to the Asimov data defined as the model at nominal to estimate the sensitivity and as a closure test. We also perform some test-fits using different ND280 covariance matrices, comparing results using the nominal binned multi-track to the binned multi-$\pi$ RHC selections, and fitting without any detector parameters.

The MCMC parameter obtained for the different studies are shown in \autoref{tab:mcmc_asimov_2018}. The primary reason behind the low acceptance for the ``Full cov'' (full ND280 covariance matrix used) is the 4368 parameters being fit simultaneously.
\begin{table}[h]
	\begin{tabular}{l | c c c}
		\hline
		\hline
		Name		&	Step length & Acceptance & Accepted steps \\
		\hline
		Nominal cov	& 	3,900,000	& 12.1\%	 & 471,900 \\
		Full cov	& 	1,367,502	& 5.8\%		 & 79,315 \\
		Multi-track & 	3,000,000	& 10.8\%	 & 324,272 \\
		No det		& 	3,900,000	& 24.3\%	 & 947,700 \\
		\hline
		\hline
	\end{tabular}
	\caption{Markov Chain parameters for the various Asimov fits in \autoref{sec:asimov_fit_2018}}
	\label{tab:mcmc_asimov_2018}
\end{table}

\subsubsection{Full and reduced ND280 cov}
Simultaneous with checking the performance of the parameterisation, we compare a fit to Asimov data using the reduced ND280 covariance matrix with 1076 ND280 detector parameters, and one using the full ND280 covariance matrix without any bin merging with 4238 ND280 detector parameters.

\autoref{fig:asimov_fit_2018_full_red_beam_fhc} and \autoref{fig:asimov_fit_2018_full_red_beam_rhc} show the flux parameters after the fit to Asimov data using the two matrices. There is a consistent bias in all flux parameters for both ND280 and SK, which appear to be mostly a normalisation-only offset by 1-2\%. The ND280 and SK follow the same pattern throughout, and the uncertainties are approximately halved compared to the prior, similar to the 2017 fit to Asimov data.

The full and reduced ND280 covariance matrices agree and are compatible in all parameters and only small shifts are seen, primarily in regions of very low statistics. 
\begin{figure}[h]
	\centering
	\begin{subfigure}[t]{0.10\textwidth}
		\includegraphics[width=\textwidth,page=1, trim={0mm 0mm 0mm 9mm}, clip]{figures/mach3/2018/asimov/2018a_MultiPi_Asimov_NoRed_TryAgain_merg_2018a_MultiPi_Binningv6_NewCov_Asimov_merge}
	\end{subfigure}

\begin{subfigure}[t]{\textwidth}
	\begin{subfigure}[t]{0.24\textwidth}
		\includegraphics[width=\textwidth,page=2, trim={0mm 0mm 0mm 9mm}, clip]{figures/mach3/2018/asimov/2018a_MultiPi_Asimov_NoRed_TryAgain_merg_2018a_MultiPi_Binningv6_NewCov_Asimov_merge}
	\end{subfigure}
	\begin{subfigure}[t]{0.24\textwidth}
		\includegraphics[width=\textwidth,page=3, trim={0mm 0mm 0mm 9mm}, clip]{figures/mach3/2018/asimov/2018a_MultiPi_Asimov_NoRed_TryAgain_merg_2018a_MultiPi_Binningv6_NewCov_Asimov_merge}
	\end{subfigure}
	\begin{subfigure}[t]{0.24\textwidth}
		\includegraphics[width=\textwidth,page=4, trim={0mm 0mm 0mm 9mm}, clip]{figures/mach3/2018/asimov/2018a_MultiPi_Asimov_NoRed_TryAgain_merg_2018a_MultiPi_Binningv6_NewCov_Asimov_merge}
	\end{subfigure}
	\begin{subfigure}[t]{0.24\textwidth}
		\includegraphics[width=\textwidth,page=5, trim={0mm 0mm 0mm 9mm}, clip]{figures/mach3/2018/asimov/2018a_MultiPi_Asimov_NoRed_TryAgain_merg_2018a_MultiPi_Binningv6_NewCov_Asimov_merge}
	\end{subfigure}
\caption{ND280}
\end{subfigure}

\begin{subfigure}[t]{\textwidth}
\begin{subfigure}[t]{0.24\textwidth}
	\includegraphics[width=\textwidth,page=10, trim={0mm 0mm 0mm 9mm}, clip]{figures/mach3/2018/asimov/2018a_MultiPi_Asimov_NoRed_TryAgain_merg_2018a_MultiPi_Binningv6_NewCov_Asimov_merge}
\end{subfigure}
\begin{subfigure}[t]{0.24\textwidth}
	\includegraphics[width=\textwidth,page=11, trim={0mm 0mm 0mm 9mm}, clip]{figures/mach3/2018/asimov/2018a_MultiPi_Asimov_NoRed_TryAgain_merg_2018a_MultiPi_Binningv6_NewCov_Asimov_merge}
\end{subfigure}
\begin{subfigure}[t]{0.24\textwidth}
	\includegraphics[width=\textwidth,page=12, trim={0mm 0mm 0mm 9mm}, clip]{figures/mach3/2018/asimov/2018a_MultiPi_Asimov_NoRed_TryAgain_merg_2018a_MultiPi_Binningv6_NewCov_Asimov_merge}
\end{subfigure}
\begin{subfigure}[t]{0.24\textwidth}
	\includegraphics[width=\textwidth,page=13, trim={0mm 0mm 0mm 9mm}, clip]{figures/mach3/2018/asimov/2018a_MultiPi_Asimov_NoRed_TryAgain_merg_2018a_MultiPi_Binningv6_NewCov_Asimov_merge}
\end{subfigure}
\caption{SK}
\end{subfigure}
	\caption{FHC flux parameters, comparing Asimov fits with full and reduced ND280 covariance matrices}
	\label{fig:asimov_fit_2018_full_red_beam_fhc}
\end{figure}

\begin{figure}[h]
	\centering
	\begin{subfigure}[t]{\textwidth}
\begin{subfigure}[t]{0.24\textwidth}
	\includegraphics[width=\textwidth,page=6, trim={0mm 0mm 0mm 9mm}, clip]{figures/mach3/2018/asimov/2018a_MultiPi_Asimov_NoRed_TryAgain_merg_2018a_MultiPi_Binningv6_NewCov_Asimov_merge}
\end{subfigure}
\begin{subfigure}[t]{0.24\textwidth}
	\includegraphics[width=\textwidth,page=7, trim={0mm 0mm 0mm 9mm}, clip]{figures/mach3/2018/asimov/2018a_MultiPi_Asimov_NoRed_TryAgain_merg_2018a_MultiPi_Binningv6_NewCov_Asimov_merge}
\end{subfigure}
\begin{subfigure}[t]{0.24\textwidth}
	\includegraphics[width=\textwidth,page=8, trim={0mm 0mm 0mm 9mm}, clip]{figures/mach3/2018/asimov/2018a_MultiPi_Asimov_NoRed_TryAgain_merg_2018a_MultiPi_Binningv6_NewCov_Asimov_merge}
\end{subfigure}
\begin{subfigure}[t]{0.24\textwidth}
	\includegraphics[width=\textwidth,page=9, trim={0mm 0mm 0mm 9mm}, clip]{figures/mach3/2018/asimov/2018a_MultiPi_Asimov_NoRed_TryAgain_merg_2018a_MultiPi_Binningv6_NewCov_Asimov_merge}
\end{subfigure}
\caption{ND280}
\end{subfigure}

\begin{subfigure}[t]{\textwidth}
	\begin{subfigure}[t]{0.24\textwidth}
		\includegraphics[width=\textwidth,page=14, trim={0mm 0mm 0mm 9mm}, clip]{figures/mach3/2018/asimov/2018a_MultiPi_Asimov_NoRed_TryAgain_merg_2018a_MultiPi_Binningv6_NewCov_Asimov_merge}
	\end{subfigure}
	\begin{subfigure}[t]{0.24\textwidth}
		\includegraphics[width=\textwidth,page=15, trim={0mm 0mm 0mm 9mm}, clip]{figures/mach3/2018/asimov/2018a_MultiPi_Asimov_NoRed_TryAgain_merg_2018a_MultiPi_Binningv6_NewCov_Asimov_merge}
	\end{subfigure}
	\begin{subfigure}[t]{0.24\textwidth}
		\includegraphics[width=\textwidth,page=16, trim={0mm 0mm 0mm 9mm}, clip]{figures/mach3/2018/asimov/2018a_MultiPi_Asimov_NoRed_TryAgain_merg_2018a_MultiPi_Binningv6_NewCov_Asimov_merge}
	\end{subfigure}
	\begin{subfigure}[t]{0.24\textwidth}
		\includegraphics[width=\textwidth,page=17, trim={0mm 0mm 0mm 9mm}, clip]{figures/mach3/2018/asimov/2018a_MultiPi_Asimov_NoRed_TryAgain_merg_2018a_MultiPi_Binningv6_NewCov_Asimov_merge}
	\end{subfigure}
\caption{SK}
\end{subfigure}
	\caption{RHC flux parameters, comparing Asimov fits with full and reduced ND280 covariance matrices}
	\label{fig:asimov_fit_2018_full_red_beam_rhc}
\end{figure}

Looking at the cross-section parameters after the fit to Asimov data in \autoref{fig:asimov_fit_2018_full_red_xsec}, the pattern is much the same as in 2017. $p_F$, 2p2h normalisation $\bar{\nu}$, 2p2h shape, BeRPA A and BeRPA E appear biased from the Asimov parameter value, whereas the rest seem to find it satisfactorily. 

The reduction in parameter uncertainties from the doubling of data is particularly noticeable for 2p2h normalisation $\nu$, which is reduced to 10\% from 20\%, $M_A^{QE}$ which now has a similar uncertainty to fits from bubble chamber data, the 2p2h shape parameters which now have 20\% uncertainty, and the BeRPA B parameter, whose uncertainty more than halves compared to the prior. We also see small uncertainties on the single pion parameters, and the CCDIS parameter (which in 2017 appeared biased in Asimov) is not biased. The ND280 fit also brings down the uncertainties on the pion rescattering probabilities, by between 1/2 to 1/4 of the prior.

The two ND280 covariance matrices are compatible for the interaction parameters.
\begin{figure}[h]
	\centering
	\begin{subfigure}[t]{0.49\textwidth}
		\includegraphics[width=\textwidth,page=18, trim={0mm 0mm 0mm 9mm}, clip]{figures/mach3/2018/asimov/2018a_MultiPi_Asimov_NoRed_TryAgain_merg_2018a_MultiPi_Binningv6_NewCov_Asimov_merge}
	\end{subfigure}
	\begin{subfigure}[t]{0.49\textwidth}
		\includegraphics[width=\textwidth,page=19, trim={0mm 0mm 0mm 9mm}, clip]{figures/mach3/2018/asimov/2018a_MultiPi_Asimov_NoRed_TryAgain_merg_2018a_MultiPi_Binningv6_NewCov_Asimov_merge}
	\end{subfigure}

	\begin{subfigure}[t]{0.49\textwidth}
		\includegraphics[width=\textwidth,page=20, trim={0mm 0mm 0mm 9mm}, clip]{figures/mach3/2018/asimov/2018a_MultiPi_Asimov_NoRed_TryAgain_merg_2018a_MultiPi_Binningv6_NewCov_Asimov_merge}
	\end{subfigure}
	\begin{subfigure}[t]{0.49\textwidth}
		\includegraphics[width=\textwidth,page=21, trim={0mm 0mm 0mm 9mm}, clip]{figures/mach3/2018/asimov/2018a_MultiPi_Asimov_NoRed_TryAgain_merg_2018a_MultiPi_Binningv6_NewCov_Asimov_merge}
	\end{subfigure}
	\caption{Interaction parameters, comparing Asimov fits with Full and reduced ND280 covariance matrices}
	\label{fig:asimov_fit_2018_full_red_xsec}
\end{figure}

\subsubsection{Comparison to multi-track}
As with the likelihood scan, we now compare results from using the rebinned old selections (FHC \numu) and the new rebinned selections (RHC \numubar, \numu) for the 2018 analysis versus the selection and binning of 2017 including the new run 7 and 8 data.

In the light of the biases in the flux parameters from using the multi-$\pi$ samples above and the biases observed in the 2017 analysis' Asimov study (\autoref{sec:asimov_fit}), it is particularly interesting to see the flux parameters largely unbiased for the multi-track selection in \autoref{fig:asimov_fit_2018_mpi_mtrack_fhc} and \autoref{fig:asimov_fit_2018_mpi_mtrack_rhc}. The two selections follow a similar pattern, although the fit using the multi-$\pi$ is offset by 1-2\%.

The ND280 and SK parameters echo each other again and are compatible.

Comparing the size of the errors from the two selections, we note a marginally smaller error for the multi-$\pi$ selection for the flux parameters, although barely discernible.
\begin{figure}[h]
	\centering
	\begin{subfigure}[t]{0.10\textwidth}
		\includegraphics[width=\textwidth,page=1, trim={0mm 0mm 0mm 9mm}, clip]{figures/mach3/2018/asimov/2018a_MultiPi_Binningv6_NewCov_Asimov_merg_2018a_NewDetMatrix_20180307_Asimov_again_merge}
	\end{subfigure}
	
	\begin{subfigure}[t]{\textwidth}
	\begin{subfigure}[t]{0.24\textwidth}
		\includegraphics[width=\textwidth,page=2, trim={0mm 0mm 0mm 9mm}, clip]{figures/mach3/2018/asimov/2018a_MultiPi_Binningv6_NewCov_Asimov_merg_2018a_NewDetMatrix_20180307_Asimov_again_merge}
	\end{subfigure}
	\begin{subfigure}[t]{0.24\textwidth}
		\includegraphics[width=\textwidth,page=3, trim={0mm 0mm 0mm 9mm}, clip]{figures/mach3/2018/asimov/2018a_MultiPi_Binningv6_NewCov_Asimov_merg_2018a_NewDetMatrix_20180307_Asimov_again_merge}
	\end{subfigure}
	\begin{subfigure}[t]{0.24\textwidth}
		\includegraphics[width=\textwidth,page=4, trim={0mm 0mm 0mm 9mm}, clip]{figures/mach3/2018/asimov/2018a_MultiPi_Binningv6_NewCov_Asimov_merg_2018a_NewDetMatrix_20180307_Asimov_again_merge}
	\end{subfigure}
	\begin{subfigure}[t]{0.24\textwidth}
		\includegraphics[width=\textwidth,page=5, trim={0mm 0mm 0mm 9mm}, clip]{figures/mach3/2018/asimov/2018a_MultiPi_Binningv6_NewCov_Asimov_merg_2018a_NewDetMatrix_20180307_Asimov_again_merge}
	\end{subfigure}
\caption{ND280}
\end{subfigure}

	\begin{subfigure}[t]{\textwidth}
\begin{subfigure}[t]{0.24\textwidth}
\includegraphics[width=\textwidth,page=10, trim={0mm 0mm 0mm 9mm}, clip]{figures/mach3/2018/asimov/2018a_MultiPi_Binningv6_NewCov_Asimov_merg_2018a_NewDetMatrix_20180307_Asimov_again_merge}
\end{subfigure}
\begin{subfigure}[t]{0.24\textwidth}
\includegraphics[width=\textwidth,page=11, trim={0mm 0mm 0mm 9mm}, clip]{figures/mach3/2018/asimov/2018a_MultiPi_Binningv6_NewCov_Asimov_merg_2018a_NewDetMatrix_20180307_Asimov_again_merge}
\end{subfigure}
\begin{subfigure}[t]{0.24\textwidth}
\includegraphics[width=\textwidth,page=12, trim={0mm 0mm 0mm 9mm}, clip]{figures/mach3/2018/asimov/2018a_MultiPi_Binningv6_NewCov_Asimov_merg_2018a_NewDetMatrix_20180307_Asimov_again_merge}
\end{subfigure}
\begin{subfigure}[t]{0.24\textwidth}
\includegraphics[width=\textwidth,page=13, trim={0mm 0mm 0mm 9mm}, clip]{figures/mach3/2018/asimov/2018a_MultiPi_Binningv6_NewCov_Asimov_merg_2018a_NewDetMatrix_20180307_Asimov_again_merge}
\end{subfigure}
\caption{SK}
\end{subfigure}
	\caption{FHC flux parameters, comparing Asimov fits with rebinned multi-$\pi$ to 2017 binned multi-track}
	\label{fig:asimov_fit_2018_mpi_mtrack_fhc}
\end{figure}

\begin{figure}[h]
	\centering
		\begin{subfigure}[t]{\textwidth}
	\begin{subfigure}[t]{0.24\textwidth}
		\includegraphics[width=\textwidth,page=6, trim={0mm 0mm 0mm 9mm}, clip]{figures/mach3/2018/asimov/2018a_MultiPi_Binningv6_NewCov_Asimov_merg_2018a_NewDetMatrix_20180307_Asimov_again_merge}
	\end{subfigure}
	\begin{subfigure}[t]{0.24\textwidth}
		\includegraphics[width=\textwidth,page=7, trim={0mm 0mm 0mm 9mm}, clip]{figures/mach3/2018/asimov/2018a_MultiPi_Binningv6_NewCov_Asimov_merg_2018a_NewDetMatrix_20180307_Asimov_again_merge}
	\end{subfigure}
	\begin{subfigure}[t]{0.24\textwidth}
		\includegraphics[width=\textwidth,page=8, trim={0mm 0mm 0mm 9mm}, clip]{figures/mach3/2018/asimov/2018a_MultiPi_Binningv6_NewCov_Asimov_merg_2018a_NewDetMatrix_20180307_Asimov_again_merge}
	\end{subfigure}
	\begin{subfigure}[t]{0.24\textwidth}
		\includegraphics[width=\textwidth,page=9, trim={0mm 0mm 0mm 9mm}, clip]{figures/mach3/2018/asimov/2018a_MultiPi_Binningv6_NewCov_Asimov_merg_2018a_NewDetMatrix_20180307_Asimov_again_merge}
	\end{subfigure}
\caption{ND280}
\end{subfigure}

	\begin{subfigure}[t]{\textwidth}
	\begin{subfigure}[t]{0.24\textwidth}
		\includegraphics[width=\textwidth,page=14, trim={0mm 0mm 0mm 9mm}, clip]{figures/mach3/2018/asimov/2018a_MultiPi_Binningv6_NewCov_Asimov_merg_2018a_NewDetMatrix_20180307_Asimov_again_merge}
	\end{subfigure}
	\begin{subfigure}[t]{0.24\textwidth}
		\includegraphics[width=\textwidth,page=15, trim={0mm 0mm 0mm 9mm}, clip]{figures/mach3/2018/asimov/2018a_MultiPi_Binningv6_NewCov_Asimov_merg_2018a_NewDetMatrix_20180307_Asimov_again_merge}
	\end{subfigure}
	\begin{subfigure}[t]{0.24\textwidth}
		\includegraphics[width=\textwidth,page=16, trim={0mm 0mm 0mm 9mm}, clip]{figures/mach3/2018/asimov/2018a_MultiPi_Binningv6_NewCov_Asimov_merg_2018a_NewDetMatrix_20180307_Asimov_again_merge}
	\end{subfigure}
	\begin{subfigure}[t]{0.24\textwidth}
		\includegraphics[width=\textwidth,page=17, trim={0mm 0mm 0mm 9mm}, clip]{figures/mach3/2018/asimov/2018a_MultiPi_Binningv6_NewCov_Asimov_merg_2018a_NewDetMatrix_20180307_Asimov_again_merge}
	\end{subfigure}
\caption{SK}
\end{subfigure}
	\caption{RHC flux parameters, comparing Asimov fits with rebinned multi-$\pi$ to 2017 binned multi-track}
	\label{fig:asimov_fit_2018_mpi_mtrack_rhc}
\end{figure}

The interaction parameters in \autoref{fig:asimov_fit_2018_mpi_mtrack_xsec} are entirely compatible and neither of the two fits show unexpected biases. 

Comparing the size of the errors on the parameters the multi-$\pi$ selection and rebinning has a much larger impact than on the flux parameters. Many parameters reduce by as much as 20-30\%, as expected from the earlier likelihood scans. The 2p2h and pion FSI parameters see the largest reductions.
\begin{figure}[h]
	\centering
	\begin{subfigure}[t]{0.49\textwidth}
		\includegraphics[width=\textwidth,page=18, trim={0mm 0mm 0mm 9mm}, clip]{figures/mach3/2018/asimov/2018a_MultiPi_Binningv6_NewCov_Asimov_merg_2018a_NewDetMatrix_20180307_Asimov_again_merge}
	\end{subfigure}
	\begin{subfigure}[t]{0.49\textwidth}
		\includegraphics[width=\textwidth,page=19, trim={0mm 0mm 0mm 9mm}, clip]{figures/mach3/2018/asimov/2018a_MultiPi_Binningv6_NewCov_Asimov_merg_2018a_NewDetMatrix_20180307_Asimov_again_merge}
	\end{subfigure}
	
	\begin{subfigure}[t]{0.49\textwidth}
		\includegraphics[width=\textwidth,page=20, trim={0mm 0mm 0mm 9mm}, clip]{figures/mach3/2018/asimov/2018a_MultiPi_Binningv6_NewCov_Asimov_merg_2018a_NewDetMatrix_20180307_Asimov_again_merge}
	\end{subfigure}
	\begin{subfigure}[t]{0.49\textwidth}
		\includegraphics[width=\textwidth,page=21, trim={0mm 0mm 0mm 9mm}, clip]{figures/mach3/2018/asimov/2018a_MultiPi_Binningv6_NewCov_Asimov_merg_2018a_NewDetMatrix_20180307_Asimov_again_merge}
	\end{subfigure}
	\caption{Interaction parameters, comparing Asimov fits with rebinned multi-$\pi$ to 2017 binned multi-track}
	\label{fig:asimov_fit_2018_mpi_mtrack_xsec}
\end{figure}

\subsubsection{Asimov without detector parameters}
To investigate the biases in the flux parameters seen in the fit to Asimov data for both the new ND280 covariance matrices for the multi-$\pi$ selections, we perform a fit to the asimov data without varying the ND280 parameters. This should help answer if there are issues related to dimensionality in the new fit, since we're fitting approximately double the number of parameters (1307 vs 781) with a large increase in number of bins (4130 vs 1624).

\autoref{fig:asimov_fit_2018_nodet_fhc} shows the FHC flux parameters without varying the ND280 systematics, showing the central values and errors evaluated with the arithmetic mean, a Gaussian fit and the highest-posterior-density. All of the biases seen in the previous Asimov fit are gone. The uncertainties are reduced due to the marginalisation over the ND280 systematics are essentially delta functions at the nominal values. Looking at the RHC parameters in \autoref{fig:asimov_fit_2018_nodet_rhc} the same holds true.
\begin{figure}[h]
	\centering
	\begin{subfigure}[t]{0.10\textwidth}
		\includegraphics[width=\textwidth,page=1, trim={0mm 0mm 0mm 9mm}, clip]{figures/mach3/2018/asimov/2018a_MultiPi_Binningv6_NewCov_Asimov_NoDet_merge_drawPar}
	\end{subfigure}
	
	\begin{subfigure}[t]{\textwidth}
	\begin{subfigure}[t]{0.24\textwidth}
		\includegraphics[width=\textwidth,page=2, trim={0mm 0mm 0mm 9mm}, clip]{figures/mach3/2018/asimov/2018a_MultiPi_Binningv6_NewCov_Asimov_NoDet_merge_drawPar}
	\end{subfigure}
	\begin{subfigure}[t]{0.24\textwidth}
		\includegraphics[width=\textwidth,page=3, trim={0mm 0mm 0mm 9mm}, clip]{figures/mach3/2018/asimov/2018a_MultiPi_Binningv6_NewCov_Asimov_NoDet_merge_drawPar}
	\end{subfigure}
	\begin{subfigure}[t]{0.24\textwidth}
		\includegraphics[width=\textwidth,page=4, trim={0mm 0mm 0mm 9mm}, clip]{figures/mach3/2018/asimov/2018a_MultiPi_Binningv6_NewCov_Asimov_NoDet_merge_drawPar}
	\end{subfigure}
	\begin{subfigure}[t]{0.24\textwidth}
		\includegraphics[width=\textwidth,page=5, trim={0mm 0mm 0mm 9mm}, clip]{figures/mach3/2018/asimov/2018a_MultiPi_Binningv6_NewCov_Asimov_NoDet_merge_drawPar}
	\end{subfigure}
\caption{ND280}
\end{subfigure}

	\begin{subfigure}[t]{\textwidth}
\begin{subfigure}[t]{0.24\textwidth}
	\includegraphics[width=\textwidth,page=10, trim={0mm 0mm 0mm 9mm}, clip]{figures/mach3/2018/asimov/2018a_MultiPi_Binningv6_NewCov_Asimov_NoDet_merge_drawPar}
\end{subfigure}
\begin{subfigure}[t]{0.24\textwidth}
	\includegraphics[width=\textwidth,page=11, trim={0mm 0mm 0mm 9mm}, clip]{figures/mach3/2018/asimov/2018a_MultiPi_Binningv6_NewCov_Asimov_NoDet_merge_drawPar}
\end{subfigure}
\begin{subfigure}[t]{0.24\textwidth}
	\includegraphics[width=\textwidth,page=12, trim={0mm 0mm 0mm 9mm}, clip]{figures/mach3/2018/asimov/2018a_MultiPi_Binningv6_NewCov_Asimov_NoDet_merge_drawPar}
\end{subfigure}
\begin{subfigure}[t]{0.24\textwidth}
	\includegraphics[width=\textwidth,page=13, trim={0mm 0mm 0mm 9mm}, clip]{figures/mach3/2018/asimov/2018a_MultiPi_Binningv6_NewCov_Asimov_NoDet_merge_drawPar}
\end{subfigure}

\caption{SK}
\end{subfigure}
	\caption{FHC flux parameters, fitting to Asimov without varying detector parameters}
	\label{fig:asimov_fit_2018_nodet_fhc}
\end{figure}

\begin{figure}[h]
	\centering
		\begin{subfigure}[t]{\textwidth}
	\begin{subfigure}[t]{0.24\textwidth}
		\includegraphics[width=\textwidth,page=6, trim={0mm 0mm 0mm 9mm}, clip]{figures/mach3/2018/asimov/2018a_MultiPi_Binningv6_NewCov_Asimov_NoDet_merge_drawPar}
	\end{subfigure}
	\begin{subfigure}[t]{0.24\textwidth}
		\includegraphics[width=\textwidth,page=7, trim={0mm 0mm 0mm 9mm}, clip]{figures/mach3/2018/asimov/2018a_MultiPi_Binningv6_NewCov_Asimov_NoDet_merge_drawPar}
	\end{subfigure}
	\begin{subfigure}[t]{0.24\textwidth}
		\includegraphics[width=\textwidth,page=8, trim={0mm 0mm 0mm 9mm}, clip]{figures/mach3/2018/asimov/2018a_MultiPi_Binningv6_NewCov_Asimov_NoDet_merge_drawPar}
	\end{subfigure}
	\begin{subfigure}[t]{0.24\textwidth}
		\includegraphics[width=\textwidth,page=9, trim={0mm 0mm 0mm 9mm}, clip]{figures/mach3/2018/asimov/2018a_MultiPi_Binningv6_NewCov_Asimov_NoDet_merge_drawPar}
	\end{subfigure}
\caption{ND280}
\end{subfigure}

	\begin{subfigure}[t]{\textwidth}
	\begin{subfigure}[t]{0.24\textwidth}
		\includegraphics[width=\textwidth,page=14, trim={0mm 0mm 0mm 9mm}, clip]{figures/mach3/2018/asimov/2018a_MultiPi_Binningv6_NewCov_Asimov_NoDet_merge_drawPar}
	\end{subfigure}
	\begin{subfigure}[t]{0.24\textwidth}
		\includegraphics[width=\textwidth,page=15, trim={0mm 0mm 0mm 9mm}, clip]{figures/mach3/2018/asimov/2018a_MultiPi_Binningv6_NewCov_Asimov_NoDet_merge_drawPar}
	\end{subfigure}
	\begin{subfigure}[t]{0.24\textwidth}
		\includegraphics[width=\textwidth,page=16, trim={0mm 0mm 0mm 9mm}, clip]{figures/mach3/2018/asimov/2018a_MultiPi_Binningv6_NewCov_Asimov_NoDet_merge_drawPar}
	\end{subfigure}
	\begin{subfigure}[t]{0.24\textwidth}
		\includegraphics[width=\textwidth,page=17, trim={0mm 0mm 0mm 9mm}, clip]{figures/mach3/2018/asimov/2018a_MultiPi_Binningv6_NewCov_Asimov_NoDet_merge_drawPar}
	\end{subfigure}
\caption{SK}
\end{subfigure}
	\caption{RHC flux parameters, fitting to Asimov without varying detector parameters}
	\label{fig:asimov_fit_2018_nodet_rhc}
\end{figure}

The interaction parameters in \autoref{fig:asimov_fit_2018_nodet_xsec} are similarly less biased than in the full fit, although the effect is less extreme than for the flux parameters.
\begin{figure}[h]
	\centering
	\begin{subfigure}[t]{0.49\textwidth}
		\includegraphics[width=\textwidth,page=18, trim={0mm 0mm 0mm 9mm}, clip]{figures/mach3/2018/asimov/2018a_MultiPi_Binningv6_NewCov_Asimov_NoDet_merge_drawPar}
	\end{subfigure}
	\begin{subfigure}[t]{0.49\textwidth}
		\includegraphics[width=\textwidth,page=19, trim={0mm 0mm 0mm 9mm}, clip]{figures/mach3/2018/asimov/2018a_MultiPi_Binningv6_NewCov_Asimov_NoDet_merge_drawPar}
	\end{subfigure}
	
	\begin{subfigure}[t]{0.49\textwidth}
		\includegraphics[width=\textwidth,page=20, trim={0mm 0mm 0mm 9mm}, clip]{figures/mach3/2018/asimov/2018a_MultiPi_Binningv6_NewCov_Asimov_NoDet_merge_drawPar}
	\end{subfigure}
	\begin{subfigure}[t]{0.49\textwidth}
		\includegraphics[width=\textwidth,page=21, trim={0mm 0mm 0mm 9mm}, clip]{figures/mach3/2018/asimov/2018a_MultiPi_Binningv6_NewCov_Asimov_NoDet_merge_drawPar}
	\end{subfigure}
	\caption{Interaction parameters, fitting to Asimov without varying detector parameters}
	\label{fig:asimov_fit_2018_nodet_xsec}
\end{figure}

In conclusion, it appears that the bias in the flux parameters come from the ND280 parameters. 
\subsection{Marginalisation effects}
\red{write this section}

\subsection{Covariance matrix}
Using the results from the fit to Asimov data we here look at the parameter correlations.

\autoref{fig:asimov_full_corr_2018} shows the full $\sqrt{}$ covariance and correlation matrix for the flux (bottom left corner) and cross-section (upper right corner) parameters. Many patterns are repeated from the 2017 case (\autoref{fig:asimov_full_corr}): the flux parameters are highly internally correlated where the cross-section parameters are mostly separated into categories of correlations (e.g. CC0$\pi$-CC0$\pi$ correlations are strong, CC0$\pi$-CC1$\pi$ are not).
\begin{figure}[h]
	\begin{subfigure}[t]{0.49\textwidth}
		\includegraphics[width=\textwidth, trim={0mm 0mm 0mm 0mm}, clip,page=2]{figures/mach3/2018/asimov/corr/2018a_MultiPi_Binningv6_NewCov_Asimov_merge_drawCorr}
		\caption{$\sqrt{\mathbf{V}_{i,j}}$}
	\end{subfigure}
	\begin{subfigure}[t]{0.49\textwidth}
		\includegraphics[width=\textwidth, trim={0mm 0mm 0mm 0mm}, clip,page=3]{figures/mach3/2018/asimov/corr/2018a_MultiPi_Binningv6_NewCov_Asimov_merge_drawCorr}
		\caption{$\rho_{i,j}$}
	\end{subfigure}
	\caption{$\sqrt{\mathbf{V}_{i,j}}$ and correlation matrix for the Asimov post-fit, showing the full flux and cross-section parameters}
	\label{fig:asimov_full_corr_2018}
\end{figure}

Looking at the more digestible version, excluding the SK flux parameters, in \autoref{fig:asimov_nd_corr_2018}, the strongest correlations between the flux and interaction parameters are for the CC DIS parameter and the $M_A^{QE}$, BeRPA A, BeRPA B and $C_5^A$ parameter in which the latter group are especially strong around the flux peak.

For the flux parameters, the largest uncertainties are seen for the high energy wrong-sign \nue parameters, since there barely is any data to constrain it. Furthermore, the production processes leading to such neutrinos are only weakly correlated with the lower energy right-sign processes, which is why it is weakly constrained by the prior covariance.

The largest interaction uncertainties are on 2p2h normalisations, BeRPA E, CC coherent normalisations and the NC parameters, which again is expected since the lack of data of such processes and $Q^2$, and it is compatible with the 2017 fit.
\begin{figure}[h]
	\begin{subfigure}[t]{0.49\textwidth}
		\includegraphics[width=\textwidth, trim={0mm 0mm 0mm 0mm}, clip,page=5]{figures/mach3/2018/asimov/corr/2018a_MultiPi_Binningv6_NewCov_Asimov_merge_drawCorr}
	\end{subfigure}
	\begin{subfigure}[t]{0.49\textwidth}
		\includegraphics[width=\textwidth, trim={0mm 0mm 0mm 0mm}, clip,page=6]{figures/mach3/2018/asimov/corr/2018a_MultiPi_Binningv6_NewCov_Asimov_merge_drawCorr}
	\end{subfigure}
	\caption{$\sqrt{\mathbf{V}_{i,j}}$ and correlation matrix for the Asimov post-fit, showing ND280 flux and cross-section parameters}
	\label{fig:asimov_nd_corr_2018}
\end{figure}

Comparing the flux covariances before and after the fit in \autoref{fig:asimov_flux_corr_2018}, it's clear that ND280 reduces the uncertainty but maintains the parameters correlations. However, four of the ND280 and SK \nuebar parameters appear weakly negatively correlated (-0.15), which is not present in the prefit covariance.
\begin{figure}[h]
	\begin{subfigure}[t]{\textwidth}
		\begin{subfigure}[t]{0.49\textwidth}
			\includegraphics[width=\textwidth, trim={0mm 0mm 0mm 0mm}, clip,page=8]{figures/mach3/2018/asimov/corr/2018a_MultiPi_Binningv6_NewCov_Asimov_merge_drawCorr}
		\end{subfigure}
		\begin{subfigure}[t]{0.49\textwidth}
			\includegraphics[width=\textwidth, trim={0mm 0mm 0mm 0mm}, clip,page=9]{figures/mach3/2018/asimov/corr/2018a_MultiPi_Binningv6_NewCov_Asimov_merge_drawCorr}
		\end{subfigure}
		\caption{Post-fit}
	\end{subfigure}
	
	\begin{subfigure}[t]{\textwidth}
		\begin{subfigure}[t]{0.49\textwidth}
			\includegraphics[width=\textwidth, trim={0mm 0mm 0mm 0mm}, clip,page=2]{figures/mach3/inputs/flux_covariance_banff_13av2.pdf}
		\end{subfigure}
		\begin{subfigure}[t]{0.49\textwidth}
			\includegraphics[width=\textwidth, trim={0mm 0mm 0mm 0mm}, clip,page=3]{figures/mach3/inputs/flux_covariance_banff_13av2.pdf}
		\end{subfigure}
		\caption{Pre-fit}
	\end{subfigure}
	\caption{$\sqrt{\mathbf{V}_{i,j}}$ and correlation matrix for the flux parameters pre and post-fit to Asimov data}
	\label{fig:asimov_flux_corr_2018}
\end{figure}

Finally comparing the cross-section correlation matrix from the fit to Asimov data from 2017 to the 2018 results in \autoref{fig:asimov_xsec_corr_2017_vs_2018} we generally see very small changes. The CC0$\pi$ parameters (bottom left corner) are unchanged, and the single pion parameters are correlating marginally more with the coherent parameters. The single pion and 2p2h normalisation correlations are slightly reduced, possibly due to the new RHC \numubar samples. The pion FSI parameters (upper right corner) are clearly less correlated in 2018, owing to the updated covariance matrix.
\begin{figure}[h]
	\begin{subfigure}[t]{0.49\textwidth}
		\includegraphics[width=\textwidth, trim={0mm 0mm 0mm 0mm}, clip,page=12]{figures/mach3/Asimov/2017b_NewDet_NewData_Asimov_Long_0_drawCorr.pdf}
		\caption{2017}
	\end{subfigure}
	\begin{subfigure}[t]{0.49\textwidth}
		\includegraphics[width=\textwidth, trim={0mm 0mm 0mm 0mm}, clip,page=12]{figures/mach3/2018/asimov/corr/2018a_MultiPi_Binningv6_NewCov_Asimov_merge_drawCorr}
		\caption{2018}
	\end{subfigure}
	\caption{Correlation matrix for the Asimov post-fit, showing cross-section parameters for 2017 and 2018 fits}
	\label{fig:asimov_xsec_corr_2017_vs_2018}
\end{figure}

\subsection{Posterior predictive}
The posterior predictive \pmu \cosmu spectrum and p-values are calculated using the MCMC with the reduced covariance matrix outlined above. The calculation proceeds the same as in \autoref{sec:asimov_pospred} using 20,000 randomly chosen steps after a conservative 1/4 burn-in (corresponding to 975,000 steps). 

For the posterior predictive p-values we expect values close to 1.0 since we're expecting a relatively tight post-fit constraints relative statistical fluctuations, as was the case in the 2017 analysis. \autoref{fig:postpred_asimov_2018} shows the two p-values, which both are exactly 1.0. 
\begin{figure}[h]
	\begin{subfigure}[t]{0.49\textwidth}
		\includegraphics[width=\textwidth, trim={0mm 0mm 0mm 11mm}, clip,page=1]{figures/mach3/2018/asimov/pred/2018a_MultiPi_Binningv6_NewCov_Asimov_merge_PostPredStore_SampLLH_procs}
	\end{subfigure}
	\begin{subfigure}[t]{0.49\textwidth}
		\includegraphics[width=\textwidth, trim={0mm 0mm 0mm 11mm}, clip,page=2]{figures/mach3/2018/asimov/pred/2018a_MultiPi_Binningv6_NewCov_Asimov_merge_PostPredStore_SampLLH_procs}
	\end{subfigure}
	\caption{Posterior predictive p-values for the Asimov data in 2018}
	\label{fig:postpred_asimov_2018}
\end{figure}

Moving attention to the posterior predictive spectrum's event rates in \autoref{tab:asimov_posterior_pred_2018}, we again see a very large reduction in the post-fit event rate compared to the prior predictive: from 12411.1 to 340.9 overall, 4245.4 to 168.3 for CC0$\pi$, 891.8 to 76.6 for CC1$\pi$ and 827.6 to 76.2 for CCOther. The reduction in uncertainty is comparable to 2017, and overall doubling the statistics has the effect of moving the percentage uncertainty from 0.40\% to 0.30\%, agreeing with the $1/\sqrt{2}$ expectation. The CC0$\pi$ uncertainty moves from 0.70\% to 0.53\%, CC1$\pi$ from 1.3\% to 0.96\%, CCOther from 1.4\% to 1.1\%. The anti-neutrino selections CC0$\pi$ moves from 1.5\% to 1.1\%, and NTrack moves from 2.0\% to 3.9\% 1$\pi$ and 2.9\% for Other: the increase coming from splitting the NTrk sample into lower statistics samples.
\begin{table}[h]
	\begin{tabular} {l | c c | c}
		\hline
		\hline
		Sample 			& Nominal	& Pos. Pred.	& $-2\log\mathcal{L}_S$ \\
		\hline
                FGD1 0$\pi$             & 31529.3   & $31545.3\pm168.3$   & 1.22   \\
                FGD1 1$\pi$             & 7998.1    & $8015.68\pm76.6$   & 0.70  \\
                FGD1 Other              & 6793.68   & $6804.29\pm76.2$   & 0.48  \\
\hline
                FGD2 0$\pi$             & 31734     & $31713.9\pm166.9$   & 0.99  \\
                FGD2 1$\pi$             & 6419.04   & $6428.7\pm68.2$    & 0.44  \\
                FGD2 Other              & 6562.75   & $6554.53\pm71.9$   & 0.38  \\
\hline
                FGD1 \numubar 0$\pi$    & 6371.34   & $6369.71\pm71.4$   & 0.56  \\
                FGD1 \numubar 1$\pi$    & 533.25   &  $537.87\pm20.8$   & 0.12  \\
                FGD1 \numubar Other     & 1023.36   & $1027.79\pm29.3$   & 0.05 \\
\hline
                FGD2 \numubar 0$\pi$    & 6283.35   & $6287.51\pm71.2$   & 0.47  \\
                FGD2 \numubar 1$\pi$    & 483.51   & $487.22\pm20.7$   & 0.04 \\
                FGD2 \numubar Other     & 943.96   & $946.53\pm28.1$   & 0.04 \\
\hline
                FGD1 \numu RHC 0$\pi$   & 2485.51   & $2513.19\pm42.1$   & 0.47  \\
                FGD1 \numu RHC 1$\pi$   & 855.91   & $844.85\pm13.7$   & 0.22  \\
                FGD1 \numu RHC Other    & 804.65   & $795.18\pm13.3$   & 0.14   \\
\hline
                FGD2 \numu RHC 0$\pi$   & 2553.51   & $2529.72\pm32.2$   & 0.35  \\
                FGD2 \numu RHC 1$\pi$   & 679.99    & $669.64\pm9.8$   & 0.18    \\
                FGD2 \numu RHC Other    & 792.17   & $783.53\pm13.1$   & 0.12  \\
                \hline
                Total					& 114847	& $114855.4\pm340.9$ & 6.98 \\
		\hline
		\hline
\end{tabular}
\caption{Posterior predictive event rates after fitting to the Asimov data}
\label{tab:asimov_posterior_pred_2018}
\end{table}

\section{Data fit}
\label{sec:data_2018}
The Asimov fit stumbled upon one main issue: the 1-2\% flux normalisation bias when using the multi-$\pi$ selection. With this in mind we keep the multi-track selection (which showed no bias) and perform data fits with it, the multi-$\pi$ selection with the reduced ND280 covariance matrix, and the multi-$\pi$ selection with the full ND280 covariance matrix.

The details for the MCMC that were obtained for the different fits are shown in \autoref{tab:mcmc_data_2018}. As for the Asimov case, the lower acceptance probability is a direct result of the number of ND280 parameters being fit: the multi-track fit has 556, the nominal covariance has 1076 and the full covariance has 4238. All chains were monitored for stability and often converged within 1/8th of the total steps requested. For the parameter plots we again use the conservative burn-in of 1/4.
\begin{table}[h]
	\begin{tabular}{l | c c c}
		\hline
		\hline
		Name		&	Step length & Acceptance & Accepted steps \\
		\hline
		Nominal cov	& 	3,900,000	& 12.2\%	 & 470,956 \\
		Full cov	& 	5,869,504	& 6.1\%		 & 358,039 \\
		Multi-track & 	1,500,000	& 17.1\%	 & 257,439 \\
		\hline
		\hline
	\end{tabular}
	\caption{Markov Chain parameters for the various data fits in \autoref{sec:data_2018}}
	\label{tab:mcmc_data_2018}
\end{table}

The summary for the data predictive distributions are shown in \autoref{tab:postfit_eventrate_2018} using the reduced ND280 covariance matrix. In contrast to the 2017 analysis, the prior predictive distribution predicts the RHC 1$\pi$ and Other distributions relatively well. This is reflected in the only marginal decrease in the test-statistic, e.g. 53.6 to 53.3 for FGD1 \numubar $1\pi$. For the same distribution we barely see a change in the prediction from the prefit to the postfit ($542.7\pm127.2$ to $542.9\pm21.6$). However, the prior predictive distributions do a poor job at predicting the central values of the data of the high statistics FHC distributions, although often inside the 1$\sigma$ uncertainties from the priors.

As in 2017, the fit reduces the uncertainties on the total event rate by almost two orders of magnitude; from 11.5\% to 0.29\% for the total, 14.1\% to 0.5\% for CC0$\pi$, 11.1\% to 1.0\% for CC1$\pi$, 12.9\% to 1.0\% for CCOther. For the new RHC selections we see reductions of 13.6\% to 1.1\% for CC0$\pi$, 23.4\% to 4.0\% for CC1$\pi$ and 19.8\% to 2.6\% for CCOther. 

Inspecting the test-statistic, the total $-2\log\mathcal{L}_S/\text{nBins}$ goes from 1.47 to 1.12, a deterioration from the 2017 value of 1.07. The fit is driven by the FHC 0$\pi$ distributions (35.6\% of statistic) and the 1$\pi$, Other and RHC 0$\pi$ making up between 10-15\% for both FGDs. The new RHC 1$\pi$ and Other distributions have a small contribution to the statistic in total and barely improve after the fit. Interestingly, all the CCOther distributions improve except FGD1 CC1$\pi$ RHC \numu.
\begin{sidewaystable}
	\centering
	\begin{tabular}{ l | c c c | c c }
		\hline
		\hline
		Sample 			& \multicolumn{3}{c|}{Event rate} & \multicolumn{2}{c}{$-2\ln\mathcal{L}_S$} \\
		& Data	& Prior & Posterior & Prior & Posterior \\
		\hline
		FGD1 0$\pi$ 	& 33553	& $32292.0\pm4121.4$ 	& $33614.9\pm171.8$ & 1096.0  & 834.5 	\\ 
		FGD1 1$\pi$ 	& 7757 	& $8101.2\pm911.2$	& $7966.8\pm77.8$  & 408.5  & 313.4 	\\ 
		FGD1 Other 		& 8068 	& $6925.9\pm886.5$	& $7869.6\pm80.5$  & 712.0  & 458.6 	\\ 
		\hline
		FGD2 0$\pi$ 	& 33462 & $32565.7\pm4102.7$	& $33397.3\pm172.7$ & 1156.0  & 868.1 	\\ 
		FGD2 1$\pi$ 	& 6133 	& $6519.8\pm720.4$	& $6271.6\pm66.6$  & 414.1 & 303.6 	\\ 
		FGD2 Other 		& 7664 	& $6694.8\pm816.1$	& $7496.1\pm76.9$  & 644.5  & 416.4 	\\ 
		\hline
		FGD1 \numubar 0$\pi$ 	& 6368 	& $6526.9\pm920.7$	& $6331.8\pm69.1$  & 409.3  & 358.3 	\\ 
		FGD1 \numubar 1$\pi$ 	& 535 	& $534.3\pm134.1$	& $542.9\pm21.6$  & 53.8  & 53.3 	\\ 
		FGD1 \numubar Other 	& 1102 	& $1024.7\pm203.9$	& $1100.6\pm29.0$  & 86.0  &  86.3	\\ 
		\hline
		FGD2 \numubar 0$\pi$ 	& 6451  & $6410.1\pm940.7$	& $6447.1\pm67.4$ & 441.3  & 406.7	\\ 
		FGD2 \numubar 1$\pi$ 	& 465 	& $476.6\pm118.8$	& $471.1\pm20.7$  & 45.8 & 40.04 	\\ 
		FGD2 \numubar Other 	& 1032 	& $1017.2\pm201.0$	& $1044.1\pm29.5$  & 119.3  & 104.71 	\\ 
		\hline
		FGD1 \numu RHC 0$\pi$ 	& 2707 	& $2541.2\pm442.2$	& $2713.4\pm44.5$  & 149.4  & 129.7 	\\ 
		FGD1 \numu RHC 1$\pi$ 	& 847 	& $860.3\pm108.0$	& $871.0\pm14.9$  & 60.3  & 51.1 	\\ 
		FGD1 \numu RHC Other 	& 1015 	& $1009.8\pm111.3$	& $928.9\pm14.6$  & 112.8  & 71.4 	\\ 
		\hline
		FGD2 \numu RHC 0$\pi$ 	& 2648  & $2598.7\pm445.1$	& $2738.0\pm34.8$ & 178.1  & 153.0 	\\ 
		FGD2 \numu RHC 1$\pi$ 	& 693 	& $689.3\pm89.6$	& $710.6\pm10.1$  & 79.6 & 73.4 	\\ 
		FGD2 \numu RHC Other 	& 932 	& $786.9\pm163.3$	& $913.0\pm14.4$  & 81.3  & 58.1 	\\ 
		\hline
		Total 				& 121432 & $117241.8\pm12775.9$	& $121429.3\pm352.3$ & 6248.1 & 4780.6 \\
		% Prior predictive updated, posterior _NOT_
		\hline
		\hline
	\end{tabular}
	\caption{Event rate and test-statistic for data, pre-fit MC and post-fit MC broken by sample, using the reduced ND280 covariance matrix}
	\label{tab:postfit_eventrate_2018}
\end{sidewaystable}

\subsection{Full and reduced ND280 covariance matrix}
As for the Asimov case in \autoref{sec:asimov_fit_2018}, we here compare the data fit results using the full and reduced ND280 covariance. We expect similar results for the two fits, as only small difference were found in the fit to Asimov data.

\autoref{fig:data_full_red_rhc} shows the FHC flux parameters for the two fits, in which we note compatibility for the two fits, similar to that seen in the Asimov. The divergence happens at higher energies, where it's likely that the reduced covariance matrix (which was binned to mostly ignore shapes at higher energies) is missing freedom to vary events which the full ND280 matrix fit has.

Looking at the shape of the FHC flux parameters, there is a tendency towards the nominal at higher energies and the majority of the shape change happens below 1 GeV. The shape below 1 GeV is to increase the flux by 10\% below 0.5 GeV, sitting at nominal at the flux peak of 0.6 GeV, to then decrease the flux by 7\% at 0.7 GeV up until 1.5 GeV. A similar shape was observed in the 2017 analysis in \autoref{fig:flux_data_fhc}, although these results are slightly shifted upwards.
\begin{figure}[h]
	\centering
	\begin{subfigure}[t]{0.10\textwidth}
		\includegraphics[width=\textwidth,page=1, trim={0mm 0mm 0mm 9mm}, clip]{figures/mach3/2018/data/2018a_MultiPi_Binningv6_FullCov_NoReduc_Data_merg_2018a_MultiPi_Binningv6_NewCov_Data_merge}
	\end{subfigure}
	
	\begin{subfigure}[t]{\textwidth}
	\begin{subfigure}[t]{0.24\textwidth}
		\includegraphics[width=\textwidth,page=2, trim={0mm 0mm 0mm 9mm}, clip]{figures/mach3/2018/data/2018a_MultiPi_Binningv6_FullCov_NoReduc_Data_merg_2018a_MultiPi_Binningv6_NewCov_Data_merge}
	\end{subfigure}
	\begin{subfigure}[t]{0.24\textwidth}
		\includegraphics[width=\textwidth,page=3, trim={0mm 0mm 0mm 9mm}, clip]{figures/mach3/2018/data/2018a_MultiPi_Binningv6_FullCov_NoReduc_Data_merg_2018a_MultiPi_Binningv6_NewCov_Data_merge}
	\end{subfigure}
	\begin{subfigure}[t]{0.24\textwidth}
		\includegraphics[width=\textwidth,page=4, trim={0mm 0mm 0mm 9mm}, clip]{figures/mach3/2018/data/2018a_MultiPi_Binningv6_FullCov_NoReduc_Data_merg_2018a_MultiPi_Binningv6_NewCov_Data_merge}
	\end{subfigure}
	\begin{subfigure}[t]{0.24\textwidth}
		\includegraphics[width=\textwidth,page=5, trim={0mm 0mm 0mm 9mm}, clip]{figures/mach3/2018/data/2018a_MultiPi_Binningv6_FullCov_NoReduc_Data_merg_2018a_MultiPi_Binningv6_NewCov_Data_merge}
	\end{subfigure}
\caption{ND280}
\end{subfigure}

	\begin{subfigure}[t]{\textwidth}
\begin{subfigure}[t]{0.24\textwidth}
	\includegraphics[width=\textwidth,page=10, trim={0mm 0mm 0mm 9mm}, clip]{figures/mach3/2018/data/2018a_MultiPi_Binningv6_FullCov_NoReduc_Data_merg_2018a_MultiPi_Binningv6_NewCov_Data_merge}
\end{subfigure}
\begin{subfigure}[t]{0.24\textwidth}
	\includegraphics[width=\textwidth,page=11, trim={0mm 0mm 0mm 9mm}, clip]{figures/mach3/2018/data/2018a_MultiPi_Binningv6_FullCov_NoReduc_Data_merg_2018a_MultiPi_Binningv6_NewCov_Data_merge}
\end{subfigure}
\begin{subfigure}[t]{0.24\textwidth}
	\includegraphics[width=\textwidth,page=12, trim={0mm 0mm 0mm 9mm}, clip]{figures/mach3/2018/data/2018a_MultiPi_Binningv6_FullCov_NoReduc_Data_merg_2018a_MultiPi_Binningv6_NewCov_Data_merge}
\end{subfigure}
\begin{subfigure}[t]{0.24\textwidth}
	\includegraphics[width=\textwidth,page=13, trim={0mm 0mm 0mm 9mm}, clip]{figures/mach3/2018/data/2018a_MultiPi_Binningv6_FullCov_NoReduc_Data_merg_2018a_MultiPi_Binningv6_NewCov_Data_merge}
\end{subfigure}
\caption{SK}
\end{subfigure}
	\caption{FHC flux parameters, fitting to data with different ND280 matrices}
	\label{fig:data_full_red_fhc}
\end{figure}

For the RHC flux parameters in \autoref{fig:data_full_red_rhc} we again note good agreement using the two covariance matrices. However, we see an entirely different shape of the flux to the FHC case, in which the low energy normalisation is nominal, falling to 95\% below 0.5 GeV and then sharply increasing to 10\% above 1 GeV, which then sink to nominal at 3 GeV. The wrong-sign flux is however increased at low energies, similar to the FHC flux parameters.

Additionally, ND280 and SK see a slightly different behaviour around 2 GeV, where ND280 prefers a 10\% increased normalisation and the SK parameter sits at nominal. Otherwise the parameters are well mirrored.
\begin{figure}[h]
	\centering
		\begin{subfigure}[t]{\textwidth}
	\begin{subfigure}[t]{0.24\textwidth}
		\includegraphics[width=\textwidth,page=6, trim={0mm 0mm 0mm 9mm}, clip]{figures/mach3/2018/data/2018a_MultiPi_Binningv6_FullCov_NoReduc_Data_merg_2018a_MultiPi_Binningv6_NewCov_Data_merge}
	\end{subfigure}
	\begin{subfigure}[t]{0.24\textwidth}
		\includegraphics[width=\textwidth,page=7, trim={0mm 0mm 0mm 9mm}, clip]{figures/mach3/2018/data/2018a_MultiPi_Binningv6_FullCov_NoReduc_Data_merg_2018a_MultiPi_Binningv6_NewCov_Data_merge}
	\end{subfigure}
	\begin{subfigure}[t]{0.24\textwidth}
		\includegraphics[width=\textwidth,page=8, trim={0mm 0mm 0mm 9mm}, clip]{figures/mach3/2018/data/2018a_MultiPi_Binningv6_FullCov_NoReduc_Data_merg_2018a_MultiPi_Binningv6_NewCov_Data_merge}
	\end{subfigure}
	\begin{subfigure}[t]{0.24\textwidth}
		\includegraphics[width=\textwidth,page=9, trim={0mm 0mm 0mm 9mm}, clip]{figures/mach3/2018/data/2018a_MultiPi_Binningv6_FullCov_NoReduc_Data_merg_2018a_MultiPi_Binningv6_NewCov_Data_merge}
	\end{subfigure}
\caption{ND280}
\end{subfigure}

	\begin{subfigure}[t]{\textwidth}
	\begin{subfigure}[t]{0.24\textwidth}
		\includegraphics[width=\textwidth,page=14, trim={0mm 0mm 0mm 9mm}, clip]{figures/mach3/2018/data/2018a_MultiPi_Binningv6_FullCov_NoReduc_Data_merg_2018a_MultiPi_Binningv6_NewCov_Data_merge}
	\end{subfigure}
	\begin{subfigure}[t]{0.24\textwidth}
		\includegraphics[width=\textwidth,page=15, trim={0mm 0mm 0mm 9mm}, clip]{figures/mach3/2018/data/2018a_MultiPi_Binningv6_FullCov_NoReduc_Data_merg_2018a_MultiPi_Binningv6_NewCov_Data_merge}
	\end{subfigure}
	\begin{subfigure}[t]{0.24\textwidth}
		\includegraphics[width=\textwidth,page=16, trim={0mm 0mm 0mm 9mm}, clip]{figures/mach3/2018/data/2018a_MultiPi_Binningv6_FullCov_NoReduc_Data_merg_2018a_MultiPi_Binningv6_NewCov_Data_merge}
	\end{subfigure}
	\begin{subfigure}[t]{0.24\textwidth}
		\includegraphics[width=\textwidth,page=17, trim={0mm 0mm 0mm 9mm}, clip]{figures/mach3/2018/data/2018a_MultiPi_Binningv6_FullCov_NoReduc_Data_merg_2018a_MultiPi_Binningv6_NewCov_Data_merge}
	\end{subfigure}
\caption{SK}
\end{subfigure}
	\caption{RHC flux parameters, fitting to data with different ND280 matrices}
	\label{fig:data_full_red_rhc}
\end{figure}

\autoref{fig:data_full_red_xsec} shows the interaction parameters, in which we again note mostly compatibility. Interestingly, $M_A^{QE}$ is pulled even further than in 2017, now fully compatible with bubble chamber results ($M_A^{QE}=1.04\pm0.06$). The 2p2h $\nu$ normalisation parameter is decreased to 2017 ($1.64\pm0.21$ to $1.31\pm0.17$), whereas the $\bar{\nu}$ parameter increases ($0.80\pm0.26$ to $0.91\pm0.23$). Importantly, the 2p2h normalisation parameter for $\bar{\nu}$ is the largest difference in using the two ND280 covariance matrices, just outside of each other's $1\sigma$ uncertainty. The 2p2h shape parameters now fit very similar values for $^{12}C$ and $^{16}O$ and are not pushed towards any boundaries. The BeRPA A parameter after the fit is much closer to the nominal compared to the 2017 result, although the opposite is true for BeRPA B, which now sits 2-3$\sigma$ from the prior central value.

The single pion parameters largely agree with 2017 albeit with smaller parameter errors. $C_5^A$ decreases slightly but still agrees with the prior uncertainty. The non-resonant $I_{1/2}$ background increases outside the 1$\sigma$ range but still has a large associated error. The CC DIS parameter moves more away from the prior---which would be expected if the poor CC Other disagreement still exists---and the coherent normalisation now fully agree with the prior. The NC other parameter remains at the upper boundary of the prior 1$\sigma$ uncertainty.

The pion FSI---which received new priors for this analysis---interestingly still sit close to their fitted values in 2017. The reduction in uncertainty is the most dramatic for the pion FSI parameters, just like in the Asimov case.
\begin{figure}[h]
	\centering
	\begin{subfigure}[t]{0.49\textwidth}
		\includegraphics[width=\textwidth,page=18, trim={0mm 0mm 0mm 9mm}, clip]{figures/mach3/2018/data/2018a_MultiPi_Binningv6_FullCov_NoReduc_Data_merg_2018a_MultiPi_Binningv6_NewCov_Data_merge}
	\end{subfigure}
	\begin{subfigure}[t]{0.49\textwidth}
		\includegraphics[width=\textwidth,page=19, trim={0mm 0mm 0mm 9mm}, clip]{figures/mach3/2018/data/2018a_MultiPi_Binningv6_FullCov_NoReduc_Data_merg_2018a_MultiPi_Binningv6_NewCov_Data_merge}
	\end{subfigure}
	
	\begin{subfigure}[t]{0.49\textwidth}
		\includegraphics[width=\textwidth,page=20, trim={0mm 0mm 0mm 9mm}, clip]{figures/mach3/2018/data/2018a_MultiPi_Binningv6_FullCov_NoReduc_Data_merg_2018a_MultiPi_Binningv6_NewCov_Data_merge}
	\end{subfigure}
	\begin{subfigure}[t]{0.49\textwidth}
		\includegraphics[width=\textwidth,page=21, trim={0mm 0mm 0mm 9mm}, clip]{figures/mach3/2018/data/2018a_MultiPi_Binningv6_FullCov_NoReduc_Data_merg_2018a_MultiPi_Binningv6_NewCov_Data_merge}
	\end{subfigure}
	\caption{Interaction parameters, fitting to data with different ND280 matrices}
	\label{fig:data_full_red_xsec}
\end{figure}

\autoref{fig:berpa_weight_2018} shows how the BeRPA weight has changed relative the 2017 weight and the nominal Nieves RPA. The behaviour below $Q^2\sim0.4\text{ GeV}^2$ is similar to 2017, although at $Q^2=0.5\text{ GeV}^2$ the enhancement is much stronger. The BeRPA parameterisation dies off more similar to the nominal than in 2017. In general though, the nominal RPA weight is heavily modified in the fit.
\begin{figure}[h]
	\includegraphics[width=0.3\textwidth, trim={0mm 0mm 0mm 0mm}, clip]{figures/mach3/2018/data/2018a_MultiPi_Binningv6_NewCov_Data_merge_BeRPA}
	\caption{BeRPA weight applied to CCQE events after the fit to data}
	\label{fig:berpa_weight_2018}
\end{figure}

\subsection{Comparison to multi-track}
A fit to data using the old multi-track selection with the 2017 binning using the run 2 to 8 data, was made. Experiences gained from the fit to Asimov data showed less bias in the flux parameters for the multi-track selection, although the likelihood scans showed the new selection with updated binning to be more sensitive to parameters.

\autoref{fig:data_multitrack_multipi_fhc}  show the FHC flux parameters after the data fit. For FHC \numu, the multi-track selection sits much closer to the priors and does not display the ``wavy'' oscillation between 0 and 1 GeV, seen when using the multi-$\pi$ selection. The multi-track FHC \numu flux parameters do however increase to $\sim8\%$ above 1.5 GeV, whereas the multi-$\pi$ is close to nominal. The SK FHC \numu parameters echo this behaviour.

Looking at the other distributions (\nue, \numubar, \nuebar), they are more compatible. There seems to be a trend of the multi-$\pi$ laying below the multi-track up to 1.5 GeV, which stays put for the \numu flux but switches for the other distributions. All the flux parameters are still within the prior uncertainty and each other's uncertainty
\begin{figure}[h]
	\centering
	\begin{subfigure}[t]{0.10\textwidth}
		\includegraphics[width=\textwidth,page=1, trim={0mm 0mm 0mm 9mm}, clip]{figures/mach3/2018/data/2018a_MultiPi_Binningv6_NewCov_Data_merg_2018a_NewDetMatrix_OrderSwitched_Data2to8_ActualData_merge}
	\end{subfigure}
	
	\begin{subfigure}[t]{\textwidth}
	\begin{subfigure}[t]{0.24\textwidth}
		\includegraphics[width=\textwidth,page=2, trim={0mm 0mm 0mm 9mm}, clip]{figures/mach3/2018/data/2018a_MultiPi_Binningv6_NewCov_Data_merg_2018a_NewDetMatrix_OrderSwitched_Data2to8_ActualData_merge}
	\end{subfigure}
	\begin{subfigure}[t]{0.24\textwidth}
		\includegraphics[width=\textwidth,page=3, trim={0mm 0mm 0mm 9mm}, clip]{figures/mach3/2018/data/2018a_MultiPi_Binningv6_NewCov_Data_merg_2018a_NewDetMatrix_OrderSwitched_Data2to8_ActualData_merge}
	\end{subfigure}
	\begin{subfigure}[t]{0.24\textwidth}
		\includegraphics[width=\textwidth,page=4, trim={0mm 0mm 0mm 9mm}, clip]{figures/mach3/2018/data/2018a_MultiPi_Binningv6_NewCov_Data_merg_2018a_NewDetMatrix_OrderSwitched_Data2to8_ActualData_merge}
	\end{subfigure}
	\begin{subfigure}[t]{0.24\textwidth}
		\includegraphics[width=\textwidth,page=5, trim={0mm 0mm 0mm 9mm}, clip]{figures/mach3/2018/data/2018a_MultiPi_Binningv6_NewCov_Data_merg_2018a_NewDetMatrix_OrderSwitched_Data2to8_ActualData_merge}
	\end{subfigure}
\caption{ND280}
\end{subfigure}

	\begin{subfigure}[t]{\textwidth}
\begin{subfigure}[t]{0.24\textwidth}
	\includegraphics[width=\textwidth,page=10, trim={0mm 0mm 0mm 9mm}, clip]{figures/mach3/2018/data/2018a_MultiPi_Binningv6_NewCov_Data_merg_2018a_NewDetMatrix_OrderSwitched_Data2to8_ActualData_merge}
\end{subfigure}
\begin{subfigure}[t]{0.24\textwidth}
	\includegraphics[width=\textwidth,page=11, trim={0mm 0mm 0mm 9mm}, clip]{figures/mach3/2018/data/2018a_MultiPi_Binningv6_NewCov_Data_merg_2018a_NewDetMatrix_OrderSwitched_Data2to8_ActualData_merge}
\end{subfigure}
\begin{subfigure}[t]{0.24\textwidth}
	\includegraphics[width=\textwidth,page=12, trim={0mm 0mm 0mm 9mm}, clip]{figures/mach3/2018/data/2018a_MultiPi_Binningv6_NewCov_Data_merg_2018a_NewDetMatrix_OrderSwitched_Data2to8_ActualData_merge}
\end{subfigure}
\begin{subfigure}[t]{0.24\textwidth}
	\includegraphics[width=\textwidth,page=13, trim={0mm 0mm 0mm 9mm}, clip]{figures/mach3/2018/data/2018a_MultiPi_Binningv6_NewCov_Data_merg_2018a_NewDetMatrix_OrderSwitched_Data2to8_ActualData_merge}
\end{subfigure}
\caption{SK}
\end{subfigure}
	
	\caption{FHC flux parameters, fitting to data with different selections}
	\label{fig:data_multitrack_multipi_fhc}
\end{figure}

Inspecting the RHC parameters in \autoref{fig:data_multitrack_multipi_rhc}, a similar pattern emerges: the ``wavy'' nature at low energy is slightly stronger for multi-$\pi$. Above 1 GeV, the multi-track sits close to the prior throughout, and the multi-$\pi$ follows a similar shape but with more extreme corrections.

For the wrong-sign component, the multi-$\pi$ sits closer to the nominal, and the multi-track agreeing in shape but with a different normalisation. Again, the two fits consistently sit within 1$\sigma$ of the prior uncertainty and each other's uncertainties.
\begin{figure}[h]
	\centering
		\begin{subfigure}[t]{\textwidth}
	\begin{subfigure}[t]{0.24\textwidth}
		\includegraphics[width=\textwidth,page=6, trim={0mm 0mm 0mm 9mm}, clip]{figures/mach3/2018/data/2018a_MultiPi_Binningv6_NewCov_Data_merg_2018a_NewDetMatrix_OrderSwitched_Data2to8_ActualData_merge}
	\end{subfigure}
	\begin{subfigure}[t]{0.24\textwidth}
		\includegraphics[width=\textwidth,page=7, trim={0mm 0mm 0mm 9mm}, clip]{figures/mach3/2018/data/2018a_MultiPi_Binningv6_NewCov_Data_merg_2018a_NewDetMatrix_OrderSwitched_Data2to8_ActualData_merge}
	\end{subfigure}
	\begin{subfigure}[t]{0.24\textwidth}
		\includegraphics[width=\textwidth,page=8, trim={0mm 0mm 0mm 9mm}, clip]{figures/mach3/2018/data/2018a_MultiPi_Binningv6_NewCov_Data_merg_2018a_NewDetMatrix_OrderSwitched_Data2to8_ActualData_merge}
	\end{subfigure}
	\begin{subfigure}[t]{0.24\textwidth}
		\includegraphics[width=\textwidth,page=9, trim={0mm 0mm 0mm 9mm}, clip]{figures/mach3/2018/data/2018a_MultiPi_Binningv6_NewCov_Data_merg_2018a_NewDetMatrix_OrderSwitched_Data2to8_ActualData_merge}
	\end{subfigure}
\caption{ND280}
\end{subfigure}

	\begin{subfigure}[t]{\textwidth}
	\begin{subfigure}[t]{0.24\textwidth}
		\includegraphics[width=\textwidth,page=14, trim={0mm 0mm 0mm 9mm}, clip]{figures/mach3/2018/data/2018a_MultiPi_Binningv6_NewCov_Data_merg_2018a_NewDetMatrix_OrderSwitched_Data2to8_ActualData_merge}
	\end{subfigure}
	\begin{subfigure}[t]{0.24\textwidth}
		\includegraphics[width=\textwidth,page=15, trim={0mm 0mm 0mm 9mm}, clip]{figures/mach3/2018/data/2018a_MultiPi_Binningv6_NewCov_Data_merg_2018a_NewDetMatrix_OrderSwitched_Data2to8_ActualData_merge}
	\end{subfigure}
	\begin{subfigure}[t]{0.24\textwidth}
		\includegraphics[width=\textwidth,page=16, trim={0mm 0mm 0mm 9mm}, clip]{figures/mach3/2018/data/2018a_MultiPi_Binningv6_NewCov_Data_merg_2018a_NewDetMatrix_OrderSwitched_Data2to8_ActualData_merge}
	\end{subfigure}
	\begin{subfigure}[t]{0.24\textwidth}
		\includegraphics[width=\textwidth,page=17, trim={0mm 0mm 0mm 9mm}, clip]{figures/mach3/2018/data/2018a_MultiPi_Binningv6_NewCov_Data_merg_2018a_NewDetMatrix_OrderSwitched_Data2to8_ActualData_merge}
	\end{subfigure}
\caption{SK}
\end{subfigure}
	\caption{RHC flux parameters, fitting to data with different selections}
	\label{fig:data_multitrack_multipi_rhc}
\end{figure}

Finally the cross-section parameters in \autoref{fig:data_multitrack_multipi_xsec} show consistent behaviour up until 2p2h normalisation for $\bar{\nu}$. The multi-$\pi$ selection prefers a normalisation much in line with the prior and has a reduced uncertainty, where the multi-track agrees much more with the value of the 2p2h normalisation for $\nu$. Compared to the 2017 value of $\sim0.8$, the multi-$\pi$ selection seems more compatible. The relative 2p2h $^{12}C/^{16}O$ normalisation parameter shows a preference for the nominal in the multi-track, whereas the multi-$\pi$ pushes it half-way down the prior.

The 2p2h shape C parameter sits close to the upper boundary for multi-track, similar to in 2017, within 1$\sigma$ of the multi-$\pi$ selection. The 2p2h shape O parameter shows a similar level of disagreement, with the multi-track sitting on the prior value, whereas the multi-$\pi$ selection shows a similar value to the carbon parameter. 

The BeRPA parameters are particularly interesting here: the multi-track selection is very similar to the 2017 values, whereas the multi-$\pi$ sees a ``softening'' of the effect at low $Q^2$ (BeRPA A), but a larger move from te prior for BeRPA B, even higher than in 2017 (previously seen in \autoref{fig:berpa_weight_2018}).

The single pion parameters are entirely compatible and similar to 2017, with the small exception of the non-resonant $I_{1/2}$ background which increases outside the prior. The CC coherent normalisations are similar to 2017 from the multi-track, sitting just on the 1$\sigma$ boundary, whereas the multi-$\pi$ agrees with the nominal.

Of the NC parameters, only NC other (encompassing NC1$\pi$ and NCDIS) see differences: the multi-track fit prefers a value in agreement with the prior's central value, whereas the multi-$\pi$ selection pulls outside the prior uncertainty.

The pion FSI parameters are expected to show large differences, as was the case in the Asimov studies of the fit and likelihood scans. This comes both from a rebinning of the FHC 1$\pi$ samples and the splitting of the NTrack into $1\pi$ and Other for RHC selections. The parameters are largely compatible, and the multi-$\pi$ consistently sits close to the prior value. Both fits still sit within the 1$\sigma$ uncertainty of the prior.
\begin{figure}[h]
	\centering
	\begin{subfigure}[t]{0.49\textwidth}
		\includegraphics[width=\textwidth,page=18, trim={0mm 0mm 0mm 9mm}, clip]{figures/mach3/2018/data/2018a_MultiPi_Binningv6_NewCov_Data_merg_2018a_NewDetMatrix_OrderSwitched_Data2to8_ActualData_merge}
	\end{subfigure}
	\begin{subfigure}[t]{0.49\textwidth}
		\includegraphics[width=\textwidth,page=19, trim={0mm 0mm 0mm 9mm}, clip]{figures/mach3/2018/data/2018a_MultiPi_Binningv6_NewCov_Data_merg_2018a_NewDetMatrix_OrderSwitched_Data2to8_ActualData_merge}
	\end{subfigure}
	
	\begin{subfigure}[t]{0.49\textwidth}
		\includegraphics[width=\textwidth,page=20, trim={0mm 0mm 0mm 9mm}, clip]{figures/mach3/2018/data/2018a_MultiPi_Binningv6_NewCov_Data_merg_2018a_NewDetMatrix_OrderSwitched_Data2to8_ActualData_merge}
	\end{subfigure}
	\begin{subfigure}[t]{0.49\textwidth}
		\includegraphics[width=\textwidth,page=21, trim={0mm 0mm 0mm 9mm}, clip]{figures/mach3/2018/data/2018a_MultiPi_Binningv6_NewCov_Data_merg_2018a_NewDetMatrix_OrderSwitched_Data2to8_ActualData_merge}
	\end{subfigure}
	\caption{Interaction parameters, fitting to data with different selections}
	\label{fig:data_multitrack_multipi_xsec}
\end{figure}

\subsection{Covariance matrix}


\subsection{Prior predictive}
Using the same method as in the Asimov study, we perform a check of the compatibility of the data with the model prescribed by the prior central values and uncertainty. In 2017 the prior model was deemed entirely incompatible with the data (\autoref{sec:prior_pred_data}), resulting in p-values of 0.0 for every selection. The p-values are calculated as two-dimensional, as described in \autoref{sec:Asimov_prior}.

\autoref{tab:data_pre_pvalue_2018} shows the prior predictive p-values, which are zero for all the high-statistics samples. The lower statistics RHC 1$\pi$ and Other \numubar and \numu selection have non-zero p-values, largely due to the sample size being less than 1000 in \autoref{tab:postfit_eventrate_2018}. The p-values are largely in agreement with the 2017 values in \autoref{tab:data_pre_pvalue_2018}, and the statistics increase of anti-neutrino data has brought the 1Trk (aka 0$\pi$) p-value to zero too. Curiously, the RHC \numubar 1$\pi$ selections both have acceptable p-values as was implied from the event rates in \autoref{tab:postfit_eventrate_2018}, although the small sample size is likely at effect here.
\begin{table}[h]
	\centering
	\begin{tabular}{l | c c }
		\hline \hline
		Sample & Draw Fluc. & Pred. Fluc. \\
		\hline
		FGD1 0$\pi$ & 0.000 & 0.000 \\
		FGD1 1$\pi$ & 0.000 & 0.000 \\
		FGD1 Other  & 0.000 & 0.000 \\
		\hline
		FGD2 0$\pi$ & 0.000 & 0.000 \\
		FGD2 1$\pi$ & 0.000 & 0.000 \\
		FGD2 Other  & 0.000 & 0.000 \\
		\hline
		FGD1 \numubar 0$\pi$ & 0.000 & 0.000 \\
		FGD1 \numubar 1$\pi$ & 0.075 & 0.075 \\
		FGD1 \numubar Other  & 0.073 & 0.076 \\
		\hline
		FGD2 \numubar 0$\pi$ & 0.000 & 0.000 \\
		FGD2 \numubar 1$\pi$ & 0.213 & 0.215 \\
		FGD2 \numubar Other  & 0.001 & 0.002 \\
		\hline
		FGD1 \numu RHC 0$\pi$ & 0.001 & 0.001 \\
		FGD1 \numu RHC 1$\pi$ & 0.026 & 0.023 \\
		FGD1 \numu RHC Other  & 0.011 & 0.013 \\
		\hline
		FGD2 \numu RHC 0$\pi$ & 0.001 & 0.002 \\
		FGD2 \numu RHC 1$\pi$ & 0.000 & 0.000 \\
		FGD2 \numu RHC Other  & 0.064 & 0.063 \\
		\hline
		\hline
	\end{tabular}
	\caption{Prior predictive p-values for each sample after the data fit}
	\label{tab:data_pre_pvalue_2018}
\end{table}

\subsection{Posterior predictive}
\begin{table}[h]
	\centering
	\begin{tabular}{l | c c }
		\hline \hline
		Sample & Draw Fluc. & Pred. Fluc. \\
		\hline
		FGD1 0$\pi$ & 0.211 & 0.208 \\
		FGD1 1$\pi$ & 0.062 & 0.061 \\
		\textcolor{red}{FGD1 Other}  & \textcolor{red}{0.000} & \textcolor{red}{0.000} \\
		\hline
		FGD2 0$\pi$ & 0.067 & 0.065 \\
		FGD2 1$\pi$ & 0.140 & 0.134 \\
		\textcolor{red}{FGD2 Other}  & \textcolor{red}{0.001} & \textcolor{red}{0.001} \\
		\hline
		FGD1 \numubar 0$\pi$ & 0.013 & 0.014 \\
		FGD1 \numubar 1$\pi$ & 0.288 & 0.292 \\
		FGD1 \numubar Other  & 0.288 & 0.296 \\
		\hline
		\textcolor{red}{FGD2} \numubar 0$\pi$ & \textcolor{red}{0.000} & \textcolor{red}{0.000} \\
		FGD2 \numubar 1$\pi$ & 0.784 & 0.786 \\
		FGD2 \numubar Other  & 0.038 & 0.036 \\
		\hline
		FGD1 \numu RHC 0$\pi$ & 0.146 & 0.150 \\
		FGD1 \numu RHC 1$\pi$ & 0.121 & 0.122 \\
		FGD1 \numu RHC Other  & 0.078 & 0.078 \\
		\hline
		FGD2 \numu RHC 0$\pi$ & 0.033 & 0.031 \\
		FGD2 \numu RHC 1$\pi$ & 0.001 & 0.002 \\
		FGD2 \numu RHC Other  & 0.373 & 0.366 \\
		\hline
		\hline
	\end{tabular}
	\caption{Posterior predictive p-values for each sample after the data fit}
	\label{tab:data_post_pvalue_2018}
\end{table}

\subsection{Post-fit distributions}


\subsection{Compatibility}
\subsubsection{FGD1 vs FGD2}
Downloading
\subsubsection{Neutrino vs anti-neutrino}
Downloading
