\chapter{Neutrino Physics}
\label{chap:theory}

\section{The Discoveries of the Neutrinos}
Neutrinos were initially proposed as a solution to the apparent violation of the conservation of four and angular momentum in James Chadwick measurements of the beta decay in 1932\cite{Chadwick1,Chadwick2}. Inspired by Wolfgang Pauli's new elementary particle ``the neutron'' (which had characteristics of what we today call a nucleon and a neutrino)\cite{pauli_1933}, Enrico Fermi built his theory of $\beta$-decay\cite{fermi_1934}, in which the observable process $n \rightarrow p + e^-$ is always accompanied by an invisible four-momentum carrier, the electron anti-neutrino.

The neutrino remained elusive until Reines and Cowan devised experiments\cite{reines_cowan_1,reines_cowan_2} in 1953, using the inverse beta decay (IBD) process, $\bar{\nu}_e + p \rightarrow n + e^+$, near a nuclear reactor. The experiment consistent of two tanks of water (100 L fiducial mass), sandwiched by liquid scintillator tanks with PMTs. The water was doped with 40 kg $\text{CdCl}_2$, which could detect free neutrons through capture. The electron anti-neutrinos were emitted by the nuclear reactor, interacted with the protons in the water, producing a prompt signal from $e^+ + e^- \rightarrow 2\gamma$. The free neutron was detected $\sim5\mu\text{s}$ after the prompt $2\gamma$ from $n + ^{108}\text{Cd} \rightarrow ^{109m}\text{Cd} \rightarrow ^{109}\text{Cd} + \gamma$. The experiment also took data from a reactor off period, demonstrating a significant reduction in neutrino event rates. The experiment was complemented by measurements by R. Davis\cite{davis} in 1964, which exposed tanks of $^{37}\text{Cl}$ to reactor electron anti-neutrinos, interacting through $\bar{\nu}_e + ^{37}\text{Cl} \rightarrow e^- + ^{37}\text{Ar}$, which would violate lepton number conservation. The experiment found no excess of $^{37}\text{Ar}$, and instead set limits on the solar neutrino flux.

The field quickly developed, and in 1962 Lederman, Schwartz, Steinberger and others\cite{lederman} observed another flavour of neutrino, the muon neutrino. They used a beam of protons impinging a target, creating a $\pi$ dominated beam which decayed following $\pi^+ \rightarrow \mu^+ + \nu_\mu$, and looked for subsequent interactions of the $\nu_\mu$ in a 10 tonne shielded aluminium spark chamber. The experiment was later confirmed by measurements at CERN in 1964\cite{cern_spark,cern_spark2}.

When the third charged lepton, the $\tau$, was discovered at SLAC's $e^+e^-$ accelerator in 1975\cite{tau_disc}, the search for its neutrino partner started. Its existence was already hinted at in $\tau$ decays, but was ultimately discovered at DONUT\cite{tau_nu_disc} in 2000. The discovery of the $\nu_\tau$ and the three neutrino flavours was confirmed by $Z \rightarrow l^+ l^-$ decays at LEP and SLAC\cite{lep}, which found the number of active neutrino flavours, assuming the Standard Model, as $N_\nu = 2.9840\pm0.0082$. This has also been confirmed by cosmological data from Planck and others\cite{planck}, $N_\text{eff} = 3.04\pm0.18$\footnote{$N_\text{eff}=3.0\pm0.4$ and $\sum m_\nu < 0.22 \text{ eV}$ when varying both $N_\text{eff}$ and $\sum m_\nu$.}.

\section{Neutrino Oscillations}
Neutrino oscillations is today an established physics phenomena, cemented by Kajita-san and Art McDonald receiving the Nobel Prize in Physics in 2015. This section briefly introduce neutrino oscillation experiments and gives an overview

\subsection{Solar Neutrinos}
Solar neutrinos emanate from various nuclear fusion products and decays in the sun. \autoref{tab:solar_flux} shows the fluxes for various sources, where clearly the $pp$ flux is strongest. However, the neutrino energy is often below threshold for the largest contributors to the flux, and most solar neutrino experiments measure the $^{8}\text{B}$ flux, shown in \autoref{fig:solar_flux}.
\begin{table}[h]
	\begin{tabular}{l | c c}
		\hline
		\hline
		Reaction & Label & Flux ($\text{cm}^{-2} \text{s}^{-1}$) \\
		\hline
		$p+p\rightarrow ^{2}\text{H} + e^+ + \nu_e$ & $pp$ & $5.95\times10^{10}$ \\
		$p+e^-+p\rightarrow ^{2}\text{H} \nu_e$ & $pep$ & $1.40\times10^{8}$ \\
		$^{3}\text{He} + p\rightarrow ^{4}\text{H} + e^+ + \nu_e$ & $hep$ & $9.3\times10^{3}$ \\
		$^{7}\text{Be} + e^- \rightarrow ^{7}\text{Li} + \nu_e$ & $^{7}\text{Be}$ & $4.77\times10^{9}$ \\
		$^{8}\text{B} \rightarrow ^{8}\text{Be}* + e^+ \nu_e$ & $^{8}\text{B}$ & $5.05\times10^{6}$ \\
		\hline
		\hline
	\end{tabular}
	\caption{Integrated solar neutrino flux from various solar processes in the $pp$ chain. Table replicated from \cite{solar_review}.}
	\label{tab:solar_flux}
\end{table}

\begin{figure}[h]
	\includegraphics[width=0.5\textwidth, trim={0mm 0mm 0mm 0mm}, clip,page=1]{figures/theory/solar_flux}
	\caption{Solar flux from different $pp$ chain fusion sources, including thresholds of experiments. Figure from \cite{sno_solar_flux}.}
	\label{fig:solar_flux}
\end{figure}

R. Davis and J. Bachall would continue measurements of the solar neutrinos from $^{8}\text{B}$ and in 1968\cite{davis_sun} announced a solar $\nu_e$ flux a factor seven of expected ($\sim2\sigma$ significance), largely attributed to solar model calculations. This was the birth of the ``solar neutrino problem'', which Bruno Pontecorvo and Vladimir Gribov in 1969\cite{pontecorvo_gribov} proposed a solution to by invoking a $\nu_e\leftrightarrow\nu_\mu$ oscillation similar to $K^0 \leftrightarrow\bar{K}^0$. In 1989, the Kamiokande experiment\cite{kamiokande_solar} confirmed the result, measuring a $^{8}\text{B}$ solar neutrino flux of $\sim0.5$ the expected, agreeing with the higher statistic data from Homestake\cite{davis_sun2}. The solar neutrino deficit was confirmed from the low threshold detectors SAGE\cite{sage_solar} and GALLEX\cite{gallex_solar}, additionally capable of detecting $p p$ neutrinos using $^{71}\text{Ga}+\nu_e \rightarrow ^{71}\text{Ge}+e^-$.

The Sudbury Neutrino Observatory (SNO) put the nail in the coffin in 2002\cite{sno_solar} by measuring the solar $\nu$ from $^{8}\text{B}$ in three channels: $\nu_e + d \rightarrow p+p+e^-$ (CC), $\nu_x + d\rightarrow p+ n + \nu_x$ (NC) and $\nu_x + e^- \rightarrow \nu_x+e^-$ (ES). The measured fluxes had a $\nu_e$ component consistent with previous measurements, a strong non-$\nu_e$ component 5.3$\sigma$ above zero, and a NC component consistent with predictions from solar models.

Additionally, the low threshold, low background, Borexino experiment has detected solar neutrinos from the $^{8}\text{B}$, $^{7}\text{Be}$, $pep$, CNO and $pp$ processes\cite{borexino_summary}. The next-generation SNO experiment, SNO+, aims to confirm and improve these measurements, and make detailed measurements of the MSW effect, solar metallicty and luminosity.

\subsection{Atmospheric Neutrinos}
Atmospheric neutrinos are emitted when cosmic rays interact with nuclei in the earth's atmosphere, producing mesons which decay into neutrinos, amongst others. The primary decay is the pion decay,
\begin{gather*}
	\pi^\pm \rightarrow \mu^\pm + \nu_\mu(\bar{\nu}_\mu) \\
	\mu^\pm \rightarrow e^\pm + \bar{\nu}_\mu (\nu_\mu) + \nu_e (\bar{\nu}_e)
\end{gather*}
giving rise to a total of three neutrinos. The neutrino flux from Honda\cite{honda_flux} is shown in \autoref{fig:atmos_flux}, which peaks in the 1-100 GeV region, notably higher than the solar neutrinos.
\begin{figure}[h]
	\includegraphics[width=0.7\textwidth, trim={0mm 0mm 0mm 0mm}, clip,page=1]{figures/theory/honda_flux}
	\caption{Atmospheric neutrino flux from \cite{honda_flux}.}
	\label{fig:atmos_flux}
\end{figure}

In 1965 F. Reines\cite{reines_atmos} and C.V. Achar\cite{india_atmos_hint} first saw hints of atmospheric $\nu_\mu$ appearance in deep underground laboratories through $\nu_\mu(\bar{\nu}_\mu) + X \rightarrow \mu^\pm + X'$. The Irvine-Michigan-Brookhaven (IMB) experiment\cite{imb} observed deficits of $\nu_\mu$ in 1986, and Kamiokande II in 1988\cite{kamiokande_atmos_hint} verified this and found muon-like events of $59\pm7\%$ the prediction, although good agreement of electron-like single-prong events. The Soudan-2 experiment\cite{soudan2} also saw indications of muon neutrino deficiency, seeing a neutrino flavour ratio of $0.72\pm0.19^{+0.05}_{-0.07}$ relative expectation. 

When Super-Kamiokande in 1998 published\cite{sk_disc} their high-statistics (4353 fully-contained+301 partially-contained) $\nu_\mu$ data, they found $R=\left( \mu/e \right)_\text{Data}/\left( \mu/e \right)_\text{MC} =0.65\pm0.05\pm0.08$. They additionally fitted the oscillation parameters, finding the data was well described by $\nu_\mu \leftrightarrow \nu_\tau$ rather than $\nu_\mu \leftrightarrow \nu_e$. The summary of atmospheric $\nu$ flavour ratios is seen in \autoref{fig:atmos_ratio}, where the majority of the high precision data sits at $R=0.6-0.8$.
\begin{figure}[h]
	\includegraphics[width=0.7\textwidth, trim={0mm 0mm 0mm 0mm}, clip,page=1]{figures/theory/flavour_ratio}
	\caption{Measured flavour ratios for various atmospheric neutrino experiments. Figure from \cite{kajita_summary}.}
	\label{fig:atmos_ratio}
\end{figure}

Atmospheric neutrino observatories after the mid 2000s have focussed on measuring the $\nu_\mu\rightarrow\nu_\mu$ to increasing precision. Furthermore, by isolating regions of specific zenith angle (and so baseline $L$), the extent of the matter effects are also studied, which may resolve the ordering of the mass states. This is largely the focus of IceCube\cite{icecube}, ANTARES\cite{antares}, SNO+ and Super-Kamiokande's atmospheric neutrino programme. Super-Kamiokande has also made attempts at isolating $\nu_\tau$ events\cite{superk_tau}, claiming $4.6\sigma$ discovery of $\nu_\tau$ appearance in 2017. Recent results can be seen in \autoref{fig:atmos_data}.
\begin{figure}[h]
	\includegraphics[width=0.7\textwidth, trim={0mm 0mm 0mm 0mm}, clip,page=1]{figures/theory/icecube_comp}
	\caption{Measured atmospheric oscillation parameters from recent atmospheric and long baseline accelerator neutrino experiments, assuming normal ordering. Figure from \cite{icecube}.}
	\label{fig:atmos_data}
\end{figure}

\subsection{Accelerator Neutrinos}
Neutrinos from accelerators are similar to atmospheric neutrinos in energy, baseline and production mechanism. Generally, these neutrinos are born from proton beam impinging on targets, producing a flurry of mesons, which decay into neutrinos amongst others. The main advantage over using atmospheric neutrinos is the relatively precise tuning of the neutrino flux which enables specific $L/E$ to be selected. The disadvantage is the reduced flux, generally reducing the baseline to a few 100 km and the impact of the matter effect.

MINOS and K2K long baseline neutrinos

OPERA and ICARUS $\nu_\tau$ appearance

T2K long baseline $\nu_e$ appearance, NO$\nu$A

NOvA $\bar{\nu}_e$ appearance

\subsection{Reactor Anti-Neutrinos}
KamLAND and reactor neutrinos
Short baseline reactor neutrinos and theta 13

\section{Neutrino Oscillations}
\label{sec:theory:osc}
This thesis could largely proceed without mentioning neutrino oscillations, Two-flavour model, See-saw model?

\section{Neutrino Interactions}
\label{sec:theory:int}
Invoking ideas from electron scattering