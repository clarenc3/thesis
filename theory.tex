\chapter{Theory}
\label{chap:theory}

\section{A Brief History}

\subsection{The Discoveries of the Neutrinos}
Neutrinos were initially proposed as a solution to the apparent violation of the conservation of four and angular momentum in James Chadwick measurements of the beta decay in 1932\cite{Chadwick1,Chadwick2}. Inspired by Wolfgang Pauli's new elementary particle ``the neutron'' (which had characteristics of what we today call a nucleon and a neutrino)\cite{pauli_1933}, Enrico Fermi built his theory of $\beta$-decay\cite{fermi_1934}, in which the observable process $n \rightarrow p + e^-$ is always accompanied by an invisible four-momentum carrier, the electron anti-neutrino.

The neutrino remained elusive until Reines and Cowan devised experiments\cite{reines_cowan_1,reines_cowan_2} in 1953, using the inverse beta decay (IBD) process, $\bar{\nu}_e + p \rightarrow n + e^+$, near a nuclear reactor. The experiment consistent of two tanks of water (100 L fiducial mass), sandwiched by liquid scintillator tanks with PMTs. The water was doped with 40 kg $\text{CdCl}_2$, which could detect free neutrons through capture. The electron anti-neutrinos were emitted by the nuclear reactor, interacted with the protons in the water, producing a prompt signal from $e^+ + e^- \rightarrow 2\gamma$. The free neutron was detected $\sim5\mu\text{s}$ after the prompt $2\gamma$ from $n + ^{108}\text{Cd} \rightarrow ^{109m}\text{Cd} \rightarrow ^{109}\text{Cd} + \gamma$. The experiment also took data from a reactor off period, demonstrating a significant reduction in neutrino event rates. The experiment was complemented by measurements by R. Davis\cite{davis} in 1964, which exposed tanks of $^{37}\text{Cl}$ to reactor electron anti-neutrinos, interacting through $\bar{\nu}_e + ^{37}\text{Cl} \rightarrow e^- + ^{37}\text{Ar}$, which would violate lepton number conservation. The experiment found no excess of $^{37}\text{Ar}$, and instead set limits on the solar neutrino flux.

The field quickly developed, and in 1962 Lederman, Schwartz, Steinberger and others\cite{lederman} observed another flavour of neutrino, the muon neutrino. They used a beam of protons impinging a target, creating a $\pi$ dominated beam which decayed following $\pi^+ \rightarrow \mu^+ + \nu_\mu$, and looked for subsequent interactions of the $\nu_\mu$ in a 10 tonne shielded aluminium spark chamber. The experiment was later confirmed by measurements at CERN in 1964\cite{cern_spark,cern_spark2}.

When the third charged lepton, the $\tau$, was discovered at SLAC's $e^+e^-$ accelerator in 1975\cite{tau_disc}, the search for its neutrino partner started. Its existence was already hinted at in $\tau$ decays, but was ultimately discovered at DONUT\cite{tau_nu_disc} in 2000. The discovery of the $\nu_\tau$ and the three neutrino flavours was confirmed by $Z \rightarrow l^+ l^-$ decays at LEP and SLAC\cite{lep}, which found the number of active neutrino flavour (assuming the Standard Model), $N_\nu = 2.9840\pm0.0082$. This has also been confirmed by cosmological data from Planck and others\cite{planck}, $N_\text{eff} = 3.04\pm0.18$.

\subsection{Neutrino Oscillations}

R. Davis at Homestake, South Dakota, and the solar neutrino problem. Bruno Pontecorvo's solution.

Super-K solar neutrino flux supporting Davis' result. Flux around 0.5.

GALLEX and SAGE \nue disappearance

SNO 2002, measuring \nue and total neutrino flux: in agreement with standard solar model. nue to total was 0.301 pm 0.033

Super-K measuring numu disappearance, also seen by Kamiokande and IMB

KamLAND and reactor neutrinos

MINOS and K2K long baseline neutrinos

OPERA and ICARUS nutau appearance

Short baseline reactor neutrinos and theta 13

T2K long baseline nue appearance, NOvA

SK nutau appearance

NOvA nuebar appearance

Number of active neutrinos from LEP data and CMB data. Decoherence from e.g. SN1987A

\section{Neutrino Oscillations in a Nutshell}
\label{sec:theory:osc}
This thesis could largely proceed without mentioning neutrino oscillations

Two-flavour model

See-saw model

\section{Neutrino Interactions in a Nutshell}
\label{sec:theory:int}
Invoking ideas from electron scattering