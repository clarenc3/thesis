\section{Systematics}
\label{sec:syst}
The fit's main goal is to minimise impact of systematic parameters for T2K-SK oscillation analyses by using near-detector data. The shared parameters between ND280 and SK are the neutrino flux parameters (since T2K and SK are in the same neutrino ``beamline''), and neutrino-nucleus interaction parameters. Nuisance parameters are mostly ND280 detector parameters and cross-section parameters that are parametrised as only effective on Carbon. As such, there are many ``parameters of interest'', which the following sections cover.

\subsection{Flux}
\label{subsec:syst_flux}
The flux systematics are evaluated by varying underlying parameters in the flux model and using NA61/SHINE \red{CITE} data. [7] N. Antoniou et al. CERN-SPSC-2006-034, 2006.

The NA61 data comes from runs using the thin target, whereas the T2K target is roughly 2 interaction lengths The data is $\pi^\pm$, $K^\pm$, $K^0_S$ and $\rho^+$.

HARP data used for pion rescattering.

Talk about the corrections in psyche
\begin{itemize}
  \item ND280, SK FHC \numu; ND280, SK RHC \numubar, :\\
    $E_\nu^{true}$: 0, 0.4, 0.5, 0.6, 0.7, 1, 1.5, 2.5, 3.5, 5, 7, 30

  \item ND280, SK FHC \numubar; ND280, SK RHC \numu:\\
    $E_\nu^{true}$: 0, 0.7, 1, 1.5, 2.5, 30

  \item ND280, SK FHC \nue; ND280, SK RHC \nuebar:\\
    $E_\nu^{true}$: 0, 0.5, 0.7, 0.8, 1.5, 2.5, 4, 30

  \item ND280, SK FHC \nuebar; ND280, SK RHC \nue:\\
    $E_\nu^{true}$: 0, 2.5, 30
\end{itemize}

\subsection{Detector systematics}
\label{subsec:syst_nd280}
The treatment of ND280 detector systematic uncertainties consists of varying the underlying detector systematics, such as TPC PID, FGD PID, TPC Momentum scale\red{talk about all the detector systematics in summary} and study the impact on the number of predicted events in each \pmu \cosmu bin. 

There are three categories of underlying ND280 systematics: observable-variation systematics, efficiency like systematics, normalisation systematic

The event variation in each \pmu \cosmu bin is assumed to be Gaussian \red{show examples of good Gaussian bins and bad ones} 

see TN 212 p 95
Field distortions
Momentum resolution
Momentum scale
TPC PID
FGD PID
Time of flight

charge ID efficiency
tpc cluster eff
tpc track eff
fgd track eff
tpc-fgd matching eff
michel electron
oofv bckg
pile-up
fgd mass
pion SI

all magnet
sand muon bkgd
total

read p39 onwards about syst

The number of detector systematics still decreased slightly from 580 to 556 due to merging bins with similar detector systematic effects for the rebinned samples.
All the consecutive bins with similar systematic values have been merged in order to reach a lower number of parameters (556) while the number of bins in the fit increased a lot (1624 bins), keeping the number of parameters under control.
It has been demonstrated with Asimov fits, the results of which are shown in \autoref{fig:2017_rebin_asimov}, that the effect of this rebinning was small.

The effect of changing the ND280 systematics binning was done with an older cross-section model than which the fit was completed in. This was due to a late delivery of the cross-section model \red{write an appendix on this model}. The flux parameters were entirely consistent.
\begin{table}
	\centering
	\begin{tabular}{ l | c | c }
		\hline
		Parameter & Fit binning & Similar syst. merge \\
		\hline
		\hline
		FSI INEL LO & $0.0 \pm 0.202$ & $0.0\pm0.200$ \\
		FSI INEL HI & $0.0 \pm 0.235$ & $0.0\pm0.233$ \\
		FSI PI PROD & $0.0 \pm 0.347$ & $0.0\pm0.344$ \\
		FSI CEX LO  & $0.0 \pm 0.416$ & $0.0\pm0.412$ \\
		FSI CEX HI  & $0.0 \pm 0.193$ & $0.0\pm0.191$ \\
		$M_A^{QE}$  & $1.2 \pm 0.0517$ & $1.2\pm0.0512$ \\
		$p_F^{C}$   & $217 \pm 36.96$ & $217\pm36.021$ \\
		2p2h norm C & $100 \pm 30.79$ & $100\pm30.56$ \\
		$E_B^{C}$   & $25.0 \pm 8.57$ & $25.0\pm8.56$ \\
		$p_F^{O}$   & $225 \pm 57.61$  & $225\pm56.16$ \\
		2p2h norm O & $100 \pm 277.98$ & $0.0\pm272.62$ \\
		$E_B^{C}$   & $27.0 \pm 9.00$ & $27.0\pm9.00$ \\
		$C_5^A$		& $1.01 \pm 0.066$ & $1.01\pm0.064$ \\
		$M_A^{1\pi}$ & $0.95 \pm 0.060$ & $0.95\pm0.059$ \\
		$I_{1/2}$ non-res & $1.30 \pm 0.180$ & $1.30\pm0.180$ \\
		CC $\nu_e$ norm & $1.00 \pm 0.030$ & $1.00\pm0.030$ \\
		DIS Shape	& $0.00 \pm 0.208$ & $0.0\pm0.208$ \\
		CC Coherent norm & $1.0 \pm 0.258$ & $1.0\pm0.257$ \\
		NC Coherent norm & $1.0 \pm 0.299$ & $1.0\pm0.299$ \\
		NC Other & $1.0 \pm 0.182$ & $1.0\pm0.181$ \\
		2p2h $\bar{\nu}$ & $1.0 \pm 0.332$ & $1.0\pm0.329$ \\
		\hline
	\end{tabular}
	\caption{Cross-section parameter results from comparisons using fit binning and a merged binning for the 2015 cross-section model.}
\label{fig:2017_rebin_asimov}
\end{table}

The merged systematics binning was improved to:
\begin{itemize}
	\item FHC $\nu_{\mu}$~CC0$\pi$ bin edges: \\
	\pmu (MeV/c): 0, 1000, 1250, 2000, 3000, 5000, 30000 \\
	\cosmu:  -1, 0.6, 0.7, 0.8, 0.85,0.94, 0.96, 1
	\item FHC $\nu_{\mu}$~CC1$\pi$  bin edges: \\
	\pmu (MeV/c):  0, 300, 1250, 1500, 5000, 30000 \\
	\cosmu: -1, 0.7, 0.85, 0.9, 0.92, 0.96, 0.98, 0.99, 1
	\item FHC $\nu_{\mu}$~CCOther bin edges: \\
	\pmu (MeV/c): 0, 1500, 2000, 3000, 5000, 30000 \\
	\cosmu:  -1, 0.8, 0.85, 0.9, 0.92, 0.96, 0.98, 0.99, 1
	\item RHC $\bar{\nu}_{\mu}$~CC 1-Track bin edges: \\
	\pmu (MeV/c): 0, 400, 900, 1100, 2000, 10000 \\
	\cosmu:  -1, 0.6, 0.7, 0.88, 0.95, 0.97, 0.98, 0.99, 1.00
	\item RHC $\bar{\nu}_{\mu}$~CC N-Track bin edges: \\
	\pmu (MeV/c):  0, 700, 1200, 1500, 2000, 3000, 10000 \\
	\cosmu: -1, 0.85, 0.88, 0.93, 0.98, 0.99, 1.00
	\item RHC $\nu_{\mu}$~CC 1-Track bin edges: \\
	\pmu (MeV/c):  0, 400, 800, 1100, 2000, 10000 \\
	\cosmu:   -1, 0.7, 0.85, 0.90, 0.93, 0.96, 0.98, 0.99, 1.00
	\item RHC $\nu_{\mu}$~CC N-Track bin edges: \\
	\pmu (MeV/c):  0, 1000, 1500, 2000, 3000, 10000 \\
	\cosmu: -1, 0.8, 0.90, 0.93, 0.95, 0.96, 0.97, 0.99, 1.00
\end{itemize}
Be mega clear that each of these above make up one parameter

\subsection{Cross-section}
\label{subsec:syst_xsec}
splines vs normalisation parameter, Maybe extra bit on BeRPA because of involvement
