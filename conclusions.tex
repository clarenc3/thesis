\chapter{Conclusion and Remarks}
\label{chap:conclusion}
The work in this thesis presented a Bayesian Markov Chain Monte Carlo method for reducing systematic uncertainties in the long baseline neutrino oscillation experiment Tokai to Kamioka (T2K). The constraints on systematics came from external sources, such as hadron production and neutrino cross-section experiments, as well as internal near detector data from the INGRID and ND280 detectors. The framework uses a simultaneous near and far detector analysis, expanding on using a frequentist ND280-only fit to constrain the systematics in analyses with far detector data. The MCMC method avoids making assumptions on the probability density functions, scales considerably better with parameter dimensionality, and does not suffer when there are local minima or discontinuities in the likelihoods. Compared to conventional frequentist likelihood minimisation, the MCMC method is much more suitable for future large neutrino oscillation fits requiring complex treatment of systematics, such as the proposed T2K--Super-Kamiokande, T2K-NO$\nu$A, Hyper-Kamiokande and DUNE analyses.

The uncertainties on the predicted event rates at Super-Kamiokande were reduced from 12-14\% to 2-4\%, enabling world-leading constraints on multiple oscillation parameters to be extracted at T2K, unachievable without using near detector data. The predictability of the model with the new constraints delivered p-values above 5\%, with one exception out of fourteen selections. The first analysis presented herein introduced a more robust systematics parameterisation, notably on the description of neutrino interactions in the nuclear medium. It was used in the official 2017 T2K publications and the results presented at Neutrino 2018. The second analysis expanded on the first, using new ND280 selections and almost double the near detector data, and will be used in official analyses after 2018. A future sensitivity study---using the full projected 7.8E21 protons-on-target---found uncertainties on the far detector event rates at the 1-2\% scale, fulfilling the requirements of the future long baseline neutrino oscillation experiments DUNE and Hyper-Kamiokande.

The analyses uncovered unsatisfactory interaction modelling and parameter values disagreeing with the priors from external data. The multi-particle ``CC Other'' selection at ND280 was found to be poorly modelled with p-values of 0.000. Numerous compatibility studies---comparing subsets of data with each other and variations of the interaction model and priors---were found to have a negligible effect on the event spectra at Super-Kamiokande. In the second analysis, the parameterisation of ND280 detector systematics and correct neutrino/anti-neutrino modelling had the largest effect on the predicted event spectra at Super-Kamiokande.

Looking ahead, the analyses will in the near future benefit from updated modelling. There are new models for CCQE (z-expansion\cite{z-exp}, multi-nucleon\cite{nieves1}), and single pion production\cite{thesis_minoo} being implemented. The priors on the flux uncertainties are expected to be reduced by $\sim50\%$\cite{flux_red} using hadron production data from the NA61/SHINE T2K replica target run. Efforts are also underway to include high and backwards going FGD selections which include ECal PID information, expanding kinematic phase space and increasing statistics by 20-30\%. Furthermore, dedicated $\nu_e$ selections are being developed to help inform $\nu_e/\nu_\mu$ differences. Lastly, exclusive P0D selections will contribute a 100-150\% increase in statistics, especially on neutrino-water interactions.

Finally, the ND280 upgrades\cite{t2k_upgrades} for the T2K-II project\cite{t2k_ii} will add another fine grained detector with a cuboid design, physically replacing the P0D. This will enable finer vertex measurements, improving the understanding of multi-nucleon and final state interaction processes. The TPC coverage is designed to be near-hermetic, matching the kinematic phase space of the far detector.