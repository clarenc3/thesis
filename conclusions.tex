\chapter{Conclusion and Remarks}
\label{chap:conclusion}
The work in this thesis presented a Bayesian Markov Chain Monte Carlo method for reducing systematic uncertainties at the long baseline neutrino oscillation experiment Tokai to Kamioka (T2K). The constraints on systematics came from external sources, such as hadron production and neutrino cross-section experiments, and the internal near-detector data primarily from the INGRID and ND280 detectors.

The impact of uncertainties on the event rates predicted at the Super-Kamiokande reduced from 12-14\% to 2-4\%, enabling world-leading constraints on multiple oscillation parameters, which would otherwise be unachievable at T2K. The predictability of the model after the new constraints delivered p-values above 5\%, with the exception of one of fourteen selections. The 2017 analysis presented herein was used in the official 2017 T2K results and those presented at Neutrino 2018. The 2018 analysis introduced new ND280 selections and almost doubled the near-detector data, and will be used in analyses after 2018. A future sensitivity study---using the full projected 7.8E21 protons-on-target---found uncertainties on the 1-2\% scale, close to fulfilling the requirements of the DUNE and Hyper-K experiments.

The analysis discussed particular problems, often centered on unsatisfactory interaction modelling and parameter values disagreeing with the priors from external data. The multi-particle ``CC Other'' selection at ND280 was found to be poorly modelled in both analyses, with p-values of 0.000. In the 2018 analysis, the impact of the ND280 systematics parameterisation was also assessed and numerous compatibility studies were made. Results were found to be compatible within $1\sigma$, with possible hints at unmodelled neutrino/anti-neutrino differences, which were found to have large effects at SK.

Looking ahead, the analysis will in the near future benefit from updated modelling. There will be new models for CCQE (z-expansion\cite{z-exp}), multi-nucleon\cite{nieves1}), and single pion production\cite{thesis_minoo}. Using NA61/SHINE replica target data, the prior flux uncertainties are expected to be reduced by $\sim50\%$\cite{flux_red}.

At the current ND280, efforts are underway to include high and backwards angle FGD selections which include ECal PID information, expanding analysis phase space and increasing statistics by $\sim20-30\%$. Furthermore, dedicated $\nu_e$ selections are being developed to help inform $\nu_e/\nu_\mu$ differences. Lastly, including P0D selections will contribute a large increase in events ($\sim100\%$ of FGD statistics), especially on water interactions.

Finally, the ND280 upgrades\cite{t2k_upgrades} for the T2K-II project\cite{t2k_ii} will add another fine grained detector with a cuboid design, replacing the P0D. This will enable much finer vertex measurements, leading to a better understanding of multi-nucleon and final state interaction processes. The TPC coverage is also extended to be near-hermetic, matching the phase space of SK.
