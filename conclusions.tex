\chapter{Conclusion and Remarks}
\label{chap:conclusion}
The work in this thesis presented a Bayesian Markov Chain Monte Carlo method for reducing systematic uncertainties at the long baseline neutrino oscillation experiment Tokai to Kamioka (T2K). The constraints on systematics came from external sources, such as hadron production and neutrino cross-section experiments, as well as internal near-detector data from the INGRID and ND280 detectors. The analysis provides T2K with a simultaneous near and far detector fit, expanding on the oscillation analyses which use the results of a frequentist ND280-only fit to constrain the systematics. This method avoids making assumptions on the probability density functions, scales considerably better with parameter dimensionality, and does not suffer when there are local minima or discontinuities in the likelihoods.

The uncertainties on the predicted event rates at Super-Kamiokande reduced from 12-14\% to 2-4\% in this thesis, enabling world-leading constraints on multiple oscillation parameters, otherwise be unachievable at T2K. The predictability of the model after the new constraints delivered p-values above 5\%, with the exception of one of fourteen selections. The first analysis presented herein was used in the official 2017 T2K publications and was presented at Neutrino 2018. The second analysis introduced new ND280 selections and used almost double the near-detector data, and will be used in analyses after 2018. A future sensitivity study---using the full projected 7.8E21 protons-on-target---found uncertainties on the far-detector event rates at the 1-2\% scale, fulfilling the requirements of the future long baseline neutrino oscillation experiments DUNE and Hyper-K experiments.

The thesis discussed particular analysis problems, often centered on unsatisfactory interaction modelling and parameter values disagreeing with the priors from external data. The multi-particle ``CC Other'' selection at ND280 was found to be poorly modelled in both analyses, with p-values of 0.000. Numerous compatibility studies were found to have a negligible effect on the event spectra at Super-Kamiokande. In the second analysis, the impact of the ND280 systematics parametrisation was also assessed and found to be compatible within $1\sigma$, with possible hints at unmodelled neutrino/anti-neutrino differences, which had the largest effect on event spectra at SK.

Looking ahead, the analysis will in the near future benefit from updated modelling. There will be new models for CCQE (z-expansion\cite{z-exp}), multi-nucleon\cite{nieves1}), and single pion production\cite{thesis_minoo}. Using NA61/SHINE replica target data, the prior flux uncertainties are expected to be reduced by $\sim50\%$\cite{flux_red}. Efforts are also underway to include high and backwards angle FGD selections which include ECal PID information, expanding kinematic phase space and increasing statistics by about 20-30\%. Furthermore, dedicated $\nu_e$ selections are being developed to help inform $\nu_e/\nu_\mu$ differences. Lastly, including P0D selections will contribute a 100-150\% increase in statistics, especially on neutrino-water interactions.

Finally, the ND280 upgrades\cite{t2k_upgrades} for the T2K-II project\cite{t2k_ii} will add another fine grained detector with a cuboid design, replacing the P0D. This will enable much finer vertex measurements, leading to a better understanding of multi-nucleon and final state interaction processes. The TPC coverage is also extended to be near-hermetic, matching the kinematic phase space of SK.