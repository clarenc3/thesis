\chapter{Conclusion and Remarks}
\label{chap:conclusion}
The work in this thesis presented a Bayesian Markov Chain Monte Carlo method for reducing systematic uncertainties at the long baseline neutrino oscillation experiment Tokai to Kamioka (T2K). The constraints on systematics come from external sources, such as hadron production and neutrino cross-section experiments, and the internal near-detector data from the INGRID and ND280 detectors.

The impact of uncertainties on the event rates predicted at the Super-Kamiokande reduces from 12-14\% to 2-4\%, enabling world-leading constraints on multiple oscillation parameters, otherwise impossible. The predictability of the model after the new constraints delived p-values above 5\%, with the exception of one selection. The 2017 analysis using T2K run 2-6 data were used in the official 2017 results and analyses presented at Neutrino 2018. The 2018 analysis introduced new ND280 selections and almost doubled the near-detector data, and will be used in analyses after 2018.

The analysis discussed particular problems, often centered on unsatisfactory interaction modelling and parameter values disagreeing with the priors from external data. The multi-particle ``CC Other'' selection at ND280 was found to be poorly modelled in both analyses, with p-values of 0.0. In the 2018 analysis, the impact of the ND280 systematics parameterisation was also assesed. Numerous compataibility studies were made and results were found to be compatible within $1\sigma$, with possible hints at unmodelled neutrino/anti-neutrino differences.

Looking ahead, the analysis will benefit from updated interaction models for CCQE (z-expansion\cite{z-exp}), multi-nucleon (2p2h\cite{nieves1}), and single pion production models\cite{thesis_minoo}. Using NA61/SHINE replica target data, the flux uncertainty priors are expected to reduce by $\sim50\%$\cite{flux_red}.

At ND280, efforts are underway to include high and backwards angle selections including ECal PID information, expanding analysis phase space and increasing statistics by $\sim25\%$. Dedicated $\nu_e$ selections are being developed to help inform $\nu_e/\nu_\mu$ differences. Lastly, selections including the P0D will contribute a large increase in statistics (50-100\%), especially on water interactions.

Lastly, the ND280 upgrades\cite{t2k_upgrades} for the second phase T2K-II project\cite{t2k_ii} will add another fine grained detector with a cuboid design. This will enable much finer vertex measurements, leading to better understanding of multi-nucleon and final state interaction processes.